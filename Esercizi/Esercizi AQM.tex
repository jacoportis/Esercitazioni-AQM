\documentclass[openany,12pt]{article}
\usepackage[utf8]{inputenc}
\usepackage[letterpaper,top=2cm,bottom=2cm,left=2.5cm,right=2.5cm,marginparwidth=1.75cm]{geometry}
\usepackage{wrapfig}
\usepackage[psdextra, colorlinks=true, allcolors=black]{hyperref}
\usepackage[italian]{babel}
\usepackage{afterpage}
\newcommand\blankpage{%
    \null
    \thispagestyle{empty}%
    \newpage} %serve a lasciare una pagina vuota
\usepackage{import}
\usepackage{physics}
\usepackage[version=4]{mhchem}
\usepackage{amsfonts}
\usepackage{mathtools}
\usepackage{graphicx}
\usepackage{amssymb}
\usepackage{amsmath}
\usepackage{physics}
\usepackage{enumitem}
\usepackage{array}
\usepackage{tikz}
\usetikzlibrary{snakes}
\usetikzlibrary{shapes}
\usetikzlibrary{decorations.pathmorphing,decorations.markings,calc} % for random steps & snake
\usetikzlibrary{arrows.meta} % for arrow size
\tikzstyle{radiation}=[-{Latex[length=2.3,width=1.8]},red!95!black!50,,thin,decorate,
                       decoration={snake,amplitude=1,segment length=2.5,post length=2.5}]
\usetikzlibrary{decorations.pathmorphing}
\usetikzlibrary{decorations.markings}
\usetikzlibrary{positioning}
\usetikzlibrary{calc}
\usetikzlibrary{shapes.misc}
\usetikzlibrary{snakes}
\usetikzlibrary{patterns}
\usepackage{pgfplots}
\pgfplotsset{compat=newest}

%servono a mettere i simboli greci nei titoli
\usepackage[open]{bookmark}
\ProvidesFile{puenc-greek.def}
\usepackage{textgreek}

\usepackage{float}
\setlength\parindent{0pt}%e si gode, toglie lo spostmento a destra di una nuova riga
\usepackage{caption}
\usepackage{subcaption}
\usepackage{fancyhdr}

\newcommand{\comment}[1]{}

\usepackage{ambienti} %pacchetto dove sono definiti i vari ambienti esercizio, approfondimento ecc

%simboli
\newcommand{\E}{È 
\hspace{0.1mm}}

\newcommand{\A}{Å 
\hspace{0.1mm}}

\begin{document}

\thispagestyle{empty}
\begin{center}

\begin{minipage}[c]{0.45\textwidth}
\begin{flushleft}
\includegraphics[width=0.8\textwidth]{logo-unict-orizzontale-grigio.png}
\end{flushleft}
\end{minipage}
\hfill
\begin{minipage}[c]{0.45\textwidth}
\begin{flushright}
\includegraphics[width=\textwidth]{logo_dfa_orizzontale}
\end{flushright}
\end{minipage}\\
\medskip
\hbox to \textwidth{\hrulefill}

\vfill
\vfill

\uppercase{\sc{ \Large{\textbf{Esercizi di meccanica quantistica avanzata}}}}\\


\vfill
\large{A cura di Joey Butchers}

\vfill
\vfill
\hbox to \textwidth{\hrulefill}
{\sc Anno Accademico 2024-2025}
\end{center}

\afterpage{\blankpage}
\newpage

\section{Teoria perturbativa}
\import{Tipologie/}{Teoria_perturbativa}

\newpage

\section{Approssimazione WKB}
\import{Tipologie/}{Approssimazione_WKB}

\newpage

%\section{Teoria dello scattering}
%\import{Esercizi/Tipologie/}{Teoria_dello_scattering}

\end{document}
