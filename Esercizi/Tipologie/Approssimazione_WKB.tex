\begin{esercizio}
   An electron is initially in the ground state of a one-dimensional infinite-well potential of length $2L$, centered at $x = 0$. At time $t = 0$, the system is perturbed by the potential:
   \begin{equation*}
      V(x) =
      \begin{cases}
         \alpha |x| & \text{if } |x| < L\\
         0 & \text{if } |x| > L
      \end{cases}
   \end{equation*}
   with $\alpha > 0$, for a time $T$.
   \begin{enumerate}[label=\alph*), leftmargin=0.6cm]
      \item Compute the transition probability to the second excited state at time $t>T$ using first-order perturbation theory. Evaluate this probability at the $T$ that maximizes it, assuming $L=1$ nm and $a=0.25$ eV/nm.
      \item Is the probability of transition to the first excited state larger or smaller than the one computed in a)? Justify your answer.
      \item Use the WKB approximation to determine the energy levels below $E=\alpha L$ for $0<t<T$. How many such levels exist for the given values of $L$ and $\alpha$ as in a)?
   \end{enumerate}
   Hint:
   \begin{equation*}
      \int_{0}^{L} \dd{x} \cos{ \qty( \frac{3 \pi x}{2 L} ) } \cos{ \qty( \frac{\pi x}{2 L} ) } x
      =-\frac{L^2}{\pi^2}
   \end{equation*}
\end{esercizio}
\begin{soluzione}
   Svolgiamo il punto a). Dobbiamo trovare la probabilità di transizione dal ground state al secondo stato eccitato. Per fare ciò ci serve l'ampiezza di transizione $c_{i \to f}$, che in questo caso è data da
   \begin{equation*}
      c_{1 \to 3}
      =-\frac{i}{\hbar} \int_{0}^{T} \dd{t} V_{3,1}(t) e^{i \omega_{3,1} t}
   \end{equation*}
   Calcoliamo innanzitutto $V_{3,1}(t)$. Esso è dato da
   \begin{equation*}
      V_{3,1}(t)
      =\mel{\phi_3}{V}{\phi_1}
      =\mel{\phi_3}{\alpha |x|}{\phi_1}
      =\alpha \int_{-\infty}^{\infty} \dd{x} \psi_3^*(x) \psi_1(x) |x|
      =\alpha \int_{-L}^{L} \dd{x} \psi_3^*(x) \psi_1(x) |x|
   \end{equation*}
   in quanto le funzioni d'onda sono nulle al di fuori dell'intervallo $[-L,L]$.\\
   In questo caso, visto che la buca si estende da $-L$ ad $L$, le funzioni d'onda sono date da
   \begin{equation*}
      \psi_n(x)
      =
      \begin{cases}
         \displaystyle \sqrt{\frac{1}{L}}\cos{ \qty( \frac{n \pi x}{2L} ) } & \text{per $n$ dispari}\\[0.3cm]
         \displaystyle \sqrt{\frac{1}{L}}\sin{ \qty( \frac{n \pi x}{2L} ) } & \text{per $n$ pari}
      \end{cases}
   \end{equation*}
   Sostituendo allora le funzioni d'onda, otteniamo
   \begin{equation*}
      V_{3,1}(t)
      =\frac{\alpha}{L} \int_{-L}^{L} \dd{x} \cos{ \qty( \frac{3 \pi x}{2L} ) } \cos{ \qty( \frac{\pi x}{2L} ) } |x|
   \end{equation*}
   Se adesso sfruttiamo il fatto che per per una funzione pari $f(x)$ vale
   \begin{equation*}
      \int_{-a}^{a} \dd{x} f(x)
      =2 \int_{0}^{a} \dd{x} f(x)
   \end{equation*}
   Possiamo scrivere
   \begin{equation*}
      V_{3,1}(t)
      =\frac{2\alpha}{L} \int_{0}^{L} \dd{x} \cos{ \qty( \frac{3 \pi x}{2L} ) } \cos{ \qty( \frac{\pi x}{2L} ) } x
      =\frac{2\alpha}{L} \qty( -\frac{L^2}{\pi^2} )
      =-\frac{2\alpha L}{\pi^2}
   \end{equation*}
   avendo usato usato il suggerimento fornito dal testo.\\
   Calcoliamo adesso $\omega_{3,1}$. Ricordiamo che per la buca quadra le energie sono date da
   \begin{equation*}
      E_n
      =\frac{\hbar^2 \pi^2}{8 m L^2} n^2
      \qq{,}
      n=1,2,\ldots
   \end{equation*}
   Quindi
   \begin{equation*}
      \omega_{3,1}
      =\frac{E_3-E_1}{\hbar}
      =\frac{8 \hbar^2 \pi^2}{8 m L^2}
      =\frac{\hbar^2 \pi^2}{m L^2}
   \end{equation*}
   A questo punto possiamo calcolare l'ampiezza di transizione, che sarà data da
   \begin{equation*}
      \begin{split}
         c_{1 \to 3}
         & =-\frac{i}{\hbar} \int_{0}^{T} \dd{t} V_{3,1}(t) e^{i \omega_{3,1} t}
         =\frac{2 i \alpha L}{\hbar \pi^2} \int_{0}^{T} \dd{t} e^{i \omega_{3,1} t}
         \\[0.2cm]
         & =\frac{2 i \alpha L}{\hbar \pi^2} \frac{1}{i \omega_{3,1}} \qty[ e^{i \omega_{3,1} t} ]_{0}^{T}
         =\frac{2 \alpha L}{\hbar \pi^2} \frac{1}{\omega_{3,1}} \qty( e^{i \omega_{3,1} T} - 1 )
      \end{split}
   \end{equation*}
   Sostituendo le espressioni trovata per $\omega_{3,1}$ e osservando che
   \begin{equation*}
      e^{i\alpha} - 1
      =2i e^{i \frac{\alpha}{2}} \frac{\qty( e^{i \frac{\alpha}{2}} - e^{-i \frac{\alpha}{2}} )}{2i}
      =2i \sin{ \qty( \frac{\alpha}{2} )} e^{i \frac{\alpha}{2}}
   \end{equation*}
   otteniamo (nel nostro caso $\alpha=\omega_{3,1}T$)
   \begin{equation*}
      c_{1 \to 3}
      =\frac{4 i \alpha m L^3}{\hbar^2 \pi^4} \sin{ \qty( \frac{\omega_{3,1} T}{2} )} e^{i \frac{\omega_{3,1} T}{2}}
   \end{equation*}
   e dunque la probabilità di transizione sarà data da
   \begin{equation*}
      P_{1 \to 3}
      =|c_{1 \to 3}|^2
      =\frac{16 \alpha^2 m^2 L^6}{\hbar^4 \pi^8} \sin^2{ \qty( \frac{\omega_{3,1} T}{2} )}
   \end{equation*}
   In particolare, la probabilità di transizione sarà massima quando la funzione seno assume il suo valore massimo, cioè 1. In corrispondenza di tale valore si avrà
   \begin{equation*}
      P_{1 \to 3}^{\rm max}
      =\frac{16 \alpha^2 {(mc^2)}^2 L^2}{(\hbar c)^4 \pi^8}
      =\rm \frac{16 \cdot 0.25^2 \; eV^2 \, nm^{-2} \cdot 0.5^2 \cdot 10^{12} \; eV \cdot 1 \; nm^6}{2^4 \cdot 10^8 \; eV^4 \, nm^4 \cdot \pi^8}
   \end{equation*}
\end{soluzione}