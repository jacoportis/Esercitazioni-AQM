\begin{esercizio}
   A particle in a one-dimensional harmonic oscillator is subject to the potential
   \begin{equation*}
      V'(t)=\hbar\omega \frac{\alpha^3 x^3 - 1}{1 + (t/\tau)^2}
      \qq{with}
      \alpha=\qty(\frac{m \omega}{\hbar})^{\frac{1}{2}}
   \end{equation*}
   Assuming the particle is in the ground state at $t=-\infty$:
   \begin{enumerate}[label=\alph*), leftmargin=0.6cm]
      \item Determine all the possible transitions allowed at first order of perturbation theory.
      \item Calculate the probability\footnotemark\;of such transitions (excluding $0 \to 0$) for $\omega=10^{15} \rm \; s^{-1}$ and $\tau=10^{-15} \rm \; s$.
      \item Is perturbation theory applicable for case b)? Briefly explain your answer.
   \end{enumerate}
\end{esercizio}
\begin{soluzione}
   \footnotetext{Il testo non lo dice esplicitamente, ma richiede di calcolare la probabilità di transizione al tempo $t=+\infty$.}
   Si tratta di un problema di teoria perturbativa dipendente dal tempo.\\
   Svolgiamo il punto a). Per prima cosa dobbiamo determinare le transizioni permesse. Ricordiamo innanzitutto che il potenziale dell'oscillatore armonico è
   \begin{equation*}
      V(x)=\frac{1}{2}m\omega^2 x^2
   \end{equation*}
   mentre i livelli di energia sono dati da
   \begin{equation*}
      E=\hbar\omega \qty(n + \frac{1}{2})
      \qq{,}
      n=0,1,2,\ldots
   \end{equation*}
   Inoltre possiamo scrivere l'operatore posizione in termini degli operatori di creazione e annichilazione come
   \begin{equation*}
      x=\sqrt{\frac{\hbar}{2m\omega}} \qty( a + a^{\dag} )
   \end{equation*}
   Gli operatori di creazione annichilazione agiscono sugli autostati dell'oscillatore armonico nel seguente modo:
   \begin{gather*}
      a \ket*{n}
      =\sqrt{n} \ket*{n - 1}
      \qq{,}
      a \ket*{0}=0
      \\
      a^{\dag} \ket*{n}
      =\sqrt{n + 1} \ket*{n + 1}
   \end{gather*}
   L'ampiezza di transizione dal ground state $\ket*{0}$ allo stato finale $\ket*{f}$ è data da
   \begin{equation*}
      d_{0 \to f}
      =-\frac{i}{\hbar} \int_{-\infty}^{+\infty} \dd{t} e^{i \omega_{f,0} t} V'_{f,0}(t)
   \end{equation*}
   dove
   \begin{equation*}
      \omega_{f,0}
      =\frac{E_f - E_0}{\hbar}
      \qq{,}
      V'_{f,0}
      =\mel*{f}{V'}{0}
   \end{equation*}
   Per capire quali sono le transizioni permesse dobbiamo andare a vedere quali sono quelle aventi ampiezza di transizione non nulla. Tale condizione implica che deve essere non nullo l'elemento di matrice $V'_{f,0}$. Esplicitamente, tale condizione significa che
   \begin{gather*}
      \mel*{f}{V'}{0}
      =\frac{\hbar\omega}{1 + (t/\tau)^2} \mel*{f}{\alpha^3 x^3 - 1}{0}
      \neq 0
      \\[0.1cm]
      \implies
      \mel*{f}{\alpha^3 x^3}{0} - \braket*{f}{0}
      \neq 0
   \end{gather*}
   Il termine $\braket*{f}{0}$ è diverso da 0 quando quando $\ket*{f}=\ket*{0}$, ma il testo diceva di escludere le transizioni $0 \to 0$, per cui bisogna soltanto ricavare in che caso si ha $\mel*{f}{\alpha^3 x^3}{0} \neq 0$.\\
   Se adesso esprimiamo $x$ in termini degli operatori di creazione e annichilazione ma utilizzando la definizione di $\alpha$ fornita dal testo, abbiamo
   \begin{equation*}
      x=\frac{1}{\sqrt{2} \alpha} \qty( a + a^{\dag} )
   \end{equation*}
   e dunque
   \begin{equation*}
      x^3
      =\frac{1}{2^{\frac{3}{2}} \alpha^3} \qty( a + a^{\dag} )^3
   \end{equation*}
   In questo modo la condizione da imporre diventa
   \begin{equation*}
      \frac{1}{2^{\frac{3}{2}}} \mel*{f}{\qty( a + a^{\dag} )^3}{0}
      \neq 0
   \end{equation*}
   Osserviamo che possiamo riscrivere l'operatore come
   \begin{equation*}
      \qty( a + a^{\dag} )^3
      =\qty( a + a^{\dag} ) \bigl( a^2 + {a^{\dag}}^2 + aa^{\dag} + a^{\dag}a \bigr)
   \end{equation*}
   Piuttosto che svolgere questo prodotto e vedere poi che cosa si ottiene dall'applicazione di ciascun termine a $\ket*{0}$, possiamo già fare una considerazione: ogni volta che applichiamo un prodotto di operatori che ha come ultimo termine $a$ allo stato $\ket*{0}$, esso darà $0$, dunque possiamo trascurarli. Nel nostro caso operatori del genere sono $a^2$ e $a^{\dag} a$, che per quanto detto posso essere omessi. In definitiva, possiamo considerare soltanto l'operatore
   \begin{equation*}
      \qty( a + a^{\dag} ) \bigl( {a^{\dag}}^2 + aa^{\dag} \bigr)
      =a{a^{\dag}}^2 + a^2 a^{\dag} + {a^{\dag}}^3 + a^{\dag} a a^{\dag}
   \end{equation*}
   Osserviamo che è possibile scartare anche il termine $a^2 a^{\dag}$ in quanto applicandolo a $\ket*{0}$ si ha
   \begin{equation*}
      a^2 a^{\dag}\ket*{0}
      =a^2 \ket*{1}
      =a \ket*{0}
      =0
   \end{equation*}
   Vediamo allora cosa danno gli altri termini:
   \begin{equation*}
      \begin{split}
         a{a^{\dag}}^2 \ket*{0}
         &
         =a a^{\dag} \ket*{1}
         =\sqrt{2} a \ket*{2}
         =2 \ket*{1}
         \\
         {a^{\dag}}^3 \ket*{0}
         &
         ={a^{\dag}}^2 \ket*{1}
         =\sqrt{2} a^{\dag} \ket*{2}
         =\sqrt{6} \ket*{3}
         \\
         a^{\dag} a a^{\dag} \ket*{0}
         &
         =a^{\dag} a \ket*{1}
         =a^{\dag} \ket*{0}
         =\ket*{1}
         \\
      \end{split}
   \end{equation*}
   In definitiva abbiamo che
   \begin{equation*}
      \mel*{f}{\qty( a + a^{\dag} )^3}{0}
      =3\braket*{f}{1} + \sqrt{6} \braket*{f}{3}
      =3 \delta_{f,1} + \sqrt{6} \delta_{f,3}
   \end{equation*}
   dove nell'ultimo passaggio abbiamo usato l'ortonomalità degli autostati dell'oscillatore armonico.\\
   Dunque le transizioni permesse sono $0 \to 1$ e $0 \to 3$, in quanto sono le uniche per cui tale termine è non nullo. Notiamo che non è permessa la transizione $0 \to 2$, come ci si poteva aspettare per questioni di simmetria. Infatti, l'interazione è dispari, poiché dipende da $x^3$, mentre gli stati $\ket*{0}$ e $\ket*{2}$ sono pari, dunque l'elemento di matrice corrispondente sarà nullo.\\
   Passiamo al quesito b). Dobbiamo calcolare il valore delle probabilità di tali transizioni in corrispondenza di certi valori di $\omega$ e $\tau$.\\
   In termini delle transizioni permesse, $V'_{f,0}$ si scrive come
   \begin{equation*}
      V'_{f,0}
      =\frac{\hbar\omega}{1 + (t/\tau)^2} \frac{1}{2^{\frac{3}{2}}} \bigl( 3 \delta_{f,1} + \sqrt{6} \delta_{f,3} \bigr)
   \end{equation*}
   Distinguiamo i due casi:
   \begin{itemize}[leftmargin=0.5cm]
      \item Transizione $0 \to 1$. In questo caso abbiamo che
      \begin{equation*}
         \omega_{1,0}
         =\frac{E_1 - E_0}{\hbar}
         =\frac{\qty( 1 + \frac{1}{2} ) \hbar\omega - \frac{1}{2} \hbar\omega }{\hbar}
         =\omega
      \end{equation*}
   quindi l'ampiezza di transizione sarà data da
   \begin{equation*}
      d_{0 \to 1}
      =-\frac{i}{\hbar} \int_{-\infty}^{+\infty} \dd{t} e^{i \omega t} \frac{3}{2^{\frac{3}{2}}} \frac{\hbar\omega}{1 + (t/\tau)^2}
      =-i \frac{3}{2^{\frac{3}{2}}} \omega \tau^2 \int_{-\infty}^{+\infty} \dd{t} \frac{e^{i \omega t}}{t^2 + \tau^2}
   \end{equation*}
   Per risolvere quest'integrale applichiamo il metodo dei residui.
   \begin{figure}[H]
      \centering
      \begin{tikzpicture}
            %dominio
            \filldraw[gray!40!] (2,0) arc (0:180:2cm) -- cycle;
            %assi
            \draw[->] (-2.3,0) -- (2.3,0) node[right] {$\Re{z}$};
            \draw[->] (0,-1.5) -- (0,2.4) node[above] {$\Im{z}$};
            %singolarità
            \filldraw[red] (0,1) circle (1.2pt) node[right] {\footnotesize$+i\tau$};
            \filldraw[red] (0,-1) circle (1.2pt) node[right] {\footnotesize$-i\tau$};
            \begin{scope}[decoration={markings,mark=at position 0.2 with {\arrow{>}}, mark=at position 0.45 with {\arrow{>}},mark=at position 0.7 with {\arrow{>}}, mark=at position 0.9 with {\arrow{>}}}] 
               \draw[postaction={decorate}] (2,0) arc (0:180:2cm) -- cycle;
            \end{scope}
      \end{tikzpicture}
   \end{figure}
   In particolare, la funzione integranda ha due poli in $t=\pm i \tau$, ma poiché dobbiamo prendere il contorno superiore il teorema dei residui ci dà
   \begin{gather*}
      \int_{-\infty}^{+\infty} \dd{t} \frac{e^{i \omega t}}{t^2 + \tau^2}
      =2\pi i \underset{t=i\tau}{\Res} \qty[ \frac{e^{i \omega t}}{t^2 + \tau^2} ]
      \\
      =2\pi i \lim_{z \to i\tau} (z - i\tau) \frac{e^{i \omega z}}{(z + i\tau)(z - i\tau)}
      =2\pi i \frac{e^{-\omega \tau}}{2i\tau}
      =\frac{\pi e^{-\omega \tau}}{\tau}
   \end{gather*}
   In definitiva
   \begin{equation*}
      d_{0 \to 1}
      =-i \frac{3}{2^{\frac{3}{2}}} \omega \tau^2 \frac{\pi e^{-\omega \tau}}{\tau}
      =-i\frac{3}{2^{\frac{3}{2}}} \omega \tau \pi e^{-\omega \tau}
   \end{equation*}
   e quindi la probabilità di transizione da $\ket*{0}$ a $\ket*{1}$ sarà pari a
   \begin{equation*}
      P_{0 \to 1}
      =| d_{0 \to 1} |^2
      =\frac{9 \pi^2}{8} (\omega \tau)^2 e^{-2\omega \tau}
   \end{equation*}
   \item Transizione $0 \to 3$. In questo caso abbiamo che
   \begin{equation*}
      \omega_{3,0}
      =\frac{E_3 - E_0}{\hbar}
      =\frac{\qty( 3 + \frac{1}{2} ) \hbar\omega - \frac{1}{2} \hbar\omega }{\hbar}
      =3\omega
   \end{equation*}
   quindi l'ampiezza di transizione sarà data da
   \begin{equation*}
      d_{0 \to 3}
      =-\frac{i}{\hbar} \int_{-\infty}^{+\infty} \dd{t} e^{i 3\omega t} \frac{\sqrt{6}}{2^{\frac{3}{2}}} \frac{\hbar\omega}{1 + (t/\tau)^2}
      =-i \frac{\sqrt{6}}{2^{\frac{3}{2}}} \omega \tau^2 \int_{-\infty}^{+\infty} \dd{t} \frac{e^{i 3\omega t}}{t^2 + \tau^2}
   \end{equation*}
   Procedendo in maniera analoga al caso precedente, il teorema dei residui ci dà
   \begin{gather*}
      \int_{-\infty}^{+\infty} \dd{t} \frac{e^{i 3\omega t}}{t^2 + \tau^2}
      =2\pi i \underset{t=i\tau}{\Res} \qty[ \frac{e^{i 3\omega t}}{t^2 + \tau^2} ]
      \\
      =2\pi i \lim_{z \to i\tau} (z - i\tau) \frac{e^{i 3\omega z}}{(z + i\tau)(z - i\tau)}
      =2\pi i \frac{e^{-3\omega \tau}}{2i\tau}
      =\frac{\pi e^{-3\omega \tau}}{\tau}
   \end{gather*}
   In definitiva
   \begin{equation*}
      d_{0 \to 3}
      =-i \frac{\sqrt{6}}{2^{\frac{3}{2}}} \omega \tau^2 \frac{\pi e^{-3\omega \tau}}{\tau}
      =-i\frac{\sqrt{6}}{2^{\frac{3}{2}}} \omega \tau \pi e^{-3\omega \tau}
   \end{equation*}
   e quindi la probabilità di transizione da $\ket*{0}$ a $\ket*{3}$ sarà pari a
   \begin{equation*}
      P_{0 \to 3}
      =| d_{0 \to 3} |^2
      =\frac{3 \pi^2}{4} (\omega \tau)^2 e^{-6\omega \tau}
   \end{equation*}
\end{itemize}
   Troviamo adesso un valore numerico per tale probabilità con i dati forniti dal testo. Osserviamo che per tali valori si ha $\omega \tau=1$, dunque:
   \begin{gather*}
      P_{0 \to 1}
      =\frac{9 \pi^2}{8} e^{-2}
      \approx 1.5
      \\
      P_{0 \to 3}
      =\frac{3 \pi^2}{4} e^{-6}
      \approx 0.02
   \end{gather*}
   Svolgiamo il punto c). I valori trovati nel quesito b) ci dicono che la teoria perturbativa non è applicabile per la transizione $0 \to 1$, in quanto la probabilità di transizione è addirittura maggiore di 1.\footnote{Tale risultato può essere collegato col fatto che l'espansione perturbativa si fa attorno ad uno stato che deve essere stabile, per cui lo stato $\ket*{1}$ potrebbe non essere stabile in quanto subisce continuamente transizioni verso altri stati e in conseguenza a cià non è valida la teoria perturbativa.} Per la transizione $0 \to 3$, invece, si ha che $P \ll 1$, dunque la teoria perturbativa è valida.
\end{soluzione}

\newpage
\setcounter{equation}{0}

\begin{esercizio}
   Given a proton scattering on the potential
   \begin{equation*}
      V(r)=
      \begin{cases}
         -V_0 & \text{for} \; r \leq a\\
         0 & \text{for} \; r > a
      \end{cases}
   \end{equation*}
   \begin{enumerate}[label=\alph*), leftmargin=0.6cm]
      \item Calculate the differential cross section and the phase shift for $\ell=0$ (not in the low-energy limit).
      \item Considering now the low-energy limit for the phase shift, calculate at what energy (in MeV) $\delta_0=\pi/6$ if $V_0=40 \rm \; MeV$ and $a=1 \rm \; fm$.
      \item Can it be considered low energy?
   \end{enumerate}
   Hint 1:
   \begin{equation*}
      \dv{x} \frac{\sin{x}}{x}
      =\frac{\cos{x}}{x} - \frac{\sin{x}}{x^2}
   \end{equation*}
   Hint 2: Use the approximation $e^{i \delta_{\ell}} \sin{(\delta_{\ell})} \simeq \delta_{\ell}$.
\end{esercizio}
\begin{soluzione}
   Osserviamo innanzitutto che il potenziale è di tipo centrale ed ha range $R_V$ pari ad $a$.\\
   Svolgiamo il quesito a). Per risolverlo usiamo il projection method. Vediamo di cosa si tratta.\\
   L'ampiezza di scattering può essere scritta, usando la partial wave analysis, come
   \begin{equation*}
      f_k(\vartheta)
      =\frac{1}{k} \sum_{\ell=0}^{\infty} (2\ell + 1) e^{i \delta_{\ell}} \sin{(\delta_{\ell})} P_{\ell}(\cos{\theta})
   \end{equation*}
   Il projection method consiste nell'integrare tra $-1$ e $1$ rispetto a $\cos{\vartheta}$ ambo i membri di tale relazione moltiplicati per $P_{\ell'}(\cos{\vartheta})$ con $\ell \neq \ell'$. In formule:
   \begin{equation*}
      \int_{-1}^{1} \dd{(\cos{\vartheta})} f_k(\vartheta) P_{\ell'}(\cos{\vartheta})
      =\frac{1}{k} \sum_{\ell=0}^{\infty} (2\ell + 1) e^{i \delta_{\ell}} \sin{(\delta_{\ell})} \int_{-1}^{1} \dd{(\cos{\vartheta})} P_{\ell}(\cos{\vartheta}) P_{\ell'}(\cos{\vartheta})
   \end{equation*}
   Per la relazione di ortogonalità tra i polinomi di Legendre si ha che
   \begin{equation*}
      \int_{-1}^{1} \dd{(\cos{\vartheta})} P_{\ell}(\cos{\vartheta}) P_{\ell'}(\cos{\vartheta})
      =\frac{2}{2\ell + 1} \delta_{\ell,\ell'}
   \end{equation*}
   dunque possiamo riscrivere il membro di destra come
   \begin{equation*}
      \frac{2}{k} \sum_{\ell=0}^{\infty} e^{i \delta_{\ell}} \sin{(\delta_{\ell})} \delta_{\ell,\ell'}
      =\frac{2}{k} e^{i \delta_{\ell'}} \sin{(\delta_{\ell'})}
   \end{equation*}
   In definitiva abbiamo
   \begin{equation*}
      \int_{-1}^{1} \dd{(\cos{\vartheta})} f_k(\vartheta) P_{\ell'}(\cos{\vartheta})
      =\frac{2}{k} e^{i \delta_{\ell'}} \sin{(\delta_{\ell'})}
   \end{equation*}
   Se sfruttiamo il suggerimento del testo di approssimare $^{i \delta_{\ell}} \sin{(\delta_{\ell})}$ con $\delta_{\ell}$, possiamo scrivere
   \begin{equation}
      \delta_{\ell}
      =\frac{k}{2} \int_{-1}^{1} \dd{(\cos{\vartheta})} f_k(\vartheta) P_{\ell}(\cos{\vartheta})
      \label{eq:delta_ell_metodo_proiezione}
   \end{equation}
   Prima di calcolare i phase shifts dobbiamo calcolare l'ampiezza di scattering. Nel fare ciò osserviamo che l'approssimazione che suggerisce il testo è applicabile quando il potenziale è debole, condizione in cui è applicabile l'approssimazione di Born. Possiamo allora adoperare quest'ultima per calcolare $f_k(\vartheta)$, che sarà quindi data da
   \begin{equation*}
      f_k(\vartheta)
      =-\frac{2m}{\hbar^2} \frac{1}{q} \int_{0}^{\infty} \dd{r} r V(r) \sin{(qr)}
   \end{equation*}
   dove $q=|\vb{k} - \vb{k}'|=2k \sin{(\vartheta/2)}$.\\ Sostituendo adesso l'espressione del potenziale si ha\footnote{Si noti che quando abbiamo a che fare con l'ampiezza di scattering, troviamo sempre il termine $\frac{2m V_0}{\hbar^2}$, per cui la sua presenza ci dice che stiamo svolgendo correttamente i calcoli.}
   \begin{equation*}
      f_k(\vartheta)
      =\frac{2m V_0}{\hbar^2 q^2} \int_{0}^{a} \dd{r} r \sin{(qr)}
   \end{equation*}
   Svolgendo l'integrale si ottiene
   \begin{equation*}
      f_k(\vartheta)
      =\frac{2m V_0 a}{\hbar^2 q^2} \qty[ \frac{\sin{(qa)}}{qa} - \cos{(qa)} ]
   \end{equation*}
   Quindi la sezione d'urto differenziale sarà
   \begin{equation*}
      \dv{\sigma}{\Omega}
      =| f_k(\vartheta) |^2
      =\qty( \frac{2m V_0 a}{\hbar^2 q^2} )^2 \qty[ \frac{\sin{(qa)}}{qa} - \cos{(qa)} ]^2
   \end{equation*}
   Per inciso, è possibile scrivere tale sezione d'urto come il prodotto della sezione d'urto di Rutherford per un fattore di forma $F(q^2)$:
   \begin{equation*}
      \dv{\sigma}{\Omega}
      =\qty( \dv{\sigma}{\Omega} )_{\rm Ruth} F(q^2)
      =\qty( \frac{2m e}{\hbar^2 q^2} )^2 \qty( \frac{ V_0 a}{e^2} )^2 \qty[ \frac{\sin{(qa)}}{qa} - \cos{(qa)} ]^2
   \end{equation*}
   Calcoliamo ora il phase shift per $\ell=0$. Mediante l'equazione \eqref{eq:delta_ell_metodo_proiezione} si ha ($P_0(\cos{\vartheta})=1$)
   \begin{equation*}
      \begin{split}
         \delta_0
         & =\frac{k}{2} \int_{0}^{\pi} \dd{\vartheta} \sin{\vartheta} f_k(\vartheta)
         \\
         & =\frac{k}{2} \int_{0}^{\pi} \dd{\vartheta} \sin{\vartheta} \frac{2m V_0 a}{\hbar^2 4 k^2 \sin^2{(\vartheta/2)}} \qty{ \frac{ \sin{ \bigl[ 2ka \sin{(\vartheta/2)} \bigr] } }{ 2ka \sin{ (\vartheta/2) } } - \cos{ \bigl[ 2ka \sin{(\vartheta/2)} \bigr] } }
      \end{split}
   \end{equation*}
   Per risolvere tale integrale è conveniente scrivere
   \begin{equation*}
      \sin{\vartheta}
      =2 \sin{(\vartheta/2)} \cos{(\vartheta/2)}
   \end{equation*}
   in modo da poter operare la sostituzione
   \begin{equation*}
      y=2ka \sin{(\vartheta/2)}
      \implies
      \dd{y}=ka \cos{(\vartheta/2)} \dd{\vartheta}
   \end{equation*}
   e calcolare l'integrale in $y$ (in cui $y(0)=0$ e $y(\pi)=2ka$):
   \begin{equation*}
      \delta_0
      =\frac{m V_0 a}{\hbar^2 k} \int_{0}^{2ka} \dd{y} \frac{\sin{y}}{y^2} - \frac{\cos{y}}{y}
   \end{equation*}
   Notiamo che abbiamo ottenuto l'espressione che era nel suggerimento, in quanto possiamo riscrivere
   \begin{equation*}
      \delta_0
      =-\frac{m V_0 a}{\hbar^2 k} \int_{0}^{2ka} \dd{y} \frac{\cos{y}}{y} - \frac{\sin{y}}{y^2}
      =-\frac{m V_0a}{\hbar^2 k} \int_{0}^{2ka} \dd{y} \dv{y} \frac{\sin{y}}{y}
      =-\frac{m V_0 a}{\hbar^2 k} \qty[ \frac{\sin{(2ka)}}{2ka} - 1 ]
   \end{equation*}
   cioè
   \begin{equation}
      \delta_0
      =-\frac{m V_0 a}{\hbar^2 k} \qty[ \frac{\sin{(2ka)}}{2ka} - 1 ]
      \label{eq:delta_0}
   \end{equation}
   Svolgiamo adesso il punto b). considerando il limite di basse energie, dobbiamo calcolare a quale energia, per certi valori di $V_0$ ed $a$, il phase shift associato a $\ell=0$ è pari a $\pi/6$. Osserviamo innanzitutto che in questo problema il limite delle basse energie corrisponde alla condizione $ka \ll 1$, per cui possiamo sviluppare in serie il termine $\sin{(2ka)}$ nell'equazione \eqref{eq:delta_0} fino al secondo ordine, ottenendo\footnote{Il motivo per cui non ci fermiamo al primo ordine è che otterremmo un phase shift nullo.}
   \begin{equation*}
      \delta_0
      =-\frac{m V_0 a}{\hbar^2 k} \qty[ \frac{1}{2ka} \qty( 2ka - \frac{8k^3a^3}{6} ) - 1 ]
      =\frac{2 m V_0 a^3}{3 \hbar^2} ka
   \end{equation*}
   ed esplicitando la dipendenza di $k$ dall'energia otteniamo
   \begin{equation*}
      \delta_0
      =\frac{2 m V_0 a^3}{3 \hbar^3} \sqrt{2mE}
   \end{equation*}
   Elevando adesso entrambi i membri al quadrato si ottiene
   \begin{equation*}
      \delta_0^2
      =\frac{8 m^3 V_0^2 a^6 E}{9 \hbar^6}
   \end{equation*}
   e quindi l'energia cercata sarà
   \begin{equation*}
      E
      =\frac{9 \hbar^6 \delta_0^2}{8 m^3 V_0^2 a^6}
   \end{equation*}
   ovvero, sostituendo i valori numerici forniti dal testo,
   \begin{equation*}
      E
      =\frac{9 (\hbar c)^6 \delta_0^2}{8 {(mc^2)}^3 V_0^2 a^6}
      =\rm \frac{9 \cdot 2^6 \cdot 10^{12} \; MeV^6 \, fm^6 \cdot (\pi/6)^2}{8 \cdot 10^9 \; MeV^3 \cdot 16 \cdot 10^2 \; MeV^2 \cdot 1 \; fm^6}
      \approx 12.3 \rm \; MeV
   \end{equation*}
   Passiamo infine al punto c). Per verificare se ci si trova nel limite delle basse energie, dobbiamo verificare che il prodotto $ka$ sia molto minore di 1. Si ha che
   \begin{equation*}
      ka
      =\frac{\sqrt{2mE}}{\hbar} a
      =\frac{\sqrt{2 mc^2 E}}{\hbar c} a
      =\rm \frac{\sqrt{2 \cdot 10^3 \; MeV \cdot 12.3 \; MeV}}{200 \; MeV \, fm} \cdot 1 \; fm
      \approx 0.8
   \end{equation*}
   Il valore trovato non è molto minore di 1, quindi siamo del tutto nel limite di basse energie.
\end{soluzione}

\newpage
\setcounter{equation}{0}

\begin{esercizio}
   A proton is scattering on the potential
   \begin{equation*}
      V(r)=V_0 \frac{e^{-\alpha r}}{\alpha r}
   \end{equation*}
   It is known that
   \begin{equation*}
      f_k(\vartheta)
      =-\frac{2m V_0}{\alpha \hbar^2} \frac{1}{q^2 + \alpha^2}
   \end{equation*}
   \begin{enumerate}[label=\alph*), leftmargin=0.6cm]
      \item Calculate the phase shift for $\ell=0$ in the low-energy limit (is not said that $\delta_{\ell}$ is small).
      \item Calculate at what energy (in MeV) one has resonant scattering if $V_0=60 \rm \; MeV$ and $\alpha=2/3 \rm \; fm^{-1}$.
      \item Evaluate the width of the resonance.
   \end{enumerate}
   Hint 1:
   \begin{equation*}
      \int_{-1}^{1} \dd{x} \frac{1}{b(1-x) + a^2}
      =\frac{1}{b} \ln\qty( 1 + \frac{2b}{a^2} )
   \end{equation*}
   Hint 2:
   \begin{equation*}
      \ln( 1 + x )
      \simeq x
      \qq{for}
      x \ll 1
   \end{equation*}
   due to the low-energy limit.\\
   Hint 3:
   \begin{equation*}
      \dv{\cot{x}}{x}
      =-\frac{1}{\sin^2{x}}
      \qq{,}
      \dv{\arcsin{x}}{x}
      =\frac{1}{\sqrt{1 - x^2}}
   \end{equation*}
\end{esercizio}
\begin{soluzione}
   Svolgiamo il quesito a). Per calcolare il phase shift è conveniente usare il projection method, dunque usiamo la relazione ($P_0(\cos{\vartheta})=1$)
   \begin{equation*}
      e^{i \delta_0} \sin{(\delta_0)}
      \simeq \frac{k}{2} \int_{-1}^{1} \dd{\cos{(\vartheta)}} \sin{\vartheta} f_k(\vartheta)
   \end{equation*}
   La forma dell'ampiezza di scattering è fornita dal testo, per cui possiamo inserirla direttamente ottenendo
   \begin{equation*}
      \begin{split}
         e^{i \delta_0} \sin{(\delta_0)}
         & =-\frac{k}{2} \frac{2m V_0}{\alpha \hbar^2} \int_{-1}^{1} \dd{(\cos{\vartheta})} \frac{1}{4k^2 \sin^2{( \vartheta/2 )} + \alpha^2}
         \\
         & =-\frac{k}{2} \frac{2m V_0}{\alpha \hbar^2} \int_{-1}^{1} \dd{(\cos{\vartheta})} \frac{1}{2k^2 ( 1 - \cos{\vartheta}) + \alpha^2}
      \end{split}
   \end{equation*}
   dove nel secondo passaggio abbiamo sfruttato la relazione trigonometrica $2 \sin^2{( \vartheta/2 )}=1 - \cos{\vartheta}$.\\
   In questo modo troviamo un integrale nella forma di quello fornito dal suggerimento, per cui possiamo scrivere
   \begin{equation*}
      e^{i \delta_0} \sin{(\delta_0)}
      =-\frac{k}{2} \frac{2m V_0}{\alpha \hbar^2} \frac{1}{2k^2} \ln\qty( 1 + \frac{4k^2}{\alpha^2} )
      =-\frac{m V_0}{2 k \alpha \hbar^2} \ln\qty( 1 + \frac{4k^2}{\alpha^2} )
   \end{equation*}
   Poiché siamo nel limite delle basse energie abbiamo $kR_V \ll 1$, che in questo caso diventa $k/\alpha \ll 1$. Possiamo allora sfruttare il secondo indizio e approssimare il logaritmo come
   \begin{equation*}
      \ln\qty( 1 + \frac{4k^2}{\alpha^2} )
      \simeq \frac{4k^2}{\alpha^2}
   \end{equation*}
   e quindi
   \begin{equation*}
      e^{i \delta_0} \sin{(\delta_0)}
      \simeq -\frac{m V_0}{2 k \alpha \hbar^2} \frac{4k^2}{\alpha^2}
      =-\frac{2 m V_0}{\alpha^2 \hbar^2} \frac{k}{\alpha}
      \label{eq:delta_0_low_energy}
   \end{equation*}
   Calcoliamo adesso il phase shift. Per fare ciò consideriamo il modulo della \eqref{eq:delta_0_low_energy} e otteniamo
   \begin{equation*}
      \qty| e^{i\delta_0} \sin{(\delta_0)} |
      =| \sin{(\delta_0)} |
      =\frac{2 m V_0}{\alpha^2 \hbar^2} \frac{k}{\alpha}
   \end{equation*}
   dunque
   \begin{equation}
      \delta_0
      =\arcsin{\qty( \frac{2 m V_0}{\alpha^3 \hbar^2} k )}
      \label{eq:delta_0_non_small}
   \end{equation}
   Passiamo al punto b). Si ha uno scattering risonante quando in corrispondenza di un processo di scattering si vede un picco nella sezione d'urto, quindi dobbiamo trovare per quali valori dell'energia possiamo avere uno scattering risonante. Per fare ciò dobbiamo cercare un picco nella sezione d'urto in funzione dell'energia, il quale si presenta quando il phase shift, in corrispondenza dell'energia di risonanza, assume il valore $\pi/2$, cioè
   \begin{equation*}
      \delta_0(E_{\rm ris})
      =\frac{\pi}{2}
   \end{equation*}
   la condizione che quindi dobbiamo imporre nel nostro caso è
   \begin{equation*}
      \delta_0=\frac{\pi}{2}
   \end{equation*}
   a cui corrisponde $\sin{(\delta_0)}=1$, ovvero
   \begin{equation*}
      \frac{2 m V_0}{\alpha^2 \hbar^2} \frac{k}{\alpha}
      =1
   \end{equation*}
   da cui ricaviamo un valore per $k$:
   \begin{equation*}
      k
      =\frac{\alpha^3 \hbar^2}{2 m V_0}
   \end{equation*}
   e quindi l'energia della risonanza sarà
   \begin{equation*}
      E_{\rm ris}
      =\frac{\hbar^2 k^2}{2m}
      =\frac{\hbar^6 \alpha^6}{8m^2 V_0^2}
   \end{equation*}
   Sostituendo i valori numerici troviamo il valore:
   \begin{equation*}
      E_{\rm ris}
      =\frac{(\hbar c)^6 \alpha^6}{8 (m c^2)^3 V_0^2}
      =\rm \frac{2^6 \cdot 10^{12} \; MeV^6 \, fm^6 \cdot (2/3)^6 \; fm^{-6}}{8 \cdot 10^9 \; MeV^3 \cdot 36 \cdot 10^2 \; MeV^2}
      \approx 0.2 \rm \; MeV
   \end{equation*}
   Se andassimo a guardare un grafico della sezione d'urto in funzione dell'energia, osserveremmo un picco in corrispondenza di tale valore. Inoltre a tale valore corrisponde uno stato quasi-legato avente un tempo di vita $\tau$ che risulta essere piuttosto lungo e un'ampiezza $\Gamma$, che è quella che ci viene richiesta nel punto c). In particolare, essa può essere calcolata mediante l'espressione
   \begin{equation*}
      \eval{ \dv{ \bigl[ \cot{(\delta_l)} \bigr] }{E} }_{E=E_{\rm ris}}
      =-\frac{2}{\Gamma}
   \end{equation*}
   da cui
   \begin{equation*}
      \Gamma
      =-2 \biggl( \eval{ \dv{ [ \cot{(\delta_l)} ] }{E} }_{E=E_{\rm ris}} \biggr)^{-1}
   \end{equation*}
   Per calcolare tale espressione è utile ricorrere al suggerimento fornito dal testo, per cui riscriviamo
   \begin{equation*}
      \dv{ \bigl[ \cot{(\delta_l)} \bigr] }{E}
      =\dv{ \bigl[ \cot{(\delta_l)} \bigr] }{\delta_\ell} \dv{\delta_\ell}{E}
      =-\frac{1}{\sin^2{(\delta_\ell)}} \dv{\delta_\ell}{E}
   \end{equation*}
   Tale quantità deve essere valutata in $E=E_{\rm ris}$, in corrispondenza della quale abbiamo visto che si ha $\sin{(\delta_0)}=1$, per cui
   \begin{equation*}
      \eval{ \dv{ \bigl[ \cot{(\delta_l)} \bigr] }{E}}_{E=E_{\rm ris}}
      =- \eval{ \dv{\delta_\ell}{E} }_{E=E_{\rm ris}}
   \end{equation*}
   Dunque possiamo scrivere
   \begin{equation*}
      \Gamma=2 \biggl( \eval{ \dv{\delta_\ell}{E}}_{E=E_{\rm ris}} \biggr)^{-1}
   \end{equation*}
   A questo punto dobbiamo esprimere $\delta_0$ in funzione dell'energia. Dalla \eqref{eq:delta_0_non_small} si ha
   \begin{equation*}
      \delta_0
      =\arcsin{ \biggl[ \frac{(2 m)^{\frac{3}{2}} V_0}{\alpha^3 \hbar^3} \sqrt{E} \biggr] }
      =\arcsin{ \qty[ \beta \sqrt{E} ] }
   \end{equation*}
   dove abbiamo posto
   \begin{equation*}
      \beta
      =\frac{(2 m)^{\frac{3}{2}} V_0}{\alpha^3 \hbar^3}
   \end{equation*}
   Se a questo punto usiamo il suggerimento, abbiamo
   \begin{equation*}
      \eval{ \dv{\delta_\ell}{E} }_{E=E_{\rm ris}}
      =\frac{1}{\sqrt{1 - \beta^2 E_{\rm ris}}} \frac{\beta}{2\sqrt{E_{\rm ris}}}
   \end{equation*}
   In definitiva
   \begin{equation*}
      \Gamma
      =2 \biggl( \frac{1}{\sqrt{1 - \beta^2 E_{\rm ris}}} \frac{\beta}{2\sqrt{E_{\rm ris}}} \biggr)^{-1}
      =\frac{ 4 \sqrt{E_{\rm ris}} \sqrt{1 - \beta^2 E_{\rm ris}} }{\beta}
   \end{equation*}
   Osserviamo che $\beta \sqrt{E_{\rm ris}}=1$, per cui $\Gamma=0$, cioè l'ampiezza è nulla. Ciò corrisponde a una lunghezza di vita infinita (in quanto $\tau=1/\Gamma$), dunque abbiamo uno stato legato.
\end{soluzione}

\newpage
\setcounter{equation}{0}