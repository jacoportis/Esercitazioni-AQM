\begin{esercizio}
   A particle in a one-dimensional harmonic oscillator is subject to the potential
   \begin{equation*}
      V'(t)=\hbar\omega \frac{\alpha^3 x^3 - 1}{1 + (t/\tau)^2}
      \qq{with}
      \alpha=\qty(\frac{m \omega}{\hbar})^{\frac{1}{2}}
   \end{equation*}
   Assuming the particle is in the ground state at $t=-\infty$:
   \begin{enumerate}[label=\alph*), leftmargin=0.6cm]
      \item Determine all the possible transitions allowed at first order of perturbation theory.
      \item Calculate the probability\footnotemark\;of such transitions (excluding $0 \to 0$) for $\omega=10^{15} \rm \; s^{-1}$ and $\tau=10^{-15} \rm \; s$.
      \item Is perturbation theory applicable for case b)? Briefly explain your answer.
   \end{enumerate}
\end{esercizio}
\begin{soluzione}
   \footnotetext{Il testo non lo dice esplicitamente, ma richiede di calcolare la probabilità di transizione al tempo $t=+\infty$.}
   Si tratta di un problema di teoria perturbativa dipendente dal tempo.\\
   Svolgiamo il punto a). Per prima cosa dobbiamo determinare le transizioni permesse. Ricordiamo innanzitutto che il potenziale dell'oscillatore armonico è
   \begin{equation*}
      V(x)=\frac{1}{2}m\omega^2 x^2
   \end{equation*}
   mentre i livelli di energia sono dati da
   \begin{equation*}
      E=\hbar\omega \qty(n + \frac{1}{2})
      \qq{,}
      n=0,1,2,\ldots
   \end{equation*}
   Inoltre possiamo scrivere l'operatore posizione in termini degli operatori di creazione e annichilazione come
   \begin{equation*}
      x=\sqrt{\frac{\hbar}{2m\omega}} \qty( a + a^{\dag} )
   \end{equation*}
   Gli operatori di creazione annichilazione agiscono sugli autostati dell'oscillatore armonico nel seguente modo:
   \begin{gather*}
      a \ket*{n}
      =\sqrt{n} \ket*{n - 1}
      \qq{,}
      a \ket*{0}=0
      \\
      a^{\dag} \ket*{n}
      =\sqrt{n + 1} \ket*{n + 1}
   \end{gather*}
   L'ampiezza di transizione dal ground state $\ket*{0}$ allo stato finale $\ket*{f}$ è data da
   \begin{equation*}
      d_{0 \to f}
      =-\frac{i}{\hbar} \int_{-\infty}^{+\infty} \dd{t} e^{i \omega_{f,0} t} V'_{f,0}(t)
   \end{equation*}
   dove
   \begin{equation*}
      \omega_{f,0}
      =\frac{E_f - E_0}{\hbar}
      \qq{,}
      V'_{f,0}
      =\mel*{f}{V'}{0}
   \end{equation*}
   Per capire quali sono le transizioni permesse dobbiamo andare a vedere quali sono quelle aventi ampiezza di transizione non nulla. Tale condizione implica che deve essere non nullo l'elemento di matrice $V'_{f,0}$. Esplicitamente, tale condizione significa che
   \begin{gather*}
      \mel*{f}{V'}{0}
      =\frac{\hbar\omega}{1 + (t/\tau)^2} \mel*{f}{\alpha^3 x^3 - 1}{0}
      \neq 0
      \\[0.1cm]
      \implies
      \mel*{f}{\alpha^3 x^3}{0} - \braket*{f}{0}
      \neq 0
   \end{gather*}
   Il termine $\braket*{f}{0}$ è diverso da 0 quando quando $\ket*{f}=\ket*{0}$, ma il testo diceva di escludere le transizioni $0 \to 0$, per cui bisogna soltanto ricavare in che caso si ha $\mel*{f}{\alpha^3 x^3}{0} \neq 0$.\\
   Se adesso esprimiamo $x$ in termini degli operatori di creazione e annichilazione ma utilizzando la definizione di $\alpha$ fornita dal testo, abbiamo
   \begin{equation*}
      x=\frac{1}{\sqrt{2} \alpha} \qty( a + a^{\dag} )
   \end{equation*}
   e dunque
   \begin{equation*}
      x^3
      =\frac{1}{2^{\frac{3}{2}} \alpha^3} \qty( a + a^{\dag} )^3
   \end{equation*}
   In questo modo la condizione da imporre diventa
   \begin{equation*}
      \frac{1}{2^{\frac{3}{2}}} \mel*{f}{\qty( a + a^{\dag} )^3}{0}
      \neq 0
   \end{equation*}
   Osserviamo che possiamo riscrivere l'operatore come
   \begin{equation*}
      \qty( a + a^{\dag} )^3
      =\qty( a + a^{\dag} ) \bigl( a^2 + {a^{\dag}}^2 + aa^{\dag} + a^{\dag}a \bigr)
   \end{equation*}
   Piuttosto che svolgere questo prodotto e vedere poi che cosa si ottiene dall'applicazione di ciascun termine a $\ket*{0}$, possiamo già fare una considerazione: ogni volta che applichiamo un prodotto di operatori che ha come ultimo termine $a$ allo stato $\ket*{0}$, esso darà $0$, dunque possiamo trascurarli. Nel nostro caso operatori del genere sono $a^2$ e $a^{\dag} a$, che per quanto detto posso essere omessi. In definitiva, possiamo considerare soltanto l'operatore
   \begin{equation*}
      \qty( a + a^{\dag} ) \bigl( {a^{\dag}}^2 + aa^{\dag} \bigr)
      =a{a^{\dag}}^2 + a^2 a^{\dag} + {a^{\dag}}^3 + a^{\dag} a a^{\dag}
   \end{equation*}
   Osserviamo che è possibile scartare anche il termine $a^2 a^{\dag}$ in quanto applicandolo a $\ket*{0}$ si ha
   \begin{equation*}
      a^2 a^{\dag}\ket*{0}
      =a^2 \ket*{1}
      =a \ket*{0}
      =0
   \end{equation*}
   Vediamo allora cosa danno gli altri termini:
   \begin{equation*}
      \begin{split}
         a{a^{\dag}}^2 \ket*{0}
         &
         =a a^{\dag} \ket*{1}
         =\sqrt{2} a \ket*{2}
         =2 \ket*{1}
         \\
         {a^{\dag}}^3 \ket*{0}
         &
         ={a^{\dag}}^2 \ket*{1}
         =\sqrt{2} a^{\dag} \ket*{2}
         =\sqrt{6} \ket*{3}
         \\
         a^{\dag} a a^{\dag} \ket*{0}
         &
         =a^{\dag} a \ket*{1}
         =a^{\dag} \ket*{0}
         =\ket*{1}
         \\
      \end{split}
   \end{equation*}
   In definitiva abbiamo che
   \begin{equation*}
      \mel*{f}{\qty( a + a^{\dag} )^3}{0}
      =3\braket*{f}{1} + \sqrt{6} \braket*{f}{3}
      =3 \delta_{f,1} + \sqrt{6} \delta_{f,3}
   \end{equation*}
   dove nell'ultimo passaggio abbiamo usato l'ortonomalità degli autostati dell'oscillatore armonico.\\
   Dunque le transizioni permesse sono $0 \to 1$ e $0 \to 3$, in quanto sono le uniche per cui tale termine è non nullo. Notiamo che non è permessa la transizione $0 \to 2$, come ci si poteva aspettare per questioni di simmetria. Infatti, l'interazione è dispari, poiché dipende da $x^3$, mentre gli stati $\ket*{0}$ e $\ket*{2}$ sono pari, dunque l'elemento di matrice corrispondente sarà nullo.\\
   Passiamo al quesito b). Dobbiamo calcolare il valore delle probabilità di tali transizioni in corrispondenza di certi valori di $\omega$ e $\tau$.\\
   In termini delle transizioni permesse, $V'_{f,0}$ si scrive come
   \begin{equation*}
      V'_{f,0}
      =\frac{\hbar\omega}{1 + (t/\tau)^2} \frac{1}{2^{\frac{3}{2}}} \bigl( 3 \delta_{f,1} + \sqrt{6} \delta_{f,3} \bigr)
   \end{equation*}
   Distinguiamo i due casi:
   \begin{itemize}[leftmargin=0.5cm]
      \item Transizione $0 \to 1$. In questo caso abbiamo che
      \begin{equation*}
         \omega_{1,0}
         =\frac{E_1 - E_0}{\hbar}
         =\frac{\qty( 1 + \frac{1}{2} ) \hbar\omega - \frac{1}{2} \hbar\omega }{\hbar}
         =\omega
      \end{equation*}
   quindi l'ampiezza di transizione sarà data da
   \begin{equation*}
      d_{0 \to 1}
      =-\frac{i}{\hbar} \int_{-\infty}^{+\infty} \dd{t} e^{i \omega t} \frac{3}{2^{\frac{3}{2}}} \frac{\hbar\omega}{1 + (t/\tau)^2}
      =-i \frac{3}{2^{\frac{3}{2}}} \omega \tau^2 \int_{-\infty}^{+\infty} \dd{t} \frac{e^{i \omega t}}{t^2 + \tau^2}
   \end{equation*}
   Per risolvere quest'integrale applichiamo il metodo dei residui.
   \begin{figure}[H]
      \centering
      \begin{tikzpicture}
            %dominio
            \filldraw[gray!40!] (2,0) arc (0:180:2cm) -- cycle;
            %assi
            \draw[->] (-2.3,0) -- (2.3,0) node[right] {$\Re{z}$};
            \draw[->] (0,-1.5) -- (0,2.4) node[above] {$\Im{z}$};
            %singolarità
            \filldraw[red] (0,1) circle (1.2pt) node[right] {\footnotesize$+i\tau$};
            \filldraw[red] (0,-1) circle (1.2pt) node[right] {\footnotesize$-i\tau$};
            \begin{scope}[decoration={markings,mark=at position 0.2 with {\arrow{>}}, mark=at position 0.45 with {\arrow{>}},mark=at position 0.7 with {\arrow{>}}, mark=at position 0.9 with {\arrow{>}}}] 
               \draw[postaction={decorate}] (2,0) arc (0:180:2cm) -- cycle;
            \end{scope}
      \end{tikzpicture}
   \end{figure}
   In particolare, la funzione integranda ha due poli in $t=\pm i \tau$, ma poiché dobbiamo prendere il contorno superiore il teorema dei residui ci dà
   \begin{gather*}
      \int_{-\infty}^{+\infty} \dd{t} \frac{e^{i \omega t}}{t^2 + \tau^2}
      =2\pi i \underset{t=i\tau}{\Res} \qty[ \frac{e^{i \omega t}}{t^2 + \tau^2} ]
      \\
      =2\pi i \lim_{z \to i\tau} (z - i\tau) \frac{e^{i \omega z}}{(z + i\tau)(z - i\tau)}
      =2\pi i \frac{e^{-\omega \tau}}{2i\tau}
      =\frac{\pi e^{-\omega \tau}}{\tau}
   \end{gather*}
   In definitiva
   \begin{equation*}
      d_{0 \to 1}
      =-i \frac{3}{2^{\frac{3}{2}}} \omega \tau^2 \frac{\pi e^{-\omega \tau}}{\tau}
      =-i\frac{3}{2^{\frac{3}{2}}} \omega \tau \pi e^{-\omega \tau}
   \end{equation*}
   e quindi la probabilità di transizione da $\ket*{0}$ a $\ket*{1}$ sarà pari a
   \begin{equation*}
      P_{0 \to 1}
      =| d_{0 \to 1} |^2
      =\frac{9 \pi^2}{8} (\omega \tau)^2 e^{-2\omega \tau}
   \end{equation*}
   \item Transizione $0 \to 3$. In questo caso abbiamo che
   \begin{equation*}
      \omega_{3,0}
      =\frac{E_3 - E_0}{\hbar}
      =\frac{\qty( 3 + \frac{1}{2} ) \hbar\omega - \frac{1}{2} \hbar\omega }{\hbar}
      =3\omega
   \end{equation*}
   quindi l'ampiezza di transizione sarà data da
   \begin{equation*}
      d_{0 \to 3}
      =-\frac{i}{\hbar} \int_{-\infty}^{+\infty} \dd{t} e^{i 3\omega t} \frac{\sqrt{6}}{2^{\frac{3}{2}}} \frac{\hbar\omega}{1 + (t/\tau)^2}
      =-i \frac{\sqrt{6}}{2^{\frac{3}{2}}} \omega \tau^2 \int_{-\infty}^{+\infty} \dd{t} \frac{e^{i 3\omega t}}{t^2 + \tau^2}
   \end{equation*}
   Procedendo in maniera analoga al caso precedente, il teorema dei residui ci dà
   \begin{gather*}
      \int_{-\infty}^{+\infty} \dd{t} \frac{e^{i 3\omega t}}{t^2 + \tau^2}
      =2\pi i \underset{t=i\tau}{\Res} \qty[ \frac{e^{i 3\omega t}}{t^2 + \tau^2} ]
      \\
      =2\pi i \lim_{z \to i\tau} (z - i\tau) \frac{e^{i 3\omega z}}{(z + i\tau)(z - i\tau)}
      =2\pi i \frac{e^{-3\omega \tau}}{2i\tau}
      =\frac{\pi e^{-3\omega \tau}}{\tau}
   \end{gather*}
   In definitiva
   \begin{equation*}
      d_{0 \to 3}
      =-i \frac{\sqrt{6}}{2^{\frac{3}{2}}} \omega \tau^2 \frac{\pi e^{-3\omega \tau}}{\tau}
      =-i\frac{\sqrt{6}}{2^{\frac{3}{2}}} \omega \tau \pi e^{-3\omega \tau}
   \end{equation*}
   e quindi la probabilità di transizione da $\ket*{0}$ a $\ket*{3}$ sarà pari a
   \begin{equation*}
      P_{0 \to 3}
      =| d_{0 \to 3} |^2
      =\frac{3 \pi^2}{4} (\omega \tau)^2 e^{-6\omega \tau}
   \end{equation*}
\end{itemize}
   Troviamo adesso un valore numerico per tale probabilità con i dati forniti dal testo. Osserviamo che per tali valori si ha $\omega \tau=1$, dunque:
   \begin{gather*}
      P_{0 \to 1}
      =\frac{9 \pi^2}{8} e^{-2}
      \approx 1.5
      \\
      P_{0 \to 3}
      =\frac{3 \pi^2}{4} e^{-6}
      \approx 0.02
   \end{gather*}
   Svolgiamo il punto c). I valori trovati nel quesito b) ci dicono che la teoria perturbativa non è applicabile per la transizione $0 \to 1$, in quanto la probabilità di transizione è addirittura maggiore di 1.\footnote{Tale risultato può essere collegato col fatto che l'espansione perturbativa si fa attorno ad uno stato che deve essere stabile, per cui lo stato $\ket*{1}$ potrebbe non essere stabile in quanto subisce continuamente transizioni verso altri stati e in conseguenza a cià non è valida la teoria perturbativa.} Per la transizione $0 \to 3$, invece, si ha che $P \ll 1$, dunque la teoria perturbativa è valida.
\end{soluzione}

\newpage
\setcounter{equation}{0}

\begin{esercizio}
   Given a proton scattering on the potential
   \begin{equation*}
      V(r)=
      \begin{cases}
         -V_0 & \text{for} \; r \leq a\\
         0 & \text{for} \; r > a
      \end{cases}
   \end{equation*}
   \begin{enumerate}[label=\alph*), leftmargin=0.6cm]
      \item Calculate the differential cross section and the phase shift for $\ell=0$ (not in the low-energy limit).
      \item Considering now the low-energy limit fot the phase shift, calculate at what energy (in MeV) $\delta_0=\pi/6$ if $V_0=40 \rm \; MeV$ and $a=1 \rm \; fm$.
      \item Can it be considered low energy?
   \end{enumerate}
   Hint 1:
   \begin{equation*}
      \dv{x} \frac{\cos{x}}{x}
      =\frac{\cos{x}}{x} - \frac{\sin{x}}{x^2}
   \end{equation*}
   Hint 2: Use the approximation $e^{i \delta_{\ell}} \sin{\delta_{\ell}} \simeq \delta_{\ell}$.
\end{esercizio}
\begin{soluzione}
   \comment{
   L'ampiezza di scattering la possiamo scrivere usando la partial wave analysis, quindi con questo applicando il potenziale centrale possiamo applicare la partial wave analysis. Quindi questo uguale a, sostituiamo a questo punto l'espressione del potenziale e diventa 2Mv0 diviso a catagliato quadro Q integrale da 0 a da A, che quindi integrano nullo solo da 0 a da, non è è è è risolve per questo integrale? Come piaccia a voi per parti? Oppure con altri metodi strani che qualcuno ha suggerito potete provare, ma per parti è semplicissimo. Allora, quindi 2M V0, vi faccio notare per l'ennesima volta che quando avete ampiezze di scattering, trovate quasi sempre questo termine così. Quando abbiamo qualcosa che sia con un potenziare costante o non solo, non abbiamo visto anche il potenziare costante per la delta, questo termine così lo trovate praticamente sempre, quindi quindi un cappellino dall'arme, se non lo trovate che stai dimenticando qualcosa. Quindi 2M V0 diviso accatagliato 4Q, vabbè i passaggi dell'integrale per partito salto arriviamo direttamente a Seno di Cupera diviso Q con A meno Coseno di Q con A. Quindi la sezione d'urto differenziale di sincrono in deomega che è uguale quindi a questa F di Born al quadrato, l'ambizziata di scatting è uguale a 2M V0 a accatagliato 4Q, 4T al quadrato per tutto questo termine elevato al quadrato. Allora, questo qua, ma beh, non lo richiede il problema, però siccome a volte lo fa, vi ricordo che in questo caso potremmo scriverla volendo come una testa in deomega di Rutherford, come l'abbiamo fatto l'altra volta per un fattore di forma F di Q4, ripeto qua non lo chiede, quindi non lo dovete fare però siccome magari potrebbe capitare un esplicizio che lo chiede, ve lo scrivo. Quindi in questo caso la sezione d'urto di Rutherford sarebbe 2M, 4H, 4Q, 4Q elevato al quadrato è il fattore di forma. In questo caso sarebbe Q0A del quadro. Ok, quindi togliendo l'essuto e il passaggio non era necessario, la cosa che ci interessava era questa qua, quindi quindi sezione d'urto differenziale, l'altra cosa che chiede è il problema è il phase shift per L uguale a 0. E quindi delta 0 è circa uguale, usiamo quindi il primo e eccio metodo quello che vi ho fatto vedere prima, K mezzi integrale da 0 a P greco di d theta, sin theta, f di K theta, sostituiamo l'f che abbiamo trovato, quindi K mezzi, scopriamo da K perché sono non c'entra, K mezzi integrale da 0 a P greco di theta, sin theta, 2M v con 0A, H tagliato quadro, 4K quadro, quindi ho sostituito anche l'espressione per Q, se in quadro di theta mezzi che chiaramente abbiamo integrale in theta quindi dobbiamo per forza scrivere espressioni in funzione di theta. Seno di Q con A e sì, veramente dovremmo fare lo stesso anche qui, ma così fa cancellare. 
   
   Quindi 2K, seno di theta mezzi, diviso 2K, seno di theta mezzi, meno. Coseno di 2K, seno di theta mezzi. A questo punto faccio anche un altro passaggio, perché perché vedete qua abbiamo theta mezzi, seno di theta mezzi, seno di theta mezzi, ok anche qui, ma ma abbiamo seno di theta e allora ce lo scriviamo come 2 seno di theta mezzi cosen theta mezzi, in modo da avere tutto in funzione di sen theta mezzi e possiamo quindi fare un operare per sostituzione, risolvere l'integrare per sostituzione, perché definiamo una nuova variabile y uguale a 2K seno di theta mezzi, quindi la nostra dey è uguale a 2K un mezzo coseno di theta mezzi, quindi automaticamente il coseno che abbiamo qui ce lo siamo ritrovati per fortuna e quindi questo uguale a K, no manca un d theta, e e questo scritto beccoppale coseno di theta mezzi di theta, e vabbè l'estremità di integrazione che non non 0 0 di 0 uguale a 0 e di coseno a PY risulta uguale a 2K con A, quindi con la nuova variabile di integrazione integrale che uguale a m v0a diviso 2H4K integrale da 0 a 2K con A, di y K con A, 2K y, diciamo faccio alcuni passaggi così, dato che è un po' camorrioso con tanti termini, potete controllare più facilmente, se c'è qualche errore riusciamo magari a rintracciarlo già ora, qualche fattore che in più o in meno. 
   
   Ok quindi a parte qualche fattore che dovrebbe essere comunque tutto ok, quindi qua abbiamo K a K che possiamo semplificare, abbiamo anche il 2 qui, semplificano altro po', ma se ci fate caso abbiamo ottenuto l'espressione che era nell'indizio nell'aiuto, perché se noi mettiamo questo y all'interno otteniamo quindi meno m v0a diviso H tagliato quadro K integrale da 0 a 2K di y, cos'è un p p p meno sen y y quadro. Ok quindi ricordate sempre quando avete degli indizi, cercate, a meno che non avete un'altra idea per risolvere lo che più semplice, cercate sempre di arrivare all'espressione che trovate nell'indizio, quindi a questo punto questo usando l'interno si può scrivere come meno m v0a H tagliato quadro K integrale a 0 a 2K di y, d in dy sen y diviso y e quindi questo risolto uguale a meno m v0a H tagliato quadro Dv. sen 2Ka diviso 2Ka meno 1. Ok abbiamo finito il primo punto, domande, dubbi? Sembra un riscovoce l'interno si blocca, non riesco più a scoprire l'alavan. interrompo la condivisione e poi la reattivo perché non mi sta rispondendo più. Ok, quindi siamo arrivati a questa, questa è il Fesh Shift, abbiamo fatto un Fesh Shift. Seconda domanda, chiade considera ora il low energy limit? Che è che chiede di calcolare a quale energia, per certi valori di 0 e di a, il Fesh Shift è Fy6, sempre quello del T0. Allora, low energy limit che cos'è? Vabbè lo sapete, l'abbiamo detto mille volte e quando KRV è molto minore di 1, in questo caso avvetto che RV è uguale ad a, quindi K ha molto minore di 1. Quindi, siccome ci chiede di calcolare quando T0 è uguale a Fy6, quindi dobbiamo lavorare sul T0 che è appunto l'espressione che avevamo prima. Cosa possiamo fare, dato che siamo in low energy limit? Sviluppare il seno. Sì, corretto, fino a che ordine? Secondo. Sì, secondo, perché se considerassimo solo il primo farebbe 0, otterremmo 1 meno 1, 0. Perché viene 1 meno 1, so. Sì, quindi giusto, secondo, e quindi facciamo delta 0 uguale meno mv0a. H4K, rivolte quindi il seno di X per X che è circa uguale a 0, rivolta secondo ordine X meno un sesto X³. Quindi questo viene 2K meno un sesto 8K³³ diviso 2K-1. E quindi facciamo uguale, facendo vari facendo vari per l'IV0 uguale a 40 mv che richiede esercizio a uguale a un fermi e delta 0 uguale a y6, ci sta chiedendo qual'energia. L'incidente. Quindi è solito uguale a K² diviso 2m. Quindi K uguale 2m sul K² alla un mezzo. E quindi, siccome compare solo K, abbiamo che delta 0 uguale a 2mv0 a cubo, 3 a K². Facciamo questo passaggio. Quindi questo uguale a 2m diviso a K². E quindi elevando a K² entrammi i membri otteniamo 2m elevato alla terza, perché mi prendo un pezzo da qui. V0²
   
   alla sesta e quindi direttamente l'energia che mi interessa, 9 a K. E quindi energia richiesta al problema. E ora, 9, metto direttamente l'Ec che ci servono per il calcolo a K, di 0². Si sta riabbiando la lavagna. No, è iniziato a fare il calcolo. Come sempre il calcolo numerico è un buon esercitarci. Fatemi il calcolo e ditemi quale questa energia. Vi lascio qualche minuto. Diamo amorevati a qui. Stiamo scrivendo. Continuiamo con il calcolo e controlliamo per risultato. E ora, come siamo combinati? Sì, che valori avevo ottenuti. Non mi ricordavo se era un protone o meno. Era un protone nella particella giusta. Sì, è un protone. Sì, è un protone. un è un protone. Alla chiedeva in mezz'r energia, quindi un'unità di misura, che voglio so in mezz'r. Sì. A me viene dell'ordine di 10 alla quarta nebb, però non so se è corretto. Forse un po' troppo. A me è venuto dell'ordine della disina di mezz. Potrei essere veglia anch'io, perché io l'ho fatta, ma non mi controllo mai, apposta per controllarli dal vivo. Quindi è divertente ogni volta vedere quanti numeri diversi riescono a sconoscere. Con me è venuto circa 6. Circa 6? Ok. Noi abbiamo quindi solo un fattore due di differenza. Vediamo gli altri. C'è qualcuno che sta per finire, vuole essere atteso o procedo? Possiamo procedere penso. Procedo. E quindi non nessun risultato per gli altri. Va bene, non ti procedo, ma vero qualcuno intanto mi dice. Allora, 9 ha catagliato C, al sueto così, guardate e mi controllate se sto svelando qualcosa. Abbiamo 10 alla 2, quindi diventa 10 alla 12. Mev alla 6, fermi alla 6. Poi abbiamo il delta 0, che era y6, quindi diventa pk4, diviso 36. Poi abbiamo sotto 8. M è un protone, quindi 10 alla 3. Mev, quindi moltiplicato alla 3a diventa 10 alla 9. Per cosa abbiamo? Mev alla 3a. 
   
   Il potenziale è 40 mev, quindi ha squatato, quindi, 2 alla 4a. Per 10 alla 2. Mev 4. 4. Fermi alla 6. Vi racconto non andate per scontata questa parte, perché molti si fermano prima, non trovi risultato. Quell'esame si deve fare e vengono tutti i punti se non viene fatta bene, quindi vi dovi d'allenare. Cosa abbiamo qui? Allora, facciamo 4. 4 e 2 alla 2, quindi questo diventa 2. Mev tra questo e questo rimane solo 1. Questo lo abbiamo in questo. Poi abbiamo 10 alla 9, 10 alla 2, quindi qua rimane solo un 10. Poi abbiamo 10 alla 2, e qua 10 alla 6 diventa Ho spagliato che c'era l'area, dovevo fare alla 4a. E poi 8. Ho spagliato qualcosa? Ha tagliato l'un quarto con il 2 alla 4a. Ah, sì, giusto, grazie. Torniamo indietro. Ma in quanto tempo farò fatto? Ecco, ci siamo. Allora, in tanto leviamo questo, con questo, 2 e questo era giusto. Poi, sì, a questo punto questo è sotto, quindi possiamo tagliarlo 2 alla 2 e con questo. Quindi questo è 4. E poi 2 alla 3 con 2 alla 4 diventa 2. E poi abbiamo 10 alla 2, 10 alla 9 e fa 11, quindi qua lo levo. Poi è rimasto è rimasto 2 per PY4, per 10, e sotto un 2 alla 4. Ah, quindi potevo cancellarlo anche questo 2. Quindi qua rimane 8. PY8, vi 10, ovviamente, web. PY è circa 10, quindi ho 108 web, che è 12.5 web. Durebbe essere così. Ok, va bene. Ok, beh, abbiamo trovato l'energia tale da avere quei valori del fastshift, eccetera. Terzo punto, chiede Puoi essere considerato lao energy? Che mi dita? Quindi cosa facciamo ora? Dite, non lo vuoi, se puoi essere considerato lao energy. È chiaro cosa dovete fare per controllare se lao energy. Bisogna confrontare K con A, no? K con A, dove il K lo tirate fuori da questo E, che è l'emplotore, e confrontarlo se è molto minore di 1. Quindi come dè, è un passo agio matematico semplice, è dei calcoli numerici che qual solito possono portare errori, quindi semplicemente da far attenzione. Cercate di farlo tutti, non aspetto risultati da più persone. Se avete dubbi, un dubbio specifico, chiedete. E come comunque avete fatto altre volte finora. Io mentre voi procedete un secondo, due minuti devo uscire rientro. Siamo in un'altra in un'altra in un'altra in Siamo in un'altra in Siamo in un'altra in Allora, risultato? Qualcuno ce l'ha? A me è venuto il K circa 0,8 fermi alla meno 1. E quindi il K con A? Era un fermi, mi pare, non ricordo male. No, un fermi. Quindi sì è minore di 1, ma non è molto minore di 1. Quanto quindi? Cioè 0,8. Ok, anche a me è venuto 0,8, quindi facciamo i passaggi urocemente, così ne affrattiamo anche gli altri. Quindi da questa energia che abbiamo trovato, abbiamo che K per A, uguale a 2 Me, ha catalirato 4, elevato a 1 mezzo per A, quindi uguale a 2 Me, e a 4 diviso a catalirato C, tutto al quadrato e tutto questo sottoradice. Sosturiamo i valori. 2 per 10 alla 3 Meb, per 12,5 Meb, per fermi al quadrato che viene dall'A, diviso 4 per 10 alla quarta, Meb, quadro fermi al quadrato, tutto elevato a 1 mezzo. Se fate il calcolo viene 0, fate voi le semplificazioni, questa volta 0,8. Quindi non è veramente molto minore di 1. Quindi siamo in quella situazione in cui la teoria dice che dovrebbe essere molto minore di 1, quindi,rebbe da dire, non siamo all'energy limit. Poi a volte, nella pratica, quando si fa ricerca, può capisare che un intanto controlla e vede cos'altro ottiene come altri risultati, e potrebbe essere ancora utilizzabile questo limite, però diciamo che non siamo not really in the energy limit. E in questo, scusate, low energy limit.
   }
\end{soluzione}

\newpage
\setcounter{equation}{0}

\begin{esercizio}
   A proton is scattering on the potential
   \begin{equation*}
      V(r)=V_0 \frac{e^{-\alpha r}}{\alpha r}
   \end{equation*}
   It is known that
   \begin{equation*}
      f(k, \vartheta)
      =-\frac{2m V_0}{\alpha \hbar^2} \frac{1}{q^2 + \alpha^2}
   \end{equation*}
   \begin{enumerate}[label=\alph*), leftmargin=0.6cm]
      \item Calculate the phase shift for $\ell=0$ in the low-energy limit.
      \item Calculate at what energy (in MeV) one has resonant scattering if $V_0=60 \rm \; MeV$ and $\alpha=2/3 \rm \; fm^{-1}$.
      \item Evaluate the width of the resonance.
   \end{enumerate}
   Hint 1:
   \begin{equation*}
      \int_{-1}^{1} \dd{x} \frac{1}{b(1-x) + a^2}
      =\frac{1}{b} \ln\qty( 1 + \frac{2b}{a^2} )
   \end{equation*}
   Hint 2: note that the $\delta_{\ell}$ is small
   \begin{equation*}
      \ln( 1 + x )
      \simeq x
      \qq{for}
      x \ll 1
   \end{equation*}
   due to the low-energy limit
   Hint 3:
   \begin{equation*}
      \dv{\cot{x}}{x}
      =-\frac{1}{\sin^2{x}} \dv{\arcsin{x}}{x}
      =\frac{1}{\sqrt{1 - x^2}}
   \end{equation*}
\end{esercizio}
\begin{soluzione}
   \comment{
   facciamo un terzo esercizio, scattere in sempre. E' proton proton scattering on the potential Vr, i quartuvi 0 e meno alpha r di visual f di k theta uguale meno 2m v0 diviso alpha catagliatto quadro, che moltiplica uno più q quadro più più più più più più più più più più più energy limit, punto B calculate at what energy in MeV, one has a resonance, resonance scattering, if v0 uguale 60 MeV and alpha uguale 2 3 fermi all'Aven1, c, evaluate the width of the resonance. Ok, allora allora allora allora allora allora allora allora allora allora allora allora Allora dice che per calcolare consiglia di usare il projection method, questo c'è bisogno che lo scrivo perché l'abbiamo già usato, poi da per il calcolo del modulo di e del tel, sen del tel, Poi dice integrale da meno una uno in the x uguale uno su B, uno meno x più a quadro uguale uno su B, logaritmo di uno più 2B diviso a quadro. E' not not that the tel is small. Poi dice anche, non ci sarebbe possibile anche bisogno di dirle questa cosa, approssimale logaritmo di uno più x a x, for x molto minore di uno più 2 del lower energy limit. E poi dice anche, voi vedete qua ci sono tutta una serie di indizi di aiuti perché magari dal potista numerico potrebbe essere un po' lungo, è complicato. Quindi vi ricordo qual'è la derivata della potangente e anche del coseno. Allora, per il primo punto chiedete di calcolare il f shift per L0 in the lower energy limit e consiglia chiaramente di usare il projection method. Per l'uguale a 0, uniei delta 0, sen di delta 0, circa uguale a k2, k mezzi, integrale da 0 a pi greco, in the theta, sen theta, fk di theta. Quindi qua stiamo usando il projection method sempre utilizzando il fatto che per 0 la legende polinomia è uguale a 1, fin 0 di cosente theta. Per yukawa potencia, avevo già scritto il testo, qual'è fk, quindi fk di theta è dato nel testo da inserire qua dentro. Ed era uguale a, lo inseriamo direttamente, quindi questo uguale a meno k mezzi, 2m v0 diviso alpha k tagliato quadro, integrale da meno 1 a 1, decosente theta, 1 diviso 4k quadro, sen quadro di theta mezzi, più alpha quadro. Quindi abbiamo sostituito la fk theta che dava il testo dell'esercizio, sostituendo per cui abbiamo fatto spesso 2k sente da mezzi. Ok, quindi quindi qua sfruttiamo il fatto che questo ce lo ricopriamo uguale, integrale da meno 1 a 1, decosente theta, sfruttando le proprietà di diriconometrica, questo lo possiamo scrivere come 2k quadro, 1 meno cosente theta, in modo da ritrovare la variabile di integrazione, più alpha quadro. Ora questo è proprio uno degli indizi, che sarebbe questo qui, quindi quindi da meno 1 a 1 1 dx sub b, 1 meno x più a quadro, che quindi otteniamo una funzione localitmo. Quindi avendo utilizzato il primo indizio, questo diventa, a parte questa parte già che ricopriamo, meno 1 su 2k quadro, localitmo di 4k quadro più alpha quadro, diviso alpha quadro. E questo è uguale, considerando anche il fatto di iniziare che non ho ricopriato, mv0 diviso 2k alpha a cataliato quadro, localitmo di 1 più 4k quadro alpha quadro. Ora sfruttiamo il low energy limit, e quindi anche un altro indizio che ci dava, che era proprio l'indizio sull'localitmo, quindi k rv molto meno di 1. In questo caso rv per l'ipotentria yk v è 1 su alpha, e e questo diventa k su alpha, molto meno di 1. Come vedete questo è all'interno del localitmo, quindi possiamo sfruttare l'indizio dato, utilizzare il log di 1 più x circa uguale a x, e quindi otteniamo, scrive intanto il localitmo, localitmo di 1 più 4k quadro alpha quadro, quindi circa uguale a 4k quadro alpha quadro. Delta 0, seno di delta 0, uguale a 2mv0 alpha quadro h quadro k alpha. Che possiamo semplificare questo con il fattore che abbiamo davanti. Se ho sostituito questo qua, il posto e il localitmo, è considerato il fattore che abbiamo davanti, si semplifica. Ora abbiamo ottenuto che, ma non solo vediamo se c'era È stressa? Sì. Scusi, ma può essere che ci siamo mangiati un meno, perché questo modo il face shift non verrà per disegno opposto al potenziale. Ah sì, è effettivamente e qua forse ci siamo mangiati, perché vediamo un attimo. 
   
   Ok sì, sì, sì, Forse? E' fatto, sto vedendo un attimo solo. Allora, qua mi sembra che sia ciolli e plu. Mi sembra che debba essere così. Non rispondo di eduare, dove potrebbe venire fuori un meno. Questo è questo qui, dove è un motocraff. Commenti cosa vi sembra, vedete che dovrebbe venire un meno, non sei sbagliato? Penso che dell'integrale non debba venire il meno secondo volinti. Quindi abbiamo solo il meno dell'inizio. Ah, perché forse mi sono trascritta, ma ti ho detto che allora vediamo. B, quello sarebbe il B, uno meno X, che sarebbe uno meno coseno, più o quadro. E E diventa uno sub-B, che sarebbe ah certo, no, è uno sub-B. L'o-charitmo di uno più è quello sarebbe il ah no, certo, non è solo uno sub-B, non c'è il meno. Perché non li inviare giusto, sì. Quindi questo qua è un un ti ve lo rischio. Questo è un più giusto? Sì. Ok, il meno invece questo qua rimane e quindi ce lo portiamo dietro, giusto. E quindi abbiamo qua questo meno qua. Qui abbiamo un meno. Si stacca dalla lavagna. Ok, grazie per notare questa cosa. Un attimo mi si sta riabbiano, intanto che avete si rievvi la lavagna, non so perché. Allora, e ora io qua ho Cosa faccio? Mentre adesso abbiamo detto il Feshift, quindi mi devo alcolare il Feshift come arcoseno. Ok, posso considerare fare una cosa, posso considerare il modulo di E i delta zero, sen delta zero. E chiaramente il modulo di sen delta zero. E quindi a questo punto tengo 2m di con zero, alfa quadro a catagliato quadro, k alpha. E quindi delta zero risulta essere l'arcoseno di 2m di con zero, alfa quadro a catagliato quadro. K su alfa. E questo siccome non ci dava valori numerici, va bene lasciare l'espressione simbolica. Ci ci interessava l'espressione simbolica e siccome aveva detto di non usare il fatto che i snot-siders, Feshift e smoll, quindi non possiamo approximare il seno di delta zero a delta zero e quindi lo facciamo l'arcoseno di quello che avevamo ottenuto. Ora il punto B è il punto nuovo rispetto agli esercizi precedenti, perché chiede di calcolare il resonant scattering. Qualcuno si ricorda, mi sa dire qualcosa, cosa vi ricorda, cosa fareste? Ok, sia uno scattering resonante, fondamentale una risonanza, quando si vede un piccolo in una sezione di dirto, questo anche esperimentalmente, quando si calcolano in un processo di scattering e si vede un piccolo, quello è un scattering risonante, infatti si è trovata, si è vista una risonanza e questo si può diostrare. Quindi Quindi questo caso è uno stato che ha una vita piuttosto lunga, quindi quindi questo stato colleghiamo una lifetime e di contro colleghiamo quindi anche una width, una ampiezza, che è quello che poi chiede la width della risonanza, il punto C. Quindi intanto dobbiamo trovare per quali valori, qualcosa che chiede, o ottenere, quindi dobbiamo vedere per quali valori dell'energia possiamo avere uno scattering risonante. Dalla toria si può dimostrare che quindi questo picco, quindi noi cerchiamo per trovare uno scattering risonante, un picco della simma infuzione allergia, e e si ottiene quando abbiamo nella partita wave analysis che stiamo considerando, questo avviene quando il phase shift passa per Pgh. Cioè il phase shift delta L, questo diciamo vale in generale con un quasi L, delta L di E, passa per Pgh. Allergia di risonanza, quindi il resonante energy. Quindi questo è proprio quello che dobbiamo fare, dobbiamo imporre, questo lo stiamo considerando delta zero, quindi dobbiamo imporre delta zero uguale y mezzi, quindi seno di delta zero uguale 1 uguale, non ho scritto seno zero uguale 1, uguale 2 m v zero, alfa quadro, H tagliato quadro, K su alfa, adesso ci serve l'energia, quindi quindi mi trovo K, K uguale. e quindi l'energia della risonanza è uguale a k² su 2m uguale a k² a k6 sostituendo k otteniamo a k6 alfa 6tiviso 8m³v0² a questo punto sostituiamo i valori che ci c'è il testo dell'esercizio che erano alpha uguale a 2t, fermi alla men 1 e v0 uguale a 60m e quindi otteniamo al solito dobbiamo moltiplicare per quindi h tagliato c alla 6t, alpha 6tiviso 8m³v0² sostituiamo i valori, io lo faccio e voi controllate quindi abbiamo 2³v0² per 10³v0² alla 12³v0², mev 6³v0², fermi alla 6³v0², questo viene questo viene e poi abbiamo 2³v0², cos'era sempre un protone quindi 10³v0² diventa 10³v0² mev³v0², 
   
   Circa un centinaio di elettronvolta, però non so se è corretto. Un centinaio, sì. Sì, quindi mezzo 2 mezzo. 0.2 mezzo, quindi 36, quindi sì, alla fine è un centinaio di elettronvolta. 200 elettronvolta. No, no, elettronvolta. No, non sarebbe 200 elettronvolta, sarebbe un centinaio di cap, caso mai. Allora controlliamo. Sì, sì, ovviamente sì. Sera saltato qualcosa, ora controlliamo. Quindi semplifichiamo. Al solito controllate se faccio errori mentre semplifico. Quindi abbiamo F6, F-6, qua 2, 5, qua B6, rimane solo 1, poi abbiamo 2. Allora, 36 sarebbe 2 alla 2 per 3 alla 2. Quindi ci rimarranno un 3 alla 8. E poi possiamo semplificare qua 2 alla 3, con 2 alla 6 che diventa 3. Sembra che ci siamo, comunque, controlliamo se. Quindi risulta, abbiamo 2 alla Ah sì, semplifichiamo anche qua 2 alla 2, via, e questo diventa rimane 2. Quindi otteniamo 2 alla 7. Qua abbiamo 3 alla 8. E come potenza del 10 c'è rimasto 9 e 2 fa 11, quindi questo rimane solo con 10. Quindi per 10, mebb. E questo a me risulta, se ho fatto i conti bene con la calcolazione, dice 0,2 mebb. Però potete controllare. Sì, al mio mebb. Ok, perfetto. Quindi abbiamo trovato che questa era l'energia. Quindi se non occupa a guardare la sezione d'urto, solo andasse a guardare un gran filo a sezione d'urto, significa che questa è l'energia in mebb, che intorno a 0,2 vedrebbe un picco. Ecco, questa è la situazione che stiamo andando a guardare. E questo risponde, quindi, a uno stato quasi bound, che quindi ha una certa vita. Come dicevo prima, c'è una certa vita media tau. E questa vita media tau corrisponde una piezza gamma, che è collegato all'inverso della vita media, della lifetime. Come si calcola gamma? Se uno comunque non se lo ricorda nel libro, che l'ho trova, quindi stiamo parlando di punto c, perché il punto c dice, evaluate the width of the resonance. La gamma si calcola in questo modo. Quindi la derivata della contangente di del tel, derivata rispetto all'energia della particella, calcolata proprio all'energia di scattering. E questo è uguale a meno 2 su gamma. Quindi a questa formula otteniamo che gamma uguale a meno 2, derivata di contangente di del tel, in D, per uguale a ecco R. Non mi sta schendo, è uguale a ecco R. E questo è qua. Guardiamo il cd1, guarda questo. Non vedo la pagna scritta, non so, se è un problema mio o anche degli altri. Ok, io la vi sono dicendo. Non la vedo. Ok, adesso la vedo anche io. Perfetto. Niente. Allora, qui abbiamo una contangente. Se vi ricordate avevamo una contangente, nel lint, però guardate come è scritto. Lint, proprio, lo rischivo qua, perché se no dovete andare a guardare il testo, non so se ci la vedete facile. Tego tangente di x in dx, è uguale a meno 1 su sen quadro di x. Questo è l'aiuto che vi davano. Quindi cosa dobbiamo fare? Per poter sottere l'aiuto. Il derivazione composta, quindi c'è un derivato rispetto all'edificio. C'è la dipendenza dell'energia del phase shift. Quindi fate attenzione, magari poi velocemente guarda l'interno, non ci fa caso che fa subito la sostituzione e sbaglia. Quindi in questo caso, ortegniamo meno, meno 2, l'idea cotangente di delta l in d delta l, in modo da ottenere l'espressione per l'indizio, d delta l in d, sempre calcolato a e energia di scattering, uguale a questo punto, oppliamo l'aiuto e diventa 2 sen quadro di delta l, all'energia di scattering, diviso del e d delta l in d,
   
   all'energia di scattering. Ora, in realtà tutto questo dobbiamo fare per l'ugla 0, per il phase shift che avevamo calcolato prima. Sen quadro di delta 0 per e all'energia di scattering, questo risulta uguale a Mi scrive più. Sen quadro di picre a mezzo, se vi ricordate avevamo visto che il scattering sia risonante, sia quando il phase shift qua è la picre a mezzo, quindi questo risulta essere 1. Quindi, sen di delta 0, quindi ora devo prendermi l'espressione che ho ottenuto prima per il sen di delta 0, che era 2m v0 alfa cubo h quadro k questa è la espressione appunto che avevamo trovato, e questo è uguale a sostituisco k in funzione di e, e quindi ottengo 2m elevato a 3 mezzi v0 diviso alfa cubo h tagliato cubo radice di e, quindi che mi viene dal scappo, e ora per semplificare la notazione che mi devo portare a presso, questo lo definisco come beta, e quindi questo risulta beta radice di e. A questo punto il delta 0 è uguale a arcoseno di beta radice di e, e quindi il delta la derivata di delta 0 rispetto a e, che quello mi serve al denominatore per calcolare la width, valutato all'energia dello scattering, è uguale, anche qui abbiamo un altro into che ci diceva qual era la derivata dell'arcoseno, secondo me non se la ricorda. Il qua sfrutto aiuto, questo diventa 1 su radice di 1 meno beta quadro e beta mezzi, 1 su radice di e. E quindi gamma è uguale a 2, il resto del numeratore ho visto che vieni a 1, perché il termine è col seno, e sotto abbiamo, al denominatore abbiamo 1, diviso 1 meno beta quadro e beta mezzi, 1 su radice di e. Valutata all'ampizia di scattering, attualmente perso il foglio che sto guardando. Ok, e niente, quindi trovo che questo, 1 meno beta quadro è valutata all'energia dello scattering. Ok, qui ho trovato che questa, siccome avevamo che beta radice di e, abbiamo trovo prima che beta radice di e, scuola che ci sembra un ritardo quando scrivo, e quindi beta radice di e era uguale a seno di delta zero, che è uguale a 1, all'energia, quando imponiamo che è uguale ai con R, quindi questo qua risulta 1 meno 1, e quindi tutto fa 0. Quindi ho trovato una gamma nulla, un'ampiazza nulla, che corrisponde alla fine a una lifetime infinita, quindi una particella in realtà stabile. Ok, quindi diciamo che questo esercizio più altro era per poter applicare questi concetti che fiorano, avevamo trovato dello scatteri risonante, e di come adesso si può trovare poi quale è la vita o l'ampiazza della visionanza.

   \begin{equation*}
      e^{i \delta_0} \sin{\delta_0}
      \simeq \frac{k}{2} \int_{0}^{\pi} \dd{\vartheta} \sin{\vartheta} f_k(\vartheta)
   \end{equation*}
   \begin{equation*}
      \begin{split}
         e^{i \delta_0} \sin{\delta_0}
         & =-\frac{k}{2} \frac{2m V_0}{\alpha \hbar^2} \int_{-1}^{1} \dd{(\cos{\vartheta})} \frac{1}{4k^2 \sin^2{\qty( \frac{\vartheta}{2} )} + \alpha^2}
         \\
         & =-\frac{k}{2} \frac{2m V_0}{\alpha \hbar^2} \int_{-1}^{1} \dd{(\cos{\vartheta})} \frac{1}{2k^2 ( 1 - \cos{\vartheta}) + \alpha^2}
      \end{split}
   \end{equation*}
   \begin{equation*}
      \ln\qty( 1 + \frac{4 k^2}{\alpha^2} )
      \simeq 4 \frac{k^2}{\alpha^2}
   \end{equation*}
   \begin{equation*}
      e^{i \delta_0} \sin{\delta_0}
      =\frac{2 m V_0}{\alpha^2 \hbar^2} \frac{k}{\alpha}
   \end{equation*}

   }
\end{soluzione}

\newpage
\setcounter{equation}{0}