\begin{esercizio}
   An electron beam scatters on an electron cloud described by the wave function
   \begin{equation*}
      \psi(r)
      =\frac{2}{\sqrt{4 \pi}} \frac{e^{-r/a}}{a^{3/2}}
   \end{equation*}
   Given that the Coulomb potential is given by
   \begin{equation*}
      V(\vb{r})
      =\int \dd[3]{\vb{r}'} \frac{e^2}{| \vb{r} - \vb{r}' |} \rho_e(\vb{r}')
   \end{equation*}
   where $\rho_e(\vb{r})$ is the electron density of the cloud.
   \begin{enumerate}[label=\alph*), leftmargin=0.6cm]
      \item Calculate in Born approximation the differential cross-section (at the end write it as a Rutherford cross-section for a function $F(q^2)$).
      \item What do you get for $qa \to 0$? Explain from a physical point of view.
      \item If at $\vartheta=60^{\circ}$ the differential cross-section decreases by a factor $3^4/4$ when the beam energy increases form 20 eV to 40 eV, calculate the value of $a$ in \A.
   \end{enumerate}
   Hint 1: with the substitution $\vb{y}=\vb{r} - \vb{r}'$, is given that
   \begin{equation*}
      \int \dd[3]{\vb{y}} \frac{e^{i \vb{q} \vdot \vb{y}}}{|\vb{y}|}
      =\frac{4 \pi}{q^2}
   \end{equation*}
   Hint 2:
   \begin{equation*}
      \int_{0}^{+\infty} \dd[r] r e^{-2r/a} \sin(qr)
      =\frac{4qa^3}{(4 + q^2a^2)^2}
   \end{equation*}
\end{esercizio}
\begin{soluzione}
    che il prossimo non debba fare lezzi. Il prossimo è finito. Lo ho fermato perché in tal caso lo sarebbe perché il ultimo merdì è più più a due ore. Dovremmo fare un'ora in più quindi finire a un'aluna alle d'Equacà a l'Octosto. Voi siete tutti a Porto? Anzi la proposta è di di da Porto. Allora lo posso tirare a Piscetta. Allora lo posso posso posso posso posso la proposta è tra venerdì e a a e venerdì di mattino. Il Il Il Il 13 e di ciascuno. 10 ok. Eccoli 10. Lo stiamo scusando, si si rinvolta. E' scusata di venerdì non c'era passato. Eh ma lo biavo vedere se gli altri hanno un'azione. Statisti. Sì ma lo biavo vedere gli altri. Noi non lo biavo un'azione ma gli altri avevano un'azione. Noi non lo biavo un'azione. Non Non Non Non La proposta è di mattina. Non lo biavo. Non lo lo Il scuole è di 15 ore. Ma i nuvi 10, 13. Questo è un'ora che vi metterò questo. Sì. E l'altro è per lì 17. L'orario io ora controllo la mia agenda. No No Non è è bello. Eh infatti per... Per il il regreco a due ore. Per un'ora. Però appunto stiamo dicendo, se tutti i cori volume non abbiamo lezione, lei sarebbe disposta ad anticipare? Ah cioè che manca anche le prime due ore? Sì, abbiamo finito. Ok. E no perché i ossiczii si augurano. Eh. Beh. Oh solo 14 ore di ezione la prossima settimana. Quindi non sembra facile trovare un altro possibilito. Possiamo sebbiamo di fare dalle 11 alle 14. Sì. Meglio adesso vi dico. Se tutti accordo. Dalle 11 le passate. Sì. Si. Si oppone. Va bene. Vingono la maggioranza. Non mi mi nessuno mezzo perché vale le condizioni. Eh beh. Do la verità. La mattina è meglio di farle un po' di 10. 11 con le cose. Ma bene. Così te arrivo quando la ho la la. Ma è il costo di una macchina. Ah sì. La aula dove fate lezioni con professore di meccanina sinistica è sempre la T? La M. E questa qua la M. Ah beh no, tanto allora. Ma Ma facendo un 14 di cedore semplicemente a lunghe alla pronotazione di un'ora nell'aula. Però ultimamente il venerdì abbia fatto lezioni in aula E. Ma la T è meglio. Però la T se c'è la T è meglio sono d'accordo. C'è caldo. C'è caldo. Allora. Allora. Ma bene. Mi sava che sono 10 e come se fosse iniziava. Allora oggi. Ah quindi scusi sono una cosa. Dicevo eventualmente magari non ci si arriverebbe prima del primo appello. Ma Ma il primo e il secondo appello le sarebbe disposto a fare altri incontri oppure no sarebbero. Allora potremmo fare una cosa. Magari mi manda la pittura. Siamo 5 studenti interessanti. Siamo 20 studenti interessanti. Così non fa molto questo recinto e io dovrei avere un stupidato in caso con uno e più. Però sarebbe anche fuori di quello che si è toccorrammato. Sì sì infatti. Però se vedo che se siete uno magari mi dico. No beh facciamo recemendo certo. Se invece siete di più magari. Però mandate i miei mail così. Allora oggi faremo le seccite di tutti i studenti. Io Io anche con la partial web analysis che mi pare che non avevo fatto la scorsa volta. E poi le prossime faccio un mix. In un primo tempo il marco che posso averte studiato avete visto avete avuto delle particolari difficoltà. Per quanto quanto potrei adattare le prossime. Qualcosa più specifico. Di più in quelli che vedono scopri. Ok. Cioè con il tipo loggero di discenso rispetto a un pesce pesce pesce pesce in genere perche' abitiva. Sì. Sì. Sì. Beh perché ci sono ambitivati. Perchè. Perchè. Perchè. Sì. E io vi sto imparando. Io chiedevo di apoi. Una sezione di pitturenzia. Una functione donna. Però davvero una funzione donna. Poi la la Poi l'altro. Diciamo. Dove vengisci e polenzia. Sì. Ma in quel caso. E non ho capito come la polenzia dovrebbe usare. Allora. Ora ne ho preparato uno. Più che avere la funzione donna. Ora la vediamo. C'è la funzione donna però in questo caso. È È funzione donna della nuvole elettronica. E E potenziale. Forse è questo. Ah sì. È È produttore. Quindi non si serve a stessa. Cioè ora ho preparato proprio questo. Che non la funzione donna della particella entrannte. Ma Ma potenziale. Cioè del lavoro elettronico. E E di potenziale. Cioè questo. Allora. Allora. Allora. Allora. Allora. Allora. Allora. Allora. Allora. Scrivo. Allora è che. Lo scriviamo di là. E poi uso quelle città all'alberra. E poi poi E l'elettron cloud. E no scusate. E' un elettron beam. Scarpes. On. On. An elettron cloud. Scarpes. By. The wave function. C. P.R. Uuale. 2. Diviso la 10 in 4 4 greco. E molti. E alla meno R sua. Diviso A alla tremente. Sip. Punto. Even that. Secondo potential. It's given by. V che diventa R, settore. Uuale. Integrale. Integraere primo. E 4. Diviso. R meno R primo. Modulo. E multiplica R e. Ti diventa R primo. Qui. No, è per roso. Rò viene il primo. Elettron delzi di ottengato. Punto. Punto. Calculato. Fin. Forn approximation. Forn approximation. It's very fresh at cross-section. La variante si. 
   
   At the end. Write it. Alla. Transferred. Cross-section. For a function. E per grande, nì, tu guarda. Quinto. Abbiamoci bisogno. Quello è il modulo del momento di tranzo. Niente diverso di questa vera. I. What do you get? For. U per A che è 3 è 0. E. Explain. Come è che si capta un dubbio? G. Teta. What is Santa Gada. The differential cross-section. Decrisse. By a factor. 3 elevata alla quarta diviso 4. When. The beam energy. Increases. From. 20. Electron volt. To. 40. Electron volt. We will calculate. The value. Of A. In. Armstrong. The. Allora. Entrate scritto. La pensare. Allora. Chi è? The. The. I. What so. The. Andra cosa. Dovrò sapere. La. Seconda. Una domanda. Non. E. Il. Con. M. La. Qualcola. E. Il. La. C'è. La. Lo. Il. Questo. Perfettino. Perché non è. Questa. Anche. Il. Non. Quindi. Questa. Quindi. Quindi quello che dobbiamo fare è iscrivere una certa. Cioè non una certa. La sezione di tutto. Differenziale. Una postinazione di porn. Spittacone. Sezione di tutto. Differenziale. Di rato e corno. Per. F. Di quadro. Uscite già interpretare così a questo test di quadro? La mia notte. Vabbè, c'è il mio volo. C'è il mio. Allora. E quindi abbiamo detto che. Desingno. Mio omica. Si calcola. Come un tuo quadro. Della piazza di scattering. Interamente questo cosa si chiama la piazza di scattering. La piazza di scattering. In una capossimation. Meno. M di miso 2 di greco. H del d'altro quadro. Integrale. Nel 3m. Elevato. Di. 2. Di. 3. Di. Di. Dove? Q. Votetini. In questo caso. Come capa? Meno capa primo. Quindi Quindi Meno finale. Se vi ricordate di trovare negli ibbi o nei vari appunti o negli esercizi. A volte scritto così o scritti molto stritte al contrario. Quello che dovete essere attento alla presenza o meno al signor. Meno qui così. Involte a essere consistenti. Quindi. A questo punto sostituiamo le spessore correziale. Meno. M di miso 2 di greco. Cattagliato. Integrale. Di. 3. Elevato. Di. 2. Di. Integrale. Di. 3. Di. 2. Di. Di. 3. Meno. Meno. Per Per densità. L'ensità. Elettronica. Si Si da per prima. Ho dimenticato un po' cosa cosa per voi. Infosso il problema c'erano dei lì. Voi vi aiuta al tesso perché dice. Nella formula di borgno. E' un un di variabile. L'unice c'è cosa da fare. Dice di fare il cambio di variabile in Y. Nuttuale a R e M, R e M. Questi sono lì. E poi. A questo punto. L'integrale. In D3Y. Di. E. Y. Di. 2. Di. 3. Di. 4. E l'altro lì. Perché l'integrale. 
   
   L'integrale in R. Per R. Di. E' elevato meno 2R di sua. In un punto di 2 L, R si è l'assimilano. In modo di fare i diversi. 0. Di. 2R. Nuttuale. 4. U. 4. A poco. Divito 4 più. 4 4 4. Più sopra e retesi elevata per l'altro. Ok, quindi quando arrivo a questa. Espressione. Penso subito all'indegnero. Faccio subito il cambio di variabile. E mi aspetto che vedrei trovare un integrale del Geno. Se non lo vedo già che c'è un intero al Geno. E quindi facendo quel cambio di variabile. Questo chiaramente. R è uguale a. Più sono il più Rv. Questo è l'esponenziale. Che va all'interno di R. E quindi ci ho passato sostituendo. M. E4. E4. Più che è è è è Più grieco è scaldato quando. Interior. I. Il n n 3. R primo. E elevato a i. Q. R primo. C'è questo qua ja? E diciamo che ancora io ho costruito questa. quindi questo questo un un di di di di quindi questo è è punto di di di di quindi quindi quindi quindi , quindi quindi quindi quindi punto punto punto punto punto quindi quindi è è è di di di tera, quindi quindi  questo è un punto punto spiegamento spiegamento spiegamento quindi quindi quindi quindi quindi quindi spiegamento spiegamento spiegamento spiegamento spiegamento spiegamento spiegamento spiegamento spiegamento spiegamento spiegamento spiegamento quindi questo è un punto punto punto punto punto punto punto punto punto punto di di di tera, quindi quindi quindi quindi quindi di di di tera, quindi questo è  spiegamento di di quindi questo questo questo questo questo questo di di di di di di di di spiegamento di di quindi questo è un un un spiegamento di tera, quindi questo questo questo punto di di di di di di di di di di di di di variability variability variability variability elevato a 2 e reprimio a 2, e le riprimi al 1. Si, questo è il termine che veniva veniva dall'arro che devo fare in quadrato di quelle spessone che hanno nel testo. Questo è un esempio l'itica. A questo punto posso fare i cambiamenti variabili che sono sono speliche. Quindi ho meno 2m² di riso pi greco, h dal dato quadro, di quadro a 2, integrale dal servo infito, in dr per l², elevato a meno 2r riso a. Voglio l'integrale in f, che si può fare in quim di metodo pi greco, e arrivare l'integrale nell'angolo popolare. Quindi Quindi dal meno 1, 10 centeta, elevato a iqr cosenteta. Eggi lo so, il primo termine è 10,2, 10,2r riso a. E quindi tutta l'espressione diventa meno 2m² di di a. 4 e quadro, diviso quale terza alla terza, se mi rimane l'integrale radiale, integrale dal 0, infito, in dr, dr e alla meno 2r riso a, seno di quim r. Cosa faccio ora? 
   
   Un solito. Chiaramente è uno che non è è solito. Però, quello che che dire, che in realtà non usa, questo espressione rilascia qui, ma se avete un lit, dovete arrivare alla forma del lit. Questo è un'altra cosa cosa è abbastanza facile, in altri casi può essere più spesso, lavorandoci, pezzandoci. Quale? 2m, 4 e quadro, 2.4, in molte fisiche riso antica di un integrale, 4.2, e qui abbiamo trovato l'estre, ora vediamo la sigla in destino in D'Omega vediamo l'estre l'estre l'estre l'estre l'estre ora sono arrivata quasi alla fine della soluzione prima perché ho calcolato nella presenza di borne, la sezione d'utile esperenziale ma mi chiede qua di farli in termini della sete di Radeford chiaramente non se la ricorda, voi avete un libro all'esame, quindi potete andare a guardare ma che cos'è quando abbiamo la situazione di Radeford? quando lo otteniamo quando è il botticone? quando è il botticone, quindi uno scatte di una particella carica quindi questo cosa deve essere proprio bene però dove abbiamo un potenziale colorombiano, il il cioè R-0 quindi la differenza è che che all'ere qua è rilgata a 0 dove tutta la carica che sia una particella è concettata qui quindi quasi in più più una particella o invece una certa nuova elettronica attorno a questo R che ha quindi una certa estensione che è certamente data la revengide potenziale allora, ora già capisco che questo termine più che otterrò questo FDQ quadro è quello che si chiama fattore di forma che mi da informazione sulla forma, l'estensione, le caratteristiche del potenziale dove un'elettronica non è un un quindi nel libro, non si ricorda che la stazione è tutta il di Radeford è uguale a un 16esimo se scrivo generalmente quando scrivo direttamente equato energia, quindi equato di energie tutta attorato di teta menzi alla in realtà non lo scrivo più generale, così non è di quello che ripartete che possono una vita da un po' tegliei esercizi in generale ho zeta e per zeta prima e ok, cioè sto considerando che sia la particella incidente che è la particella che si trova ai rupo alla zero, quello che si è toprima possono avere cari da zeta e o zeta e l'altra zeta prima e in questo caso specifico si è toprima di teta prima solo quella 1, perché sto considerando un elettrone che scatta e colita una nuova o un'elettronica quindi nel mio caso zeta e zeta prima, 
   
   quella 1 e quella la testifica altra senti, più che significazione, ora io devo in qualche modo volerare questa o questa di chiarimento è subito cosa si può fare? cosa possiamo scrivere? l'espressione di Q non si potrebbe scrivere come 2k seno di teclamenti perfetto, proprio questo quindi Q è uguale è uguale al 2k seno di teclamenti vi ho fatto vedere che così trova questa? no, non lo possiamo fatta con questo tutto che l'avevo fatto lo facciamo perché è ben successivamente questo l'ho fatto e lì ci trova però magari uno può essere utile perché a volte ci sono degli esercizi in qualche modo una cosa simile si si ottiene cioè è semplice però questo vero per avere la fresca questa cosa quindi, principalmente per te è uguale al 2k e a k, k prima io considero il quadrato e se lo dico che ci lo rinottiamo spesso a qua l'altro uguale a qua l'altro k quattro per le le damezzi e quindi questo viene veramente k quattro più k quattro prima quattro meno 2 posso esplorare k quattro prima qual è la relazione tra modo di k e modo di k quattro prima? sono uguali e qui tra i due c'è una moda e a me ne ho per tempo k che poi va in k prima le relazioni sono uguali a questo entero quindi questo qua posso scrivere come 2k quattro meno 2k quattro cosa intenta? diventa 2k quattro 1 meno cosa intenta? ecco, è una moda ricorometrica potenco quella quattro quattro quattro se lo dico se è un quadro di tete damezzi che però posso fare maledicio attento quello che ho scritto lì quindi a questo punto la volta in cui ho sostituito a q o al contrario da questa poi posso segnare più qua ce l'ho preso se sostituisco q mi serve il k il k lo trovo qui questo qua ce lo possiamo scrivere come se lo si basso in esercizio è alla quadra 4m quadro diviso a quadra 4k quattro quello che mi serve 1 se è un quadro no, se è un quadro se è un quadro di tete damezzi e quindi 4k quattro se è un quadro di tete damezzi ecco, non mi serve tutto il 16, 16k quattro se è un quadro di tete damezzi e il 8 quattro quindi tengo 4m quadro a quadra 4 e quattro 2 quadro mi mi si è clarinato è molto così è facile vedere ok il tese dima il diromico è uguale al diromico il diromico girato e cortico quello che ho spitto per una quadra che sarebbe 16 quattro diviso 4 più quattro a quadro elevato alla quadra mi mi so il mio 2 e per il quadro 2 2 più di onde 2 2 2 2 come si attiene quando considera il q per a per 3 e a 0 per te ti spiegate insieme al punto matematico è più semplice cosa si attengono quando che la seccia è d'urto diventa quando ti rada il fordo e basso sì quindi quando il quadro è uguale a 0 cioè quando coopera a tende a 0 sotto abbiamo 4 alla quarta che si dice al quadrato e quindi otteniamo questo termine qua 1 questo qua 1 e quindi ritorniamo con il quadro che il diromico uguale a il diromico girato e presto e quindi ha più di soluzione che significa che il fascio incidente vede la nuova e la nuova di quasi il fascio incidente vede 
   
   la nuova come puntiforme e cosa stavendo perché? che voi so le caratteristiche in questo fascio o la nuova e piccola o la lunghezza da bel fascio perfetto quindi veramente questo questo significa che ne dobbiamo confrontare a con cioè dobbiamo confrontare a con uno sotturro e quindi quando puntiamo alla fine, dopo il colore da B, lo puntiamo con la lunghezza da bel fascio quindi confrontando con la lunghezza da bel fascio come avete detto se la lunghezza da bel fascio rispetto ad a è grande la vedo come si è spostata un soggetto di tutti i buoni perfetto posso rispondere questa cosa? come? posso rispondere questa cosa? allora, intanto il passaggio prima di ancora di imparare la discussione in fisica il passaggio matematico ci sono pronti allora non abbiamo trovato in questo limite abbiamo la sezione tutto di raterford ma avevamo dettagli, cosa è la sezione tutta di raterford? è la sezione che troviamo quando delle particelle cariche in questo caso l'elettrone vanno a fare uno scattering con una soggette puntiforme di carica quindi quando in qualche modo lo stiamo più considerando la lunghella elettronica ma proprio una soggette puntiforme una carica in tutto il lupo del zero e quindi cosa significa? perché quello che abbiamo in realtà è sempre questa lunghella elettronica quello che ci significa è che q per raga scende al zero significa dire che, allora aspetta questo è un costruzione questo è un costruzione questo è un costruzione ok stiamo confrontando stiamo dicendo che q è molto minore di almeno 1 ok? ma q o al contrario ah è molto maggiore e q ma quello è almeno 1 che cosa è collegato? è collegato alla lunghezza donna che è collegato che collegato che che è collegato k con lambda come è chiamato? come si fa lambda la? lambda tagliata lambda subigliela ah un nome è fatto ah dite più lì quindi stiamo confrontando la lunghezza della lunghezza del fascio incidente che è collegata a q e quindi è collegato alla collegato alla che è quella che compare quindi è collegato alla size questa, la dimensione di questo l'elettronio, stiamo dicendo che è la lunghezza donna del fascio incidente è molto grande rispetto a da perché quando parliamo di grande pippo dove puoi sempre fare rispetto a cosa se dire semplicemente la lunghezza donna è grande significa grande la lunghezza donna è grande è alla size della portata del potenziale allora questo non vede la sua estensione, ma la vediamo se è possibile poi è più lunghezza donna deve essere più piccola la lunghezza donna deve essere più piccola no, più grande, la lunghezza donna è più grande della portata del potenziale ok questo è una cosa che abbiamo spesso in fisica ma più tu hai una lunghezza donna più riesci a vedere a rivelare la struttura del tuo campione, in questo caso più se tu volessi andare a guardare 
   
   il piccolo sempre più microscopico nel mondo mitoscopico devi avere una sonda che ha la dimensione confrontabile con questo sistema che puoi studiare in questo caso se invece la tua il suo fascio che è sperimentale in qualche modo che ti deve rivelare il tuo campione che ha una mezzoglietta molto grande non riesce a risolvere la traduzione dell'inglese non riesce a vedere le differenze quindi riesce a vedere soltanto una lunghezza elettronica cosa fosse tutt'altro e questo lo vediamo, abbiamo visto ma che risultate anche in realtà abbiamo un partilimite in cui coincide con il tuo caldo che si è scritto altre cose che dimentichino ah, dottor, mi è venuto grazie un bel punto ok, ora passiamo all'altro punto anche il domande sul piatto nel tuo punto si è chiedetto ora, si è considerando un angolo specifico delta un angolo di scatte di tè e poi sta dicendo che quando quando l'energia passa la mente elettronvolta a 40 elettronvolta la sezione tutto, differenziale diminuisce e c'è il fattore se non per più diminuisce e c'è in questa condizione otteniamo informazioni sulla size a della lunghella elettronica e quindi chiedete di calcare come questa e poi lo pensi anche, fate l'ora in angelo ricordate quanto è un angelo 10 al melo 10 perfetto, e quindi a livello di size in direzione cosa è la dimensione di cosa è già domino dell'altro un altro, per del loro dimensione lo dici alla livello 10 tra i due, tre e quindi così e se ci sta dicendo questo è l'elettro di questa disintoteggione ci sta dicendo poi alla fine di confrontare a queste dimensioni che troviamo di questa nuova elettronica con quel lato in altro, ma sia molto più lato tutto picco alla stessa dimensione ci otteniamo poi un confronto di tendenze fisiche e in questo modo si modo iniziato a realizzare sia realizzare lato in alto in quindi si tratta di tratta di alto e quindi e quindi in alto e alto e alto e quindi in quindi in quindi in alto in alto pedici che vedrete allora,
   
   quindi si sta dicendo che prendiamo del 60° e otteniamo che si mette ora si mette ora ma intanto che lo sappiamo che ha tendo a 60° ma non cancellate quelle di mirafela vede voi nel quaderno intanto abbiamo che 4 uguale a 4k4 5k di 2.5 e quindi subterresto uguale a 4k4 a crescendo di 5.5 per il seno e quindi risulta k4 e quindi chiaramente in fondo abbiamo l'energia che ha tagliato k4 diviso 2m per questi risultati h4 q4 diviso 2m quindi in tanto alla relazione tra l'energia e il q4 aumentare l'energia e aumentare l'energia di un fattore 2 che è quello che stiamo facendo lì passando da vendere il 3V a 40L e questo è l'antozione del faschio aumentare il fattore 2 e equivale aumentare il q4 di un fattore 2 e quindi aumentare q, corrispondente a 4K3 radice di 2 e quindi aumentare quindi aumentare l'1V perchè chiaramente ora quando utilizziamo tutto ci ritroveremo sempre q questo è un giorno fatto una relazione tra q corrispondente ai valori che ci dicono il testo e quindi quello che ci dice il testo non dice il testo è che desigma in deomega corrispondente a un energia di 40L diminuisce il fattore e quindi quello che si dice è che il fattore 3 a 4 diviso 4 per 2 la seconda la rostruttosi penso che sia il fattore desigma in deomega corrispondente a un energia di 20L a questo punto sostituisco l'espressione di 5 in deomega che ho trovato nei punti precedenti e ho teguo q alla quarta dove q è quello corrispondente a 40L un bambino tilica 4 più il quadro di 40L a quadro 1 uguale 2 alla 2 e alla quarta qui c'è un elevato alla quarta che l'ho scritto e qui ho q 20L a quarta e modifica 4 più più alla quarta di 20L a quarta a quarta e ho sostituito l'espressione di 100L a questo punto ho fatto la reggibatta lo faccio qua stessi quindi questo è la via questo è la via questo è la via questo è la via questo mi diventa la 12L questo è la via quindi ho tego q di 40L a quarta e quadro di 40L a quadro uguale q di 40L a quarta e modifica 4 più q di 40L a quarta e il quadro di 40L di l'ho scritto a cuocere e ho scritto a cuocere e ho scritto a cuocere il quadro di quadro di questo punto diventa questa e ottengo dopo salto un passaggio ma quando io vado a casa 4 più 2q al quadro di 20L a quadro uguale 6 più 3 mezzi 4 più 2q al quadro quindi a questo punto si trova una regazza tra 2 e q a di solo 2 q di 20L che è quello che c'era il problema perché la piccola che ci sta chiesto di calcolare a questo punto dobbiamo semplicemente determinare qual è il valore di q di 20L considerando la relazione che abbiamo trovato tra e q quando questa viene che il quadro uguale a 2m e a catacchia il quadro uguale a 2m facciamo i suoi passaggi in cui dobbiamo fare mc al quadrato per e diviso hc catacchia il docci tutto al quadrato abbiamo che fare con un elettrone quindi abbiamo mezzo-merzo in un mezzo 10 al 6 per elettro q per energia e 20L perché sono arrivati a questo lucido 20L e sotto hc al quadrato quindi 4 per 10 alla quarta elettro q al quadrato per 1nm al quadrato facciamo le nostre semplicitazioni questo lo abbiamo questo elettro q elettro q per elettro q per elettro q e qui abbiamo rimanere un 10 alla 2 per quindi abbiamo ottenuto 5 per 10 alla 2 a 9m alla 2 5 per 10 alla 2 a 9m alla 2 sì, per 500 a 9m alla 2 a questo punto possiamo alcolare a da quelle passione 
   
   2d500 3nm mi ricordi a prendere al nanometriano pure sul provare a 10 e quindi sul test c'è uguale a 0.09 nanometri ovvero considerando a 9m di 10 a 9m a 9m di 10 a 9m di 10 a 10 a a 9m quindi circa a 1m quindi abbiamo un altro proprio che questo è un'altra con le toniche che avevano una della sezione di un auto chiaro? dubbio domande? è in tartilata la loro mezzo non è in tartilata non capito sì, ok il mio coso è che qui siamo nel caso in cui stiamo considerando l'F più sì, cioè questo è qui che l'ho cancellato dovevo rivedere così di dicono ma quel termine questo termine, qua, qua, qua per la base di Cercodrato era in realtà proprio il fatto di forma quindi quando io scritto il predecessore di questo come iniziale è perché l'ha sezione tutto tirato e poi lo ho cancellata perché no, l'ho cancellata allora, ecco, c'è rimane sotto questo punto lo ho cancellato i fattori quindi diciamo, io ho solo il termine di concuro e poi su questi faccio i ragionamenti dopo quindi quello che ho fatto qui lasciando slare questa cosa che dice con dei valori specifici questi ritormenti rottati sono indipendenti al punto precedente poi effettivamente, come risultato abbiamo trovato che quando c'è questa relazione tra la sezione tutto alla ventra dell'eligia effettivamente siamo, in caso non si sia in un caso un però è una cosa che troviamo non abbiamo la stessa rastro ho sbagliato qualcosa no, è giusto perché non è sotto tipo o no finiamo la domanda senza interpretare può essere così bagnato deve essere 0,05 5 per piacere è più di 100 ah, si, infatti qua ho scritto 500 si, sono più stupi ho sbagliato un meno che l'ho 100 ok qua è il prezzo ma grazie per la domanda infatti qua questo viene 500 altre domande o commenti? solo una cosa in quelle trariche il passaggio, oggi, con la seconda e la terza rica mi sono perso qual era? dalla seconda o dalla terza c'è l'ogicor, non c'è somma, matematica ma matematica, intendo sono intenzione che ho fatto questa tipa una animatematica, perché mi vedremo in un momento in quel momento e ho semplicemente ah, ho detto a voce ho detto, sto applicando questa cioè la relazione tra q e 40 ah, ho fatto il rapporto e la ricorda di avvicinato ok, ok, grazie avevate scritto prima che era appunto con la ricorda di q che l'ho trovato dalle gambe o leggi altri domande o commenti? ok, ok, ma perso, dobbiamo fare una cosa ora se ci vediamo in qualsiasi, le tre, la ci vediamo in qualsiasi se vuole, lei, tu, lo vedo, lo vedo, è siccice, così no, la fine era questo
   \begin{equation*}
      \dv{\sigma^{\rm Born}}{\Omega}
      =\eval{\dv{\sigma}{\Omega}}_{\rm Ruth} F(q^2)
   \end{equation*}
   \begin{equation*}
      \dv{\sigma}{\Omega}
      =|f^{\rm Born}|^2
   \end{equation*}
   \begin{equation*}
      f_k
      =-\frac{m}{2 \pi \hbar^2} \int \dd[3]{\vb{r}} e^{i \vb{q} \vdot \vb{r}} V(\vb{r})
   \end{equation*}
   dove $\vb{q}=\vb{k} - \vb{k}'$
   \begin{equation*}
      f_k
      =-\frac{m}{2 \pi \hbar^2} \int \dd[3]{\vb{r}} e^{i \vb{q} \vdot \vb{r}} \int \dd[3]{\vb{r}'} \frac{e^2}{| \vb{r} - \vb{r}' |} \rho_e(\vb{r}')
   \end{equation*}
   $\vb{r}=\vb{y} + \vb{r}$
   \begin{equation*}
      -\frac{m e^2}{2 \pi \hbar^2} \int \dd[3]{\vb{r}'} \rho_e(\vb{r}') \int \dd[3]{\vb{y}} e^{i \vb{q} \vdot \vb{y}} \int \dd[3]{\vb{r}} \frac{e^{i \vb{q} \vdot \vb{r}}}{| \vb{y} |}
   \end{equation*}
\end{soluzione}

\newpage
\setcounter{equation}{0}

\begin{esercizio}
   A proton moving along the $z$ axis scatters on the potential
   \begin{equation*}
      V(\vb{r})=
      \begin{cases}
         V_0 \delta(z) & \text{if } x^2 + y^2 < R_T^2\\
         0 & \text{elsewhere}
      \end{cases}
   \end{equation*}
   \begin{enumerate}[label=\alph*), leftmargin=0.6cm]
      \item Find the transition amplitude $f(\vb{k},\vb{k}')$.
      \item Consider the low-energy limit and calculate the total cross-section
   \end{enumerate}
\end{esercizio}
\begin{soluzione}
   Io sono un esercizio e inizio a sfoggerlo voi. Cosiamo che vi mette dall'arrore, ma ma abbiamo già fatto, sempre su Staterintiore, e poi non lo sono messo, ma ne ho fatto con la scosta volta, io ho fatto uno, vedi, la vedete tre, i concetti, io poi magari ne abbiamo lo... dove inizio. Cosiamo che vi dico dico Non è tanto tanto li piamano, non è tanto che li li Ma poi quel paio di azienti, ma non è è dice che è un po' mi dote, ma che è, ora questa è una bella nuova. Cioè, stanno male, stanno male. Ah, che ha preso il mio benzo. Che Che dote? È un bordo martedì benzo. Allora, spettando. Che passi, se non mi dote, poi iscrivo. Ieri, chiunque, stanno bene. E' pronto? Ieri, chiunque, e' ieri. Cioè, se mi dote, mi dote. E' l'onga del zeta maxi scatters, scatters. On the potential p dr che è uguale a p0 del zeta per x quadro di x1 quadro in uno uguale di rt quadro. F0. Tutto a find the transition f in other words. F in k-pop ecco al primo p e considera the low energy limit. E' the total procession. Più. Alla, c'è l'altro punto, non lo do per farlo fare a voi, perché sulla passione wave analysis che appunto non ha ancora fatto, perché ci sono alcune stridizia anche sulla passione wave analysis, perché non c'è un momento Potete collaborare, chiaramente, perché non è un esame quindi, per stare su te e per il primo perché appiola il limit, per accedere a domande, così, per gli oggetti che sapete come costavano, per vedere il più orta a leggere, quello leggo. Limit. Si, per il below energy limit e calculate the total procession. Poi, in questo questo se la diceva che la diceva che i due due non c'è un altro, però la dicevo più. Ancuno? E non diremo questa roba. Se lo facciamo una questa macchina di porto, quindi con la salsagolla di bevo. Sì. Sì, non c'è la salsagolla di bevo. Sì. Sì. Allora, che simitriava questo potenziale? Cilindrica. Cilindrica. Mi attenziona l'osterica di poter utilizzare i cornelli di cilindrica nella forma della fascinazione di tuo. Sì. Sì. Sì. Sì. Sì. Sì. Sì. Sì. Vado Vado vedere vedere po' di di La teglia si viene in un un Non è è tanto che avremmo esercizzi con una delta, due delta e due due due due due due c'è un lavoro in un un Sì. Non c'è qualcosa di buono. Non lo stai a sistirci? Non hai ristimato. Allora, le cartone dei cilindri che me le ricordate? Sì. Ora che ne scrivo? come va? è ancora costa dopo il giudice tempo però ovviamente non possiamo stare qua due ore non ha senso perché il previsso è andato avanti quindi voglio però che iniziate a riflettere da soli per trovare un po' in situazione segli dalle delle tante questo totale è un giudice di 1,3 che è un giudice giudice 1,3? sì, sì, sì, revo scuida del z di per direzione antostattuale qui come va? vabbè, vabbè, vabbè dobbiamo ancora ripostare non si è revo scuida del z prego il mio esperito si, si, si, è vero che hai capito che sono erdi diversi quindi dobbiamo fare un rom però tutto è consente del z ok, questo è il z qua però c'è un rom se volessimo un cambio di due e e x è e se c'è qualche problema il z è in gordia di x questo è il r questo è un r, spero quindi è il z si, si, si questo poi è r, spino e esperito che non lo finisci un r che si in un un più vero vabbè, compi, anche r non è vabbè, non si può nonare un z si, si, si e quella di un r se non lo fatti di quanto è cosi non è una gran corazia quindi diciamo questo in grado e poi ho la dichicolata di intende come scrivere quel rapporto scala, quello è un r scala, io lo concediamo prima non lo ricordo ricordo un r, soprattutto il nostro c, c'è c'è un piccolo r però però non non non ci sono sono cose stra più per cambiare il nostro no, non è non è non è non è non è una una una non è un r, è un un non è un g, non è un g g è è è è è è è è è un r, non è un un un è è g però la mia scusa è apensane sì non devi 
   
   dire perché potenziali tutti i formami e tutti i formami in quanto quanto cura mi ho scelto io io allo spazio interno di te Hai un un che è un valore che è zeta in zeta, c'è il 0 per il punto. Solo in un punto. No, non posso disegnare. No, sono un'asse. Non posso disegnare, in realtà posso disegnare. Non posso disegnare. Non posso disegnare. Si, ma si chiama sessualore. Faccio un colore di questo del cibo, sopra il colore, alla gradazione del colore. Possono doggere il colorato e il non colorato. Quindi Quindi realtà... Nel cerchio sul piano z vuole il 0 e di una giurrettiva le v0. Perfetto, e poi il z. Quindi intendo dobbiamo vedere il riuscire il maggiore e il maggiore. Si, quindi sono i miei miei Quindi, stanno le tante, per l'algo della fortina. Non risulta che l'ha l'ha bene, però se avete capito poi di questo il mio x, y, z. Qua ci vogliamo x, y. Ho un cerchio di raggio rt, in questo caso in in con un cerchio di raggio rt sul piano x, y. Ok? Non, un cerchio. Certo. Di raggio rt. Certo. Ne mettiamo x, x. Quindi qua in questo cerchio, che è un valore di 0, il colore è 0, il colore è 2. Qui, b0, e 2.000, prima o 6.000. Quindi perché c'è un simpio di ghiacilindrica? Perché veramente con un valore di z ho un simpio di ghiacilindrica e z z tutto, e qui invece ho una simpio di ghiacilindrica. Però se mi sposto lungo z, è chiaramente cambio. Quindi non ho la simpio di asferica. Sì. Ma sui cilindri vendi oricina erano i... Ma solo perché io però... Ok, allora scriviamo la... La etica di Pornia, che è un posto scritto in le volte, da N2, direi che ha avvertato il corso, integrale in D3, e lo chiamato in stimo. E' un un lo spazio di l'impliante, lo chiamato in P2, e lo chiamato in P2, e levato a I, e e a X primo, VX primo. E questa è la prima cosa che non deve fare quando l'esercizio della Pornia possima. Se in meglio c'è il potenziale simpio di asferica a posare quella semplificata col z? Giusto, perché per per la notazione sarebbe quima la in terra e prima, e con con abbiamo detto... C'è l'insenso, siamo allo spazio e si intende R vetto. Sì, sì, sì, E' semplicemente spazio. Sono in X, lo chiamo 2X, sì. E' un internavole spaziale in 3 dimezioni. Qua l'ho chiamato il sprimo, ma l'ho chiamato sprima apposta, perché poi invece dopo avrò il D3, per non confondere quello che è vero in mismo. E' assolito lo specifico sempre, questo caso, questo sempre, e questo questo questo questo iniziale. Il momento in cui passa. Allora, quindi il mio mio R X, il momento in cui cui R X primo, sarebbe un vetto, un spazio di dimensionale di componenti X, Y, Z, con questo sistema di infrimimento che usiamo sotto. Quando passa a chiungliare cindri, questo X, l'evolutina cindrica in tanto per essere R, P e Z, quindi devo spiare quelli che chiungliate cindri. R, P, Z, R, P, Z, R, 
   
   quindi ho l'angolo azimutale, ma non ho l'angolo apolario che mi generevo in genere, quindi veramente, quindi l'integrale, che che riusciamo a chiungliare, per essere specifico, l'integrale deve essere il X, l'eterno del Z, che va da N, P, P, N, P, N, P, N, P, P, e questo sarebbe quindi l'integrale che è lì, integrale in delle istimose di tutto spazio. Questo qua, quando quando so da spore, il cortinato è cilimerico, che vende integrale da 0 e infinito di R, di di di R, l'integrale da 0 al più greca, di P, integrale in Z, da mani un filo, da P. Quindi a questo punto, il mio P diventa quindi un P di R, P e Z è uguale a P0 del Z per R minore o uguale di R, P e del 0 Faccio un termine passato che quello che abbiamo messo è stato chiesto del pronto scalac UX che lo dovrei fare. Q, X, 1 fa cosa uguale, dato che le comune di X sono queste. UX, X, 2 UX, UX o Z. Questo sono già che rimane possibile perché la morgolazione non è cambiata. Deve sprivere questi sarelloti in Z, Y sono queste che sono localmente Q. Quindi lo pò qua utilizza il fatto che oltre a quelle vale anche QX un certo QT quindi la prevezione di Q, che puo puo quello che ha un Q transversio con S, G e qui sono uguale QT sempre. Iniziamo questo iniziamo quando ottengo QT, R cosa è un quadro di Q? più QT, R è un quadro di Q più Q zeta zeta quindi QT, R più Q zeta più più zeta più più più Bene, ora provate a procedere, va? Ci voleva un po' di immaginazione, ma non è stato. Quella viene anche allenando, sì, per sé. Cioè, il primo primo esercizi più grande, da lì, da lì, da da e no. Allora, il discorso dei coordinate cilindri che mi avevo detto io, che mi avevo detto io. Quali sono le coordinate cilindri? Che Che usciva un coseno il seno della X, la X, però non ho avuto un senato di scoporre anche QX. Perché se ci pensa alla fine quello che fa con il colettore cioè, il vettore X non è che speciale, un vettore qualunque non è è è piano caccello, non non l'ordinate del scoporre. Non è magari immediato, però. Ok, allora se le prende che si lo che vuole, se le considera a Cintro per disegnato, deve considerare la prima prima Se considera a D, con il colettore, deve considerare la seconda verità. Quindi, dalla seconda verità, le mie favoriche a C, dalla prima verità, le mie favoriche comunque che a D, però le mie favoriche non dalla seconda verità. Cioè, forse rispetto a quello detto prima, posso gli biorare in maniera proprescente a questo punto. Però, la teta, che sono 1,0 ,1,0 ,1,0, vettore fa 1, per un numero diverso da 0 perché, come ho detto detto è un regalo di z, di di di z, di z. All'iporio di z, da 0, con 1,0. Sì, questo lo deve, con tutto il riuscito, da fare con potere. Poi, chiaramente, lo 0 fuori, lo 0 qui, quindi, per un'anistanza minore di RT. Non lo ottengo da questo. Lo ottengo da quello sopra, questo, perché, ovvero, qua, vuoi risultare facile, ma vediamo che ho ho però poi ne trovo un altro problema di ciliblette. Noi, da questo z, da 0, è elsewhere rispetto a un celtio di raggiuretti. Quindi, fuori dal cilindro, è 0 per la seconda rica. All'interno del cilindro, dove z è da 0, è di 0. Dove z è da da e 0. Sperto. C'è una pezzina, no? Cioè, come si è tagliassola. 
   
   
   Dico ha fatto solo il minico comune multile, è questo? No, no, no, solo ho messo la parentesi i e tutti, vedo che c'è un tutti qua. Ah, perché là è il quadrato, vabbè. Sì, perché era un senso diciamo tutto per quello che ti stiamo vedendo, va bene, tutti qua e tutti qua, quindi diciamo tutti non messi l'acqua. Sì, è un senso un buono. Però vabbè, anche se avete avuto questa espressione, va bene, è stato il terior che ho fatto, perché ci sta in grado. È una cosa. Ok, vabbè, forse eccoci, ve lo dico, diciamo, principalmente che quando vedete, non so se avete notato, ma quando vedete i sacciti scatting, quelli che avevo fatto un quadrato a prossima ecce, che abbiamo dato con la relette, troviamo sempre questo termine, davanti al fattore, o NDH4, perché termine comunque anche il quadrato posiziale. Quindi quando non trovate termine del genere, potrebbe essere un errore, quindi sapevo un po' di cose che, ma qui fa la sezione, perché sono ricorre. Ma come inivita passaggi, e possiamo metterci anche gli esponenti del ruolo 1? Ecco, dite me lo vuoi, vabbè, il primo punto ci siamo, possiamo andare avanti? Sì, sì. Ok, secondo punto, qual è il diritto dell'o energy limit? L'o energy limit, che significa? Il capa è molto minore di 1. Molto minore di 1, sì? Capa e arricchio. Capa, rege del potenziale, in questo caso, è un ticombo minore di 1. Perfetto. Il collega mi ha chiesto, posso metterci un po' a uno? E' guardato in questa espressione, dove è che io trovo il... In questo caso diciamo il capa che cosa è collegato. A cutino. A cutino, quindi io, fondamentale, ho un Qt, Rt, molto minore di 1. Dove è che trovo Qt a cutino? Nelle scuole del ziale, qua e qua. Quindi posso semplicemente affrontarlo come uno? È quello per vedere. Eh, no, io sto richiedendo, sto rifacendo dove hai fatto questa analisi. Io lo farei, però, non lo segnate il giusto. Ma questa cosa è simile, l'abbiamo trovata forse nel primissimo esercizio che ho fatto, o il secondo, quello là, nel potenziale che non abbiamo quantificato, in cui, a c'è una cosa degli Alfa, se ti ricordate, 
   
   i termini che ho scritto non è Alfa, e poi se si vedete la soluzione che era proprio posta qui, c'è no, posta diversa, questa. Cioè che se mi fermo il primo termine, ok, perché c'ho uno su Q, il secondo. In questo caso, io qua ho il Qt quadro, quindi arrivo fino ad il primo. Il primo, cioè il primo mi devo fermare. Il secondo c'è il secondo. Il secondo. Uno più, uno più, quello con Qt più, si termina con Qt quadro. Quindi mi fermo al secondo, perché c'è un Qt. Poi dopo, alla fine, quando c'è un Qt di calcoli, non mi riferino a dire niente di termini, Qt quadro e di calcello. Non lo faccio dopo, non lo faccio prima. Ok, ok. Quindi io volevo eliminare subito questo termine. Però, no, avevo sì fatto dopo proprio perché è la presenza di Qt quadro. Ok. Quindi in questo caso, ci fermiamo a... Secondo termine, io vedo che c'è una potenza Qt quadro. Quindi, ragionmente questo. Come faccio a trascurare qualcosa che all'ordine Qt quadro, quando io c'è una mia scrissola di potenza, ho il termine Qt quadro. Ok. Cioè, il gioco che si espliziona, dice che per il momento non lo posso trascurare, perché c'è l'ordine Qt quadro. E quindi poi, in tanto io, non ha senso di fare, io potrei fare un Qt cubo. Poi dico, vero, Qt cubo è solo lì, lo trascuro. Però il Qt quadro, non ho un attimo, c'è un dominatore, non posso trascurare l'ordine. Io faccio il di calcoli, poi alla fine, magari. Quindi, passaggio a coretto e passare allo sviluppo in serie. Alla sviluppo in serie, è farlo fino all'ordine Qt quadro. Quindi. Possiamo dare una cosa che, a me, l'alto esercizio abbia tornato a sicuro, quindi fate girare. Sper' attento, ne prendo un esercizio. Quindi, abbiamo detto, via un Qt, e elevato a I, Qt, Qt, l'ultimo, più lì, Qt, Qt, meno, il mezzo, Qt quadro, è retraccolato. E quindi, guarda. Ti tengo. Voi passaggi matematici, che ci fate voi, l'altro, una volta che avete sossito tutto questo, sono passaggi semplici. E' metodo a zero. Si può fare che da un lato di Qt si fermo al primo al fine, il due Qt. Tanto arriva fino al termine Qt quadro in un piano. Ok. No, perché forse ricordo, ma io, ma la lezione forse era capitata, una cosa in generale in provvedimento, il prezzo era capito. Perché questo era al fine, se lo sopra dopo, ma se lo sopra dopo, il primo, potrebbe dire. Sì, perché se lei va al secondo ordine, come ho detto io, in modo d'essere consistente, poi trova che è lo stesso risultato se trascura fino all'inizio questo qua. Cioè, per questo, poi, al fine si può fare. Però, io parto dall'idea che in un altro al solo, una doppia unica espressione. Poi, in realtà proprio perché c'è il semio meno, chiaramente, lo posso fare, perché davvero non lo posso significare. Però, io su consiglio, intanto, di andare sempre con l'ordine che devo considerare in generale all'espesso. Allora, M0, T1, T, Ft, T, M0, Ft, al tolamo, M0, T2, Ft, Ft, M0, T, Ft, Ft, Ft, Ft, Ft, Ft, Ft, Ft, Ft, a questo punto, questo e questo, va via. Quindi, il primo ordine che ho trovato è quello in retti quadro. E, chiaramente, se non avessi inserito il termine, non avrei trovato lo stesso termine, è giusto? Eh, qua posso fare il ragionamento, dato che il primo termine non è nullo, è quello al quadrato in retti. Allora, quello retti, un po' dato che stiamo parlando di un retti molto minore di uno scuttino, quindi veramente lo posso trascurarlo o lo faccio. A questo punto, questo lo trascuro. C'è vero? Se avessi avuto termini in ordine retti, poto trascurare anche i retti quadri. Però, in questo caso, invece, quello il primo non è un retti quadri. Il mio esempio è C uguale uguale a M 0 2 H 4 R di quadro. Cosa ha noto a livello di dipendenza? C'è qua una stupro, era partita da un mestretutto che, come sempre, la trovo sempre che dipende da K, da T, da K primo, sul punto scritto, qui cosa ho trovato la trovo dipendenza. Quindi non dipende da K o da K. Quindi, va bene, che non dipende da K, se mi fai anche non dire che non dipende dalla T, che K c'è la pro-T, che mette le sue. Andato davanti. A me non risulta il meno, perché forse si cancella prima. Ma poi risulta il meno? E lo riconfogliamo. Il meno, il meno dove partiva l'inizio, giusto? Poi, qua, io ho preso la... Si, si, fino all'alto. Ok, dove hai preso il meno? 
   
   All'alto, all'alto, è il problema. Meno uno, quattro. Ah, qua, sei giusto. Ok, c'è me lo qua, perché non lo ha sentito. Ok, quindi, e dipendente a K o di T? Vi guardate, come io sto andando a guardare una situazione che di qua, una unità di scattering, che poi vi interessa la sessuzione di K, K primo, o T e Q. Cioè, cosa vi sto chiedendo? Mi sto chiedendo qual'è l'effetto dello scattering al variare del T? Un peccione, un gendro di scattering. Una zona di percento potenziale. Questa è la linea passa incidente. Rispetto a questa linea, in cui andrebbe la particella libera, se non fosse potenziale, come vi avete flesso? Cioè, come è coccurale l'ampiezza? Quindi, il peccione di T, di T. Ok, mettiamo il uso dell'immergigio incidente. Quindi, quando io trovo questo, sto trovando che la linea non è importante a quelle dette di escappare, o a quelle di te la stessa. Non è un peccione di escappare, io sto dicendo la stessa. E non so, perché la linea è un'energia incidente. Sempre, però, nell'energia. Cioè, piccole variazioni, sempre, però, con la passata dell'energia, non hanno alcun effetto nel risultato. Ok, ora, quello che ci devono, però, era la total cross-section. Quindi, dovremmo preparare due passaggi. Il più importante era la sigma, l'indegnome, perché quello che abbiamo fatto fino alla sempre, e poi la stessa matodra. E grazie a questo, è semplicissimo. Quindi, se devi sponere, più integrale. Il desigmio in grado, per essere, semplicemente, è il passato in grado, è il modulo 4. Quindi, su T, M, 4, 0, 4, liso 4, la teglia dalla quarta, l'Rt alla quarta, la sigma, sento che lo passate. Il sigma, ecco, è l'integrale, l'indegnome, della sezione di tutto l'isprezza, l'indegnome. Quindi, integrale, l'indegnome, di quello che ho scritto prima, che appunto non dipende da come è che è che lo sette e fì. Non dipendendo da sette e fì. E' solo 4 più di grado. E' solo 4 più di grado. Quindi, qua, di grado, per quello che ho fatto. E quindi, poi, semplificando, ottengo, di grado, sento qui, e mi porto, l'indegnome, la teglia della quarta. Ah, prende la auto, alla quarta. Va bene, domande, dubbi? Sono una, qual'era il punto C? Eh? Qual'era il punto C? Il punto C è che si può fare con la parciacola anali. Spero c'è tutta una storia che chiedeva di calcolare che questo... Che è difficile. Chiedeva di calcolare il fascista, che ora faremo in altri esercizi. Ah, ok. Però, per esempio, qua, c'è si può dimostrare che, in questo caso, si può fare. Però è un po' compriato, dimostresca, come non è uno sospiro che una cosa che, per esempio, non avete inoltre in teoria, non volevo toccarlo, perché sono già costruito... No, mi spalto un dubbio. Però, dicono, questo non lo deve spendere poi, ma perché mi risolto, soggia il dubbio, se può usarlo o non l'intershift? S'atticipando a quello che dobbiamo dire dopo. Quando è che si può stare la parciacola in manazza? Caso di poter usare? Ah, si mette in... Sì, sì. C'è qualcosa di sicuro di poter usare. Questo caso che è simulato? Il massero specifico, da dove sto considerato, in detta, perché quello che deve era, soltanto quello di oggi, che passo, si può fare. Però, diciamo, perché se questo non lo deve fare, scumento, riamo un personale, si è toccato. Previste che parlo con un esercizio in cui è chiaro che siamo al mario. Va bene... Non lo resisti. Ci dobbiamo... C'è un'altra. Sì.
\end{soluzione}

\newpage
\setcounter{equation}{0}

\begin{esercizio}
   Given a proton scattering on the potential
   \begin{equation*}
      V(r)=
      \begin{dcases}
         -\frac{V_0}{r} & \text{for } r<0\\
         0 & \text{for } r>0
      \end{dcases}
   \end{equation*}
   \begin{enumerate}[label=\alph*), leftmargin=0.6cm]
      \item Calculate the differential cross section $\dv*{\sigma}{\Omega}$ in $1^{\rm st}$-order Born approximation as a function of $\vartheta$ and $E$.
      \item Calculate the phase shift for $\ell=1$ in the low-energy limit.
      \item Given $V_0=36 \; \rm MeV \, fm$ and $a=2 \; \rm fm$, what is the value in MeV of the proton energy to have $\delta_1=\frac{\pi}{10}$.
   \end{enumerate}
\end{esercizio}
\begin{soluzione}
   allora, faccio notare una cosa che mi hanno chiesto, io volevo essere interessante questo è un altro colpso, ma la sensazione è tutto che è in misura la dimensione di un'area quindi per esempio in misura molto trovate il barn che è più più 100 fermi quadro quindi come è quello che lavoriamo sempre on, on, on, fermi quindi chiaramente ho scritto la potenza, ho scritto a 20 mili barn, con barn 100 fermi quadro però se ti investi in un'applica, queste cose qui in maniera immediata senza ripetere su alcune cose trova che è dimensionata, se usa jeb per v0 come in molti altri esercizi però vi voglio notare che qua v0 non ha le dimensioni di un jeb no, m è per fermi, no? è un jeb jeb fermi perché per fermi? perché io potevo avere jeb ah certo, perché c'è la delta, giusto e quindi queste cose che c'è, lo attengo io certo quindi se inizio questo, poi lo fai più controllo l'altro, ho il tempo di fermi quadro quindi questo controllo effettivamente, fatelo sempre perché è un un ottimo di vedere la mia insidia dimensionale, di vedere le cose cose ho attenue se non trovate la dimensione che vi aspettate, ho un errore o appunto c'è qualcosa che non è una ricordina è un scherzo, un gamba allora abbiamo un poco poco perché d'accordo è da via di 100-150 però iniziamo, insomma è il proprio proprio perché non lo vedeva di inizia solo allora di tutte le informazioni, quindi abbiamo un po' accennato prima ma quant'è che si fa la parcia quei analisi quando abbiamo un potenziale spericamente simmetico e l'idea è che quando abbiamo una situazione del genere, cose che si conserva il momento angolare e quindi tramite allora viene l'idea dell'analisi non departiali cioè di scrivere la situazione di urto in termini non departiali e questo è qua, scatterà in in maniera indipendente cioè i vari modi con L diverso, diciamo di diverso valore è un'altra cosa, un'interessa c'è da considerare che quando avete un onda incidente il modo quadro, cioè cosa può succedere a una funzione d'onda quando c'è uno scatto si deve conservare la probabilità chiaramente quindi il modo quadro può cambiare e quindi cosa può cambiare? la fase quindi l'effetto di uno scatto è l'interessante di quello quello creare un space sheet modificare la fase dell'onda se il potenziale è piccolo questo space sheet è molto piccolo che è più grande potenziale che un effetto di questo fase quindi questo è l'ingrediente di quando usate le onde partiziali e anche perché è utile lavorare con il space sheet sia per questo scorso che per un chiaro significato fisico cioè il potenziale che cosa mi crea, dato che c'è la possibilità della probabilità il mio se può fare è creare un space sheet l'altro motivo è che è semplifico il punto di vista matematico perché la sezione 2, la pizza di scattering è un numero complesso quindi ho due numeri invece il space sheet è un numero reale, non solo numero quindi anche il punto di vista matematico è conveniento allora la pizza di scattering quando posso usare questo superlone depaziale, se mi fai il questo modo quindi f di theta, uguale uno sul k e e e sommatoria viene uguale a zero da l uguale a zero a 5 di 2l più 1 e i del tl questo del tl è il phase sheet che parliamo di ordine l come se non fosse sempre tl del phase sheet era tl di cosa intera questo sono i polinomini legandoli per per vesimo in termini dei phase sheets la sezione è tutta si può scrivere come 4t grado k quadro sommatoria dell'uk al zero infinito di 2l più 1 secquato di del tl poi un'altra espressione che serve in alcuni eserciti in alcuni casi è quella che viene chiamata espressione integrale in formato integrale del phase sheet che sarebbe e i del tl se erano i del tl uguale a meno 2m4 per lato quadro per k per integrale del zero a infinito in r in r che mantivita g l ma r di r u l di k r dove?
   
   queste sono le pensioni isteriche di test quindi elesima spherica test e pancia che ha una forma tipo un seno e questa cosa la si distrutta a volte in cui eserciti e questi qua invece sono le soluzioni delle qualsioni radiale perché sapete che quando abbiamo un qualsione estericamente simmetico possiamo separare parte radiale, parte radiale, armonica sferica e abbiamo la soluzione della parte radiale delle qualsioni. Questa sono alcune cose per il formule principale che useremo poi se mi dimentico qualcuna una mano che servira ah sì, è una cosa interessante che spesso, oltre a poter applicare la facciola che è una analisi per quanto potenziale è esplicamente simmetica, posso in più anche applicare la borna prossimation perché sono nella cruzione di borna prossimation e quindi quindi piccolo rispetto all'energia della particella. Allora in questo caso cosa succede? che che è cifto uguale, perché avete detto potenziale è troppo basso, cosa succede? è dei shift passi piccoli e quindi questo è cifto uguale a DL anche questo per esempio il DL, il sigma del DL è cifto uguale al DL e quindi tutti questi casi chiaramente la stessa cosa lì e quindi si semplifionano queste parole. Quindi quando vedete che avete nel caso che potete applicare con una prossimation potete fare con l'interio esemplificazione. Quindi nel caso di potenziale piccolo potete, non fate caso perché porti all'utilizzo di tutte e due le strategie. Vabbè, l'esercizio deve andare. Abbiamo detto che questo vuole il generale l'ultima formula, però quando si avvicina. La formula integrale è generale. Se il dettaglio è piccolo, allora questo c'è uguale a DL e anche il seracio è uguale a DL. Il dettaglio è uguale a a un po' di massimaggio. Dicono che con l'interio esemplificazione si rientra alle esemplificazioni con una prossimation perché quando ha un potenziale d'erbole, allora in quel caso l'onne incidente risente poco di un'utiliziale e quindi, scusate, chi verrà, è pronto, scatta e rinvita. In questo questo questo questo d'erbole è è a meno di 0 per r in colre r minor di a e in zero per r maggiore di a. Quindi, a. Sampulet vesigma neomega la depresa d'impostazione in perste orte borne a prossimation alla function oltre, descatti il rinvelo tito e la inserenza è l'energy che è una una che cosa sono, bisogna rischiarli in questo questo V trasforma il passaggio per l'energia in deplo energia c diventa il zero uguale entro sei magre per fermi quindi vedete che in questo caso non abbiamo un gel per l'energia per l'energia e app il quattro fermi è il riso il riso è un value in m e d oltre, proton proton quando cetta un uguale il greca 10 E' questo, quindi controllate cosa avete dubbio, non mi risulta la posizione di la siota, invece volevo fare appunto poco tempo il secondo punto, per il secondo punto provo quello sulla fascia quelli ad analisi. Chi è di gasolare? No, credo che la prima trapassione, non è uno coseno di cosa. Di tutto questo? Ok. Sì, ovviamente questa è in termini di Q. Però scopri che è, come volevo fare, che è stato il tempo, volevo dire che l'avete tempo a casa doveva fare. Quindi, in realtà è qualcosa in cui c'è il Q, appunto da questa. Ok, ora in questo caso noi dobbiamo vedere il free shift, possiamo utilizzare questa formula e dice, come è appunto a, parlato di corner post-merch, possiamo poter continuare a fare, quindi consideriamo questo, però c'è una altra cosa da considerare, che quando lo trovo al tempo che lo ha fatto, che è quando siamo nella corner post-merch, questa U risulta essere, quindi è stata la corner post-merch, questa U è circa uguale a R per JL. E quindi questa formula diventa, se è uguale a meno di 2 Mk HdA4 integrale da 0 in P in R, in R quadro di R JL diventa KR. Poi, se in più sono nel low energy limit, come in questo caso, in the low energy limit, poche questa JL di KR di questo male è nella low energy. Low energy da JL di KR si può scrivere come kR elevato L, di 2L più 1, poche un attore di KR, in R in di KR, in R quadro di KR, in R quadro di KR, in R in R in in rucco.. pari di porto, se non ricordate cos'è il coppio potteriale potteriale uguale a 1, n è quella zero a, n per n minus 1, eccetera, 3, 3, 1, il caso in cui n è dispari e a n a n meno 2 e c3, c'è da 2 e qua c'è quattro 2, c'è n e qua è 2k per pari non riesco a dirci di coppere tutti sono tre termini pari proprio tutti sono tre termini noi chiaramente qua c'è per chi non deve venire a quindi, non sapete ci sono state cosa e sostituisco l'espressione potenziale insomma, sostituiamo l'espressione potenziale questo diventa 2mk2n1 xverbi 0 scusi, si la parte della piada è la piada allora, eh no lo cancellavo? no, per l'angelo per scusi la panna potenziale si può scrivere come r per gl e quindi ho tempo qua un r4 e qua un g4 poi sostituisco g4 questo quattro e l'altro direte il 2, per farciare il costantivo dei maghi per far ritronare quelli esponenti e tutto l'1
   
   c'è chiaramente un rl che mi viene da questo il 4 diventa 2l e qui dovrei avere più di 2 non c'è un menù che viene da uno sr ah, del 4 non c'è passato tutto il sr grazie e qua appunto con quello uno sr diventa 2l più, perfetto e quindi questo che dovrebbe essere 2mv0 h4 app k verrà 2l più 2l più mi sembra un mio cosi 2l più si è sempre così l'estremo supero l'indegra è l'infino, giusto? iniziale l'indegra è l'infino, però poi diventa no, perché questo è l'indegra io sto giusto, mi spassino meglio chiedere le passi così il menù davanti il termo, il termo, il termo, il termo, il termo il più e il potenziale allora per il termo ah, il coniziale è che ti occhia perfetto ora quello che chiedeva era che la colore era l'eternal per l'1 va bene, ci raffirmo anche per l'1 a 0 da 0 perché l'1 a 0 risulta del stano a 0 uguale m di 0 anche a tegliale 4 a per k invece per l uguale a 1 non tengo detto 1 uguale m di 0 a 4 a 18 per k a alla terza lo scritto sì, perché è una cosa che ricorre in questo caso, come l'avete sentito, è più tutto compliteramica se voi vedevate i monti esercizi, lo tenete che quando avete l'uguale 0 avete sempre questo per qualcos'altro, quando avete l'uguale 1 avete sempre questo quindi queste cose a cercate di dicem, di notarle se non le trovate, che vedete perché va bene, questo chiedeva, non dava numeri, quindi all'inizio abbiamo trovato le espressioni che chiedeva che era quelle le colore 1 per il gloo energi-input poi invece ci dà le valore specifici e ci chiede di calcolare l'energia del protone quando il delta 1, quindi questo qua abbiamo calcolato la valore specifica che è il grigatello il delta 1 uguale m, di 0 18, a, tende a quadro alla 4, a, perquale è il grigatello e questo attengo poi do' trovare l'energia cioè qui l'unico incognita è k con quella valore che richiede e poi è una k più entero con l'energia quindi più che tutto viene naturale, cercando i k al cubo k al cubo è un po' a, di 8 di greco a, da l'alto c al quadrato 10 mx al quadrato di 0 a, da l'alto di suzza ci ha situato i valori che da parecchi di 0, che si tracchia a un cuore di due termini di suzza il greco 8 mx per termini allamente 10 tutto la lenge del potone che diciamo di valore quale, eh? quale, quadro quadro, 2m nello scrivo poi hc al quadrato k al cubo e le vado a 2 tex lo faccio se mi chiedete poter utilizzarmi l'espressione 2m c4 e in questo punto, ottengo 4 10 a l'alto 1 m quadro, 3 di quadro 3 5 4 5 8 9 10 e l'innovità di misura sono i fenghi alla meno 3 quindi rispettacqua fermi alla 3 anche questo è levato alla 2 tex e finiamo allora questo questo è il punto 2 questo diventa un fermitualadro che a questo punto si cancella con questo e il mezzo e il quadro si cancella con questo e quindi rimane solo mezzo qua il 3 e il 3, la pieca finirà in un 4 che a questo punto si può sentirci con questo ah, non è visto, la pieca rimane solo in 2 e poi l'altra cosa ah, si è la torrenza, quattro c'è la tre, la mia torrenza quindi risulta 5 ah, si mette 10 in 2, quindi 5 e l'inveglare con la 2 tex oppero 5 uguale a 10.7 siamo un attimo veloci però diciamo era semplice in realtà una volta è piuttosto che le formi di pannaxia vissate un pannaxia vissate da utilizzare e quelle da utilizzare è cosa specifica di bornappassimation poi è una rossione di alcun punto di marcio come sei complicato al punto di vista concettuale ma le stesse, sì cioè, sono soltieri io sa che sono soltieri di cipro è
\end{soluzione}

\setcounter{equation}{0}