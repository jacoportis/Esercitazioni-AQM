\begin{esercizio}
   An electron beam scatters on an electron cloud described by the wave function
   \begin{equation*}
      \psi(r)
      =\frac{2}{\sqrt{4 \pi}} \frac{e^{-r/a}}{a^{3/2}}
   \end{equation*}
   Given that the Coulomb potential is given by
   \begin{equation*}
      V(\vb{r})
      =\int \dd[3]{\vb{r}'} \frac{e^2}{| \vb{r} - \vb{r}' |} \rho_e(\vb{r}')
   \end{equation*}
   where $\rho_e(\vb{r})$ is the electron density of the cloud.
   \begin{enumerate}[label=\alph*), leftmargin=0.6cm]
      \item Calculate in Born approximation the differential cross-section (at the end write it as a Rutherford cross-section for a function $F(q^2)$).
      \item What do you get for $qa \to 0$? Explain from a physical point of view.
      \item If at $\vartheta=\frac{\pi}{3}$ the differential cross-section decreases by a factor $3^4/4$ when the beam energy increases form 20 eV to 40 eV, calculate the value of $a$ in \A.
   \end{enumerate}
   Hint 1: with the substitution $\vb{y}=\vb{r} - \vb{r}'$, is given that
   \begin{equation*}
      \int \dd[3]{\vb{y}} \frac{e^{i \vb{q} \vdot \vb{y}}}{|\vb{y}|}
      =\frac{4 \pi}{q^2}
   \end{equation*}
   Hint 2:
   \begin{equation*}
      \int_{0}^{+\infty} \dd{r} r e^{-2r/a} \sin(qr)
      =\frac{4qa^3}{(4 + q^2a^2)^2}
   \end{equation*}
\end{esercizio}
\begin{soluzione}
   Svolgiamo il punto a). Dobbiamo scrivere la sezione d'urto differenziale in approssimazione di Born come il prodotto della sezione d'urto di Rutherford per una certa funzione $F(q^2)$, cioè dobbiamo giungere alla relazione
   \begin{equation*}
      \qty( \dv{\sigma}{\Omega} )_{\rm Born}
      =\qty( \dv{\sigma}{\Omega} )_{\rm Ruth} F(q^2)
   \end{equation*}
   Sappiamo che la sezione d'urto differenziale è data da
   \begin{equation*}
      \dv{\sigma}{\Omega}
      =|f^{\rm Born}|^2
   \end{equation*}
   dove $f^{\rm Born}$ è l'ampiezza di scattering in approssimazione di Born, la quale è data da
   \begin{equation*}
      f_k
      =-\frac{m}{2 \pi \hbar^2} \int \dd[3]{\vb{r}} e^{i \vb{q} \vdot \vb{r}} V(\vb{r})
   \end{equation*}
   dove $\vb{q}=\vb{k} - \vb{k}'$. Sostituiamo quindi l'espressione del potenziale:
   \begin{equation*}
      f_k
      =-\frac{m}{2 \pi \hbar^2} \int \dd[3]{\vb{r}} e^{i \vb{q} \vdot \vb{r}} \int \dd[3]{\vb{r}'} \frac{e^2}{| \vb{r} - \vb{r}' |} \rho_e(\vb{r}')
   \end{equation*}
   A questo punto sfruttiamo il primo suggerimento: usando la sostituzione $\vb{y}=\vb{r} - \vb{r}'$, da cui $\vb{r}=\vb{y} + \vb{r}'$, otteniamo
   \begin{equation*}
      f_k
      =-\frac{m e^2}{2 \pi \hbar^2} \int \dd[3]{\vb{r}'} e^{i \vb{q} \vdot \vb{r}'} \rho_e(\vb{r}') \int \dd[3]{\vb{y}} \frac{e^{i \vb{q} \vdot \vb{y}}}{| \vb{y} |}
      =-\frac{m e^2}{2 \pi \hbar^2} \frac{4 \pi}{q^2} \int \dd[3]{\vb{r}'} e^{i \vb{q} \vdot \vb{r}'} \rho_e(\vb{r}')
   \end{equation*}
   avendo utilizzato il risultato fornito dal primo suggerimento.\\
   Calcoliamo ora l'altro integrale. Per fare ciò notiamo che $\rho_e(\vb{r}')$ è data da $|\psi|^2$, per cui sostituiamo il modulo quadro dell'espressione della funzione d'onda:
   \begin{equation*}
      f_k
      =-\frac{2 m e^2}{\hbar^2 q^2} \int \dd[3]{\vb{r}'} e^{i \vb{q} \vdot \vb{r}'} |\psi(\vb{r}')|^2
      =-\frac{2 m e^2}{\hbar^2 q^2} \int \dd[3]{\vb{r}'} e^{i \vb{q} \vdot \vb{r}'} \frac{4}{4 \pi} \frac{e^{-2 r'/a}}{a^3}
   \end{equation*}
   Passando ora in coordinate sferiche abbiamo
   \begin{equation*}
      f_k
      =-\frac{2 m e^2}{\pi \hbar^2 q^2 a^3} \int_{0}^{+\infty} \dd{r'} {r'}^2  e^{-2 r'/a} \, 2\pi \int_{-1}^{1} \dd{(\cos{\vartheta})} e^{i q r' \cos{\vartheta}}
   \end{equation*}
   Osserviamo adesso che
   \begin{equation*}
      \int_{-1}^{1} \dd{(\cos{\vartheta})} e^{i q r' \cos{\vartheta}}
      =\frac{e^{i q r'} - e^{-i q r'}}{i q r'}
      =2 \frac{\sin(q r')}{q r'}
   \end{equation*}
   per cui inserendo tale risultato nell'espressione di sopra otteniamo
   \begin{equation*}
      f_k
      =-\frac{4 m e^2}{\hbar^2 q^3 a^3} \int_{0}^{+\infty} \dd{r'} r' e^{-2 r'/a} \sin(q r')
      =-\frac{4 m e^2}{\hbar^2 q^3 a^3} \frac{4 q a^3}{(4 + q^2 a^2)^2}
      =-\frac{2m}{\hbar} \frac{e^2}{q^2} \frac{16}{(4 + q^2 a^2)^2}
   \end{equation*}
   avendo utilizzato il risultato fornito dal secondo suggerimento.\\
   In definitiva, la sezione d'urto differenziale in approssimazione di Born sarà data da
   \begin{equation}
      \dv{\sigma}{\Omega}
      =|f_k|^2
      =\frac{4m^2}{\hbar^4} \frac{e^4}{q^4} \frac{256}{(4 + q^2 a^2)^4}
      \label{eq:sezione_durto_differenziale_nube_elettronica}
   \end{equation}
   Il testo chiede però di calcolare la sezione d'urto in termini della sezione d'urto di Rutherford. Ricordiamo che quest'ultima si ottiene quando abbiamo scattering di una particella carica con un potenziale coulombiano puntiforme (cioè con range del potenziale nullo), mentre nel problema in esame abbiamo una nuvola elettronica avente una certa estensione data dal range del potenziale. Appare allora chiaro che il fattore $F(q^2)$ che dobbiamo ottenere altro non è che il fattore di forma, il quale ci dà informazioni sulle caratteristiche del potenziale come forma ed estensione.\\
   Ricordiamo che la sezione d'urto di Rutherford è data da
   \begin{equation*}
      \qty( \dv{\sigma}{\Omega} )_{\rm Ruth}
      =\frac{1}{16} \qty( \frac{Z_1 e Z_2 e}{E} )^2 \frac{1}{\sin^4(\vartheta/2)}
   \end{equation*}
   dove $Z_1$ e $Z_2$ sono la carica della particella incidente e della particella su cui scatteriamo. In questo caso stiamo considerando un elettrone che collide con una nuvola elettronica, per cui $Z_1=Z_2=1$. Inoltre $E=\frac{\hbar^2 k^2}{2m}$, dunque possiamo riscrivere
   \begin{equation*}
      \qty( \dv{\sigma}{\Omega} )_{\rm Ruth}
      =\frac{1}{16} \frac{e^4 4 m^2}{\hbar^4 k^4} \frac{1}{\sin^4(\vartheta/2)}
   \end{equation*}
   Per ricondurci a tale forma, osserviamo che sussiste la relazione
   \begin{equation}
      q=2k \sin(\vartheta/2)
      \label{eq:relazione_q_k}
   \end{equation}
   Per dimostrarla, partiamo dalla definizione di $\vb{q}$ ed eleviamola al quadrato:
   \begin{equation*}
      \vb{q}
      =\vb{k} - \vb{k}'
      \implies
      q^2
      =(\vb{k} - \vb{k}')^2
      =k^2 + k'^2 - 2\vb{k} \vdot \vb{k}'
   \end{equation*}
   Poiché ci troviamo nel caso di scattering elastico abbiamo che $k=k'$, inoltre tra i due vettori d'onda c'è un angolo $\vartheta$, per cui possiamo scrivere
   \begin{equation*}
      q^2
      =2k^2 - 2 k^2 \cos{\vartheta}
      =2k^2 ( 1 - \cos{\vartheta} )
   \end{equation*}
   e poiché $(1 - \cos{\vartheta})=2 \sin^2(\vartheta/2)$, in definitiva abbiamo
   \begin{equation*}
      q^2
      =4k^2 \sin^2(\vartheta/2)
   \end{equation*}
   da cui la relazione di sopra prendendo la radice di ambo i membri.\\
   Inserendo allora la \eqref{eq:relazione_q_k} nella \eqref{eq:sezione_durto_differenziale_nube_elettronica} otteniamo
   \begin{equation*}
      \frac{4m^2}{\hbar^4} \frac{e^4}{q^4} \frac{256}{(4 + q^2 a^2)^4}
      =\frac{1}{16} \frac{4m^2}{\hbar^4} \frac{e^4}{\sin^4(\vartheta/2)} \frac{256}{(4 + q^2 a^2)^4}
      \equiv \qty( \dv{\sigma}{\Omega} )_{\rm Ruth} \frac{256}{(4 + q^2 a^2)^4}
   \end{equation*}
   da cui segue che il fattore di forma è
   \begin{equation*}
      F(q^2)
      =\frac{256}{(4 + q^2 a^2)^4}
   \end{equation*}
   Passiamo al quesito b). Osserviamo per $qa \to 0$ la sezione d'urto si riduce a quella di Rutherford in quanto $F(q^2) \to 1$. Dal punto di vista fisico, ciò significa che il fascio incidente vede la nube come puntiforme, e ciò accade perché il fascio risulta avere lunghezza d'onda molto grande rispetto ad $a$, che è una lunghezza legata alla dimensione della nube e dunque alla portata del potenziale. Infatti, la condizione $qa \to 0$ può essere interpretata anche come $a \ll q^{-1}$, ma $q^{-1}$ è a sua volta legato alla lunghezza d'onda di De Broglie del fascio incidente.\\
   Passiamo infine al punto c). Considerato un angolo specifico di scattering $\vartheta=\frac{\pi}{3}$, si ha che quando l'energia aumenta da 20 eV a 40 eV la sezione d'urto differenziale diminuisce di un fattore $3^4/4$. Sapendo ciò, dobbiamo determinare il valore di $a$ associato alla nube elettronica.\\
   Osserviamo innanzitutto che per $\vartheta=\frac{\pi}{3}$ dalla \eqref{eq:relazione_q_k} si ha
   \begin{equation*}
      q^2
      =4 k^2 \sin^2{ \qty( \frac{\pi}{6} ) }
      =4 k^2 \qty( \frac{1}{2} )^2
      =k^2
   \end{equation*}
   da cui segue che l'energia può essere riscritta come
   \begin{equation*}
      E
      =\frac{\hbar^2 k^2}{2m}
      =\frac{\hbar^2 q^2}{2m}
   \end{equation*}
   Data quindi tale relazione tra l'energia e $q^2$, aumentare l'energia di un fattore 2 equivale ad aumentare $q$ di un fattore $\sqrt{2}$. Più esplicitamente, se $E=20 \rm \; eV$ ed $E'=40 \rm \; eV$, dunque $E'=2E$, per i corrispondenti valori dell'impulso trasferito, $q$ e $q'$, vale la relazione $q'=\sqrt{2}q$. Per quanto riguarda le sezioni d'urto, abbiamo che
   \begin{equation*}
      \qty( \dv{\sigma}{\Omega} )_{E'}
      =\frac{4}{3^4} \qty( \dv{\sigma}{\Omega} )_{E}
   \end{equation*}
   che utilizzando la \eqref{eq:sezione_durto_differenziale_nube_elettronica} diventa
   \begin{equation*}
      \frac{1}{{q'}^4} \frac{1}{(4 + {q'}^2 a^2)^4}
      =\frac{4}{3^4} \frac{1}{q^4} \frac{1}{(4 + q^2 a^2)^4}
   \end{equation*}
   Facendo la radice quarta di ambo i membri otteniamo
   \begin{equation*}
      \frac{1}{q'} \frac{1}{4 + {q'}^2 a^2}
      =\frac{\sqrt{2}}{3} \frac{1}{q} \frac{1}{4 + q^2 a^2}
   \end{equation*}
   che, inserendo la relazione tra $q$ e $q'$, diventa
   \begin{equation*}
      \frac{1}{\sqrt{2}q} \frac{1}{4 + 2q^2 a^2}
      =\frac{\sqrt{2}}{3} \frac{1}{q} \frac{1}{4 + q^2 a^2}
      \implies
      4 + 2q^2 a^2
      =6 + \frac{3}{2} q^2 a^2
   \end{equation*}
   da cui segue che
   \begin{equation*}
      a
      =\frac{2}{q}
   \end{equation*}
   Per calcolare tale quantità, osserviamo che per quanto visto in precedenza si ha
   \begin{equation*}
      q^2
      =\frac{2mE}{\hbar^2}
      =\frac{2mc^2 E}{(\hbar c)^2}
      =\rm \frac{2 \cdot 0.5 \cdot 10^6 \; eV \cdot 20 \; eV}{4 \cdot 10^4 \; eV^2 \, nm^2}
      =\rm 5 \cdot 10^2 \; nm^{-2}
   \end{equation*}
   e quindi
   \begin{equation*}
      a
      =\rm \frac{2}{\sqrt{500}} \; nm
      \approx 0.09 \rm \; nm
      =0.9 \; \text{\A}
   \end{equation*}
\end{soluzione}

\newpage
\setcounter{equation}{0}

\begin{esercizio}
   A proton moving along the $z$ axis scatters on the potential
   \begin{equation*}
      V(\vb{r})=
      \begin{cases}
         V_0 \delta(z) & \text{for } x^2 + y^2 < R_T^2\\
         0 & \text{elsewhere}
      \end{cases}
   \end{equation*}
   \begin{enumerate}[label=\alph*), leftmargin=0.6cm]
      \item Find the scattering amplitude $f(\vb{k},\vb{k}')$.\footnotemark
      \item Consider the low-energy limit and calculate the total cross-section.
   \end{enumerate}
\end{esercizio}
\begin{soluzione}
   \footnotetext{In realta non troveremo l'ampiezza di scattering in funzione di $\vb{k}$ e $\vb{k'}$ in quanto non ci serve per svolgere il punto b).}
   Osserviamo innanzitutto che il potenziale ha simmetria cilindrica, in quanto esso è non nullo solo su un cerchio di raggio $R_T$ che giace sul piano $xy$ ed è dunque quella che dobbiamo adoperare nell'approssimazione di Born. In particolare, passando dalla coordinate cartesiane $(x,y,z)$ alle coordinate cilindriche $(r,\varphi,z)$, il potenziale diventa
   \begin{equation*}
      V(\vb{r})=
      \begin{cases}
         V_0 \delta(z) & \text{for } r < R_T\\
         0 & \text{altrimenti}
      \end{cases}
   \end{equation*}
   avendo utilizzato le leggi di trasformazione $(x,y,z) \mapsto (r\cos{\varphi},r\sin{\varphi},z)$. Sempre mediante tali trasformazioni, possiamo scrivere il prodotto $\vb{q} \vdot \vb{x}$ come
   \begin{equation*}
      \vb{q} \vdot \vb{r}
      =q_x x + q_y y + q_z z
      =q_x r \cos{\varphi} + q_y r \sin{\varphi} + q_z z
   \end{equation*}
   Se adesso poniamo $q_x=q_T \cos{\varphi}$ e $q_y=q_T \sin{\varphi}$ dove $q_T$ è la proiezione di $\vb{q}$ sul piano $xy$, abbiamo
   \begin{equation*}
      \begin{split}
         \vb{q} \vdot \vb{r}
         & =q_T r \cos{\varphi} \cos{\varphi} + q_T r \sin{\varphi} \sin{\varphi} + q_z z
         \\
         & =q_T r \cos^2{\varphi} + q_T r \sin^2{\varphi} + q_z z
         \\
         & =q_T r + q_z z
      \end{split}
   \end{equation*}
   In definitiva l'ampiezza di scattering sarà data da
   \begin{equation*}
      f_k^{\rm Born}
      =-\frac{m}{2 \pi \hbar^2} \int \dd[3]{\vb{r}} e^{i \vb{q} \vdot \vb{r}} V(\vb{r})
      =-\frac{m}{2 \pi \hbar^2} \int_{0}^{2\pi} \dd{\varphi} \int_{0}^{+\infty} \dd{r} \int_{-\infty}^{+\infty} \dd{z} e^{i (q_T r + q_z z)} V(r,\varphi,z)
   \end{equation*}
   \begin{equation*}
      f_k^{\rm Born}
      =-\frac{m V_0}{2 \pi \hbar^2} \int_{0}^{2\pi} \dd{\varphi} \int_{0}^{R_T} \dd{r} e^{i q_T r} \int_{-\infty}^{+\infty} \dd{z} e^{i q_z z} \delta(z)
   \end{equation*}
   Osserviamo che l'integrazione rispetto a $\varphi$ dà un fattore $2\pi$, mentre l'integrale in $z$ è pari a 1.\footnote{Ricordiamo che in generale si ha
   \begin{equation*}
      \int_{-\infty}^{+\infty} \dd{x} f(x) \delta(x - x_0)=f(x_0)
   \end{equation*}} In definitiva otteniamo
   \begin{equation*}
      f_k^{\rm Born}
      =-\frac{m V_0}{\hbar^2} \int_{0}^{R_T} \dd{r} r e^{i q_T r}
      =\frac{m V_0}{\hbar^2} \qty( \frac{i R_T e^{i q_T R_T}}{q_T} - \frac{e^{i q_T R_T} - 1}{q_T^2})
   \end{equation*}
   Passiamo al quesito b). Ricordiamo che il limite di basse energie corrisponde alla condizione $k R_V \ll 1$ dove $R_V$ è il range del potenziale, che nel problema in esame è pari a $R_T$. Inoltre nelle espressioni che abbiamo trovato $k$ è legato a $q_T$, dunque in questo caso la condizione diventa $q_T R_T \ll 1$. Notiamo che tale quantità figura ad esponenziale, per cui possiamo sviluppare quest'ultimo in serie. Bisogna però stare attenti a quale ordine bisogna troncare lo sviluppo in serie: per capirlo osserviamo che al denominatore abbiamo un termine $q_T^2$, per cui dobbiamo approssimare fino al secondo ordine. In definitiva, ricordando che
   \begin{equation*}
      e^{i q_T R_T}
      \simeq 1 + i q_T R_T - \frac{q_T^2 R_T^2}{2}
   \end{equation*}
   avremo
   \begin{equation*}
      f^{\rm Born}
      \simeq \frac{m V_0}{\hbar^2} \qty( i \frac{R_T}{q_T} - R_T^2 - \frac{i}{2} q_T R_T^3 - i \frac{R_T}{q_T} + \frac{1}{2} R_T^2)
      =-\frac{m V_0}{\hbar^2} \qty( \frac{1}{2} R_T^2 + \frac{i}{2} q_T R_T^3 )
   \end{equation*}
   Troviamo quindi che il primo ordine non nullo è quello in $R_T^2$. Ne segue, visto che siamo nella condizione $R_T \ll 1/q_t$, che possiamo trascurare i termini di ordine superiore. Pertanto possiamo scrivere
   \begin{equation*}
      f^{\rm Born}
      \simeq \frac{m V_0}{\hbar^2} \qty( i \frac{R_T}{q_T} - R_T^2 - \frac{i}{2} q_T R_T^3 - i \frac{R_T}{q_T} + \frac{1}{2} R_T^2)
      =-\frac{m V_0}{2 \hbar^2} R_T^2
   \end{equation*}
   Osserviamo che tale ampiezza di scattering non dipende né da $k$ (dunque dall'energia), né dall'angolo $\vartheta$ di scattering. Ricordiamo che l'ampiezza di scattering ci dice qual è l'effetto dello scattering al variare dell'angolo $\vartheta$, per cui un risultato come questo ci dice che non è importante a quale angolo $\vartheta$ viene scatterata la particella, in quanto l'ampiezza sarà la stessa. Inoltre il fatto che l'ampiezza non dipenda neanche dall'energia (rimanendo sempre nel limite di basse energie) ci dice che piccole variazioni dell'energia non hanno alcun effetto nel risultato.\\
   Calcoliamo ora la sezione d'urto totale. Per fare ciò dobbiamo prima trovare la sezione d'urto differenziale, la quale è data da
   \begin{equation*}
      \dv{\sigma}{\Omega}
      =|f^{\rm Born}|^2
      =\frac{m^2 V_0^2 R_T^4}{4 \hbar^4}
   \end{equation*}
   La sezione d'urto totale sarà quindi data da
   \begin{equation*}
      \sigma
      =\int \dd{\Omega} \dv{\sigma}{\Omega}
      =\int \dd{\Omega} \frac{m^2 V_0^2 R_T^4}{4 \hbar^4}
      =\frac{m^2 V_0^2 R_T^4}{4 \hbar^4} 4\pi
      =\frac{\pi m^2 V_0^2 R_T^4}{\hbar^4}
   \end{equation*}
   in quanto l'integrale sull'angolo solido dà semplicemente un fattore $4\pi$ dato che come già notato l'espressione trovata non dipende dall'angolo.\\
   Concludiamo con un commento sull'analisi dimensionale. La sezione d'urto ha le dimensioni di un'area e spesso viene misurata in barn (b), i quali sono tali che $\rm 1 \; b=100 \; fm^2$; pertanto se esplicitassimo i valori delle quantità presenti nell'espressione di sopra dovremmo trovare tale quantità. Per trovare il risultato corretto bisogna notare una cosa: $V_0$ non ha le dimensioni di un'energia, bensì quelle di un'energia moltiplicata per una lunghezza. Il motivo è che nell'espressione del potenziale esso è accoppiato a $\delta(z)$, la quale ha le dimensioni dell'inverso di una lunghezza, dunque per ottenere dimensionalmente un'energia per $V(r)$ deve valere quanto detto.
\end{soluzione}

\newpage
\setcounter{equation}{0}

\begin{esercizio}
   Given a proton scattering on the potential
   \begin{equation*}
      V(r)=
      \begin{dcases}
         -\frac{V_0}{r} & \text{for } r<a\\
         0 & \text{for } r>a
      \end{dcases}
   \end{equation*}
   \begin{enumerate}[label=\alph*), leftmargin=0.6cm]
      \item Calculate the differential cross section $\dv*{\sigma}{\Omega}$ in $1^{\rm st}$-order Born approximation as a function of the scattering angle $\vartheta$ and the scattering energy $E$.
      \item Calculate the phase shift for $\ell=0,1$ in the low-energy limit.
      \item Given $V_0=36 \; \rm MeV \, fm$ and $a=2 \; \rm fm$, what is the value in MeV of the proton energy to have $\delta_1=\pi/10$?
   \end{enumerate}
\end{esercizio}
\begin{soluzione}
   Svolgiamo il punto a). Osserviamo che il potenziale è di tipo centrale, dunque per calcolare per calcolare l'ampiezza di scattering possiamo usare l'espressione
   \begin{equation*}
      f^{\rm Born}
      =-\frac{2m}{\hbar^2} \int_{0}^{\infty} \dd{r} r V(r) \frac{\sin{(qr)}}{q}
   \end{equation*}
   Inseriamo l'espressione del potenziale: poiché il potenziale è nullo per $r>a$, si ha
   \begin{equation*}
      f^{\rm Born}
      =-\frac{2m}{\hbar^2} \int_{0}^{a} \dd{r} r \qty( -\frac{V_0}{r} ) \frac{\sin{(qr)}}{q}
      =\frac{2m V_0}{q \hbar^2} \int_{0}^{a} \dd{r} \sin{(qr)}
      =\frac{2m V_0}{q^2 \hbar^2} \bigl[ 1 - \cos{(qa)} \bigr]
   \end{equation*}
   Riscriviamo tale espressione in termini di $\vartheta$ ed $E$. Ricordiamo innanzitutto che vale la relazione
   \begin{equation*}
      q
      =2k \sin{(\vartheta/2)}
   \end{equation*}
   che inserita nell'espressione di sopra dà
   \begin{equation*}
      f^{\rm Born}
      =\frac{2m V_0}{4 \hbar^2 k^2 \sin^2{(\vartheta/2)}} \Bigl\{ 1 - \cos{\bigl[ 2ka \sin{(\vartheta/2)} \bigr] } \Bigr\}
   \end{equation*}
   Ricordiamo che adesso che $k$ è legata all'energia mediante la relazione
   \begin{equation*}
      E
      =\frac{\hbar^2 k^2}{2m}
   \end{equation*}
   e quindi possiamo riscrivere ulteriormente l'ampiezza di scattering come
   \begin{equation*}
      f^{\rm Born}(\vartheta,E)
      =\frac{V_0}{4E} \qty{ \frac{1 - \cos{\qty[ 2 \frac{\sqrt{2mE}}{\hbar} a \sin{(\vartheta/2)} ] }}{\sin^2{(\vartheta/2)}} }
   \end{equation*}
   La sezione d'urto differenziale sarà quindi data da
   \begin{equation*}
      \dv{\sigma}{\Omega}
      =|f^{\rm Born}(\vartheta,E)|^2
      =\qty( \frac{V_0}{4E} )^2 \qty{ \frac{1 - \cos{\qty[ 2 \frac{\sqrt{2mE}}{\hbar} a \sin{(\vartheta/2)} ] }}{\sin^2{(\vartheta/2)}} }^2
   \end{equation*}
   Passiamo al punto b). Supponendo ancora di essere nell'approssimazione di Born, il phase shift sarà dato da
   \begin{equation*}
      \delta_{\ell}
      =-\frac{2mk}{\hbar^2} \int_{0}^{+\infty} \dd{r} r^2 V(r) j_{\ell}(kr)^2
   \end{equation*}
   Poiché inoltre ci troviamo nel limite di basse energie, le funzioni di Bessel seguono l'andamento
   \begin{equation*}
      j_{\ell}(kr)
      \simeq \frac{(kr)^{\ell}}{(2\ell+1)!!}
   \end{equation*}
   dove il semi-fattoriale è definito come
   \begin{equation*}
      n!!=
      \begin{cases}
         1 & \text{se } n=0,1\\
         n \cdot (n - 2) \cdot \ldots \cdot 3 \cdot 1 & \text{se } n=2k + 1 \qq{,} k \in \mathbb{N}\\
         n \cdot (n - 2) \cdot \ldots \cdot 4 \cdot 2 & \text{se } n=2k \qq{,} k \in \mathbb{N}\\
      \end{cases}
   \end{equation*}
   Possiamo quindi scriveremo il phase shift come
   \begin{equation*}
      \delta_{\ell}
      \simeq - \frac{2 m k^{(2\ell+1)}}{\bigl[ (2\ell+1)!! \bigr]^2 \hbar^2} \int_{0}^{+\infty} \dd{r} r^{2 \ell + 2} V(r)
   \end{equation*}
   Inserendo l'espressione del potenziale, otteniamo
   \begin{equation*}
      \delta_{\ell}
      \simeq \frac{2 m k^{(2\ell+1)} V_0}{\bigl[ (2\ell+1)!! \bigr]^2 \hbar^2} \int_{0}^{a} \dd{r} r^{2 \ell + 1}
      =\frac{2 m V_0}{\hbar^2} a \frac{(ka)^{(2\ell+1)}}{(2 \ell + 2) \bigl[ (2\ell+1)!! \bigr]^2}
   \end{equation*}
   In particolare, per $\ell=0$ e $\ell=1$ si ha
   \begin{equation*}
      \begin{split}
         \delta_0
         & =\frac{m V_0}{\hbar^2} a (ka)
         \\
         \delta_1
         &=\frac{m V_0}{\hbar^2} \frac{a}{18} (ka)^3
      \end{split}
   \end{equation*}
   Passiamo infine al punto c). Dobbiamo trovare l'energia del protone tale che $\delta_1=\pi/10$. Imponiamo allora tale condizione in modo da trovare un valore per $k$:
   \begin{equation*}
      \frac{m V_0}{\hbar^2} \frac{a}{18} (ka)^3
      =\frac{\pi}{10}
      \implies
      k^3
      =\frac{18 \pi \hbar^2}{10 m V_0 a^4}
   \end{equation*}
   Sostituendo esplicitamente i valori numerici otteniamo
   \begin{equation*}
      k^3
      =\frac{18 \pi (\hbar c)^2}{10 mc^2 V_0 a^4}
      =\rm \frac{18 \pi \cdot 4 \cdot 10^{4} \; MeV^2 \, fm^2}{10 \cdot 10^3 \; MeV \cdot 36 \; \rm MeV \, fm \cdot 2^4 \; fm^4}
      =\frac{\pi}{8} \; \rm fm^{-3}
   \end{equation*}
   Ricordando che la relazione tre $E$ e $k$ è data da
   \begin{equation*}
      E
      =\frac{\hbar^2 k^2}{2m}
   \end{equation*}
   il valore di $E$ sarà dato da
   \begin{equation*}
      E
      =\frac{\hbar^2 k^2}{2m}
      =\frac{\hbar^2 {(k^3)}^{\frac{2}{3}}}{2m}
      =\frac{(\hbar c)^2 {(k^3)}^{\frac{2}{3}}}{2 mc^2}
      =\rm \frac{4 \cdot 10^4 \; MeV \, fm \, \pi^{\frac{2}{3}}}{2 \cdot 10^3 \; MeV (2^3)^{\frac{2}{3}} \; fm^2}
      =5 \pi^{\frac{2}{3}} \; \rm MeV
      \approx 10.7 \; \rm MeV
   \end{equation*}
\end{soluzione}

\setcounter{equation}{0}