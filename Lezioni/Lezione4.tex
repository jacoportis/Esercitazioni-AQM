\begin{esercizio}
   An electron is under the action of the constant magnetic field $\vb{B}=B_z(0,0,1)$. At $t=0$ a magnetic field along the $\hat{x}$ direction is turned on; its strenght increases uniformly from $0$ to $B_x$ at time $t=T$ and then remains constant. The electron is initially in the spin-up state. Assume $B_x \ll B_z$.
   \begin{enumerate}[label=\alph*), leftmargin=0.6cm]
      \item Find the transition amplitude to the spin-down state at $t=T$ in the first order of perturbation theory.
      \item Calculate with a good approximation the transition probability for $B_z=5 \cdot 10^6 \rm , \; V/m$, $B_x=10^5 \rm \; V/m$ and $T=\frac{mc}{e B_x}$. Is perturbation theory valid?
      \item Calculate in eV the first nonzero correction to the energy levels for $t \gg T$
   \end{enumerate}
\end{esercizio}
\begin{soluzione}
   Allora, come stavate ora abbiamo lezionale di oggi, lezionale di Velletti, di porto 11-14, lezionale di Velletti, oggi faremo due di teoria perturbativa, toccando un po' tutto, teoria perturbativa di vento in tattempo, di vento in tattempo e buca a coppì, e poi uno di scatteri, per me lo bisogna trovare, ma al contrario, le scatteri nel momento di teoria perturbativa. Allora, iniziamo. L'elecrono è under the action Ove constant, magnetic fit, magnetic fit, v z, quindi sotto il deso lungo la stessa z, atto t qua a zero, e magnetic fit, e l'ultima, a lungo, in stessa direzione, è inizio. E' un'altra strada, e e strada, il stent increases uniformity fra un zero, tu, bx, at time, t equal di grande. And then, remains constant. punto. Deletron is inisce in the spin at state, punto. Assi un bx molto minore di b, zero. punto. A, Find the transition amplitude to the spin at state att t, t equal o uguale t grande, in the order of the passion here. punto. Min. Calculate with a good approximation. The transition probability for, e qua abbiamo la volita di canzone magnetica, v z uguale 5 per 10 alla 6 volt per 100 metri, v x 10 alla 5 volt per 100 metri. punto. And, grande uguale mc diviso bx, punto. Is perturbation theory varied, c, calculate in electron volt the first non zero correction to the energy level, forti, quanti piccoli sono più grandi t grand? t grande mc diviso e bx, t grande grande grande grande bx, t t mc diviso e bx, t grande mc mc e bx, bx, bx, mc diviso e bx, bx, bx, mc diviso e bx, t grande grande diviso e e e grande mc diviso e bx, t 
   
   e e t grande mc diviso e e e e e e e bx, t grande mc diviso diviso bx, t t t diviso e bx, bx, 1 che è accadagliato 2 mc di z che nemmamo la e con 1 ed e con 0 è uguale a meno che è accadagliato 2 mc di z per l'esplosione l'esplosione l'esplosione se questo è lo stato di più bassergero di il gross state questo è il sorcitato e c'è uno splitting e accadagliato di z di viso mc tra i due livelli energetici quindi questa è la situazione prima del tempo 2 con 0 appena arriviamo al tempo 2 con 0 viene accesso un campo elettrico che varia come varia? scrivetemi la forma è lineare no? sì quindi dovrei fare y v è lungo x quindi bx di t è uguale a bx di t su t grave in modo che inizialmente quando t uguale a 0 questo è 0 come deve essere appena t diventa t grande questo fa 1 e quindi assume il valore di t di x in realtà il test del problema lo dava però siccome mi può capitare che non lo dava preferito quindi è stato veloce per l'altro e quindi a questo punto in alto tempo t0 si accende questo campo elettrico cosa chiede? fine, transition amplitude due to the spin down state al tempo t grande in first order per torpecianti e quindi stiamo calcolando quindi in realtà questo è time dependent per torpecianti e dobbiamo calcolare l'ambizione transizione di questo campo magnetico che cambia col tempo quindi c quindi non dò se è come lo chiamavamo se era c o viva o te comunque la transition amplitude da lo stato in zero allo stato finale in questo caso è meno i accadagliato da 0 a pi grande e elevato a i omega phi di potenziale di f e di di che dipende dal tempo i dobbiamo omega f e f f è uguale e quello stato con spin down meno e con il stato di spin up perché ne partiamo da spin up e andiamo spin down non giusto si, dove vuol dire che si si qua spin up? di visto accadagliato e quindi scopola almeno è accadagliato mc di zero ora dobbiamo calcolare il elemento di questo elemento di matrice scopola il elemento di di di il elemento di matrice scopola
   
   scopola scopola scopola matrice matrice matrice elemento elemento elemento elemento p f e di i e qua questo elemento di matrice lo stresso, mi scusi ma visto che la differenza di energia è accadagliato su mc e la dividiamo per accadagliato non sarebbe scompare nella onega grazie grazie questo è uguale no invece l'ho scritto quindi a questo punto come potenziale che devo considerare è quello la con il sambagnetico rungo x quindi lo riscrivo qua sarebbe accadagliato e diviso 2 mc sigma x di piccoli e sotto i grandi e di x porto tutto fuori tranne sigma x accadagliato è 2 mc di piccoli e sotto sotto grandi ora sigma sigma x è quella che è l'elemente anticarbonale e quindi è facile ricordarsi che la sua azione è fondamentalmente di scambiale se avete app diventa down diventa up quindi lo potete fare solo niente se non lo ricordate lo calcolate e quindi risulta questo risulta questo qua e quindi attengiamo accadagliato è 2 mc diviso di grande manca però se vedete il sambagnetico questo è il risultato del vx il risultato risultato risultato risultato risultato punto insegniamo questo è la omega all'interno dell'integrale non è bella quello che è è è lunga della fraun quindi meno i su accadagliato integrale da 0 a t grande mt e elevato a meno i e di z diviso di mc di piccolo accadagliato è 2 mc diviso di piccolo piccolo di grande di x portando tutto fuori semplificando un po' attengo meno i se vi vuolosimo controllare che fattori è corretto alla semplificazione dei vari fattori quindi meno i è diviso 2 mc t grande questo è il il grande vx e poi l'integrale da 0 a t grande di elevato a meno i e su di z diviso mc t per di questo lo integriamo per i patti vi comprallate i voi tutti i passaggi e si ottiene questo integrale si può fare anche con il trick di Feynman anche se è complesso cioè riscriverlo come derivata rispetto a i e b z su mc di soltanto l'esponenziale ovviamente moltiplicando che eventuali fattori che mancano questo è un'oversaliva comunque è semplicissimo farlo per parte in questo caso fin e levri vi devi farlo per parte però forse in questo caso non è è mia volta però forse c'è sempre il problema che il tempo è finito no non ne sono sicura ma per carine possiamo parlare dopo ok quindi integrando per parti si ottiene no non non li attende di una ultima passazione un mezzo beta x su beta z su b z e b b su b z e vado a meno i e b z mc e levri levri levri levri meno i mc 2 t b su b z al quadrato e levato a meno i e veda e grande mc meno i ok finiciamo il primo punto è fatto poi ho almeno la mia z transizione e le le simboli ci non richiede di fare sostituzioni per tra zero e t grande e l'infrastor del perturbe scottile quello che sarete diciamo ormai fare è una pizza di probabilità probabilità una buona approssimazione sostituendo questi valori abbiamo già questi valori che dicono che è molto maggiore ma lo diceva ancora più chiaramente prima quindi possiamo semplificare questa qua lo vedete subito possiamo scrivere in questo modo ok dobbiamo considerare anche l'espressione t che dov'è costesce l'espressione t grande quindi sostitutiamo questa espressione ah certo perché se no sostitutiamo questa espressione e questo diventa quindi la c da e a f diventa un mezzo e a a a sub b z e elevato a meno e sub b x e elevato a a a sub b x e elevato a a a a a z al quadrato e poi e elevato a meno e sub b b su b z su b x Qua possiamo fare? Questa good approximation che chiede? Magari intanto approssimo le esponenze alla 1, mi vado a dire? O forse lo sviluppiamo in serie e più approssimo alla 1? Sì, potremmo fare questo. Oppure il fatto che abbiamo un termine al primo ordino e al secondo ordino in questo rapporto? Quindi ovvero trascuriamo il secondo termine?
   
   Perfetto, possiamo già trascurare tutto questo. Perché dato che bx diviso bz molto minore di 1, non te consideriamo quella più morbida, il secondo il deserso è meno che che che si è rascurato. E quindi questa è la pieza di transizione, è costruita la probabilità di transizione per quale minia sta al quadrato, diventa semplicemente un quarto di x diviso di z al quadrato. a questo punto sostiamo il sostiamo il sostiamo il sostiamo il sostiamo il e si va bello io quando scrivo così devo dire da da andam ma per la di prima è la stessa la chiede il secondo mondo giusto perché la dice transition probability infatti scusi la convusora la convusora la quanto 10 9 10 9 10 con il tempo di Si, è talmente ben, perché rimasto costante al valore finale, cioè al valore attivo e piccolo corretti grande. Quindi dobbiamo calcolare le correzione energie, però sta specificando the first non zero correction. Quindi cosa facciamo? Per quella primordine, la 0 è possibile. Perfetto, quindi se non ci pensiamo a questa cosa ci calcoliamo quella primordine e troviamo che è 0. Però alla fine, per simmetria sono miele, quindi possiamo anche capire fin da subito per il fatto che abbiamo Pz e Vx. Però, diciamo, se uno non solo ricorda questa cosa, cercare di utilizzarli di scorsi per la prossima metria, che bisogna significare e velocizzare, diciamo, i calcoli. Se uno non solo ricorda un calcolo, è qui, scoprata, che è la corretzione di energie al livello generico N,
   
   fa più in ordine, se vi ricordate, ormai in quelli tempi che non lo facciamo, ma in maggiori, poi di essere esercitati in questi settimane, è uguale all'elemento di matrice del potenziale, dove il potenziale stiamo usando la perturbazione soltanto, tra gli stati imperturbati, quindi, sozzero, di quanto che è stato imperturbato, e lo stesso ordine che si deve dare della correzione. E quindi, per lo stato spin down, ora questo è il spazio, il spazio il spazio lascio mostrare i vari fattori davanti, è ora uguale a spin down, SX spin up, questo scambia, fa diventare spin down, e, ma non ma di questo scambia. Quindi abbiamo per questo vero al fatto che nel caso imperturbato, che era proporzionale a sigma z, avevamo gli autoclopo valori, che erano qui, appunto, di agonali, mentre SX, abbiamo visto che, in legge, nella base di SZ, ha assolutamente quelli off di agonali, quindi, non è vero. A questo punto, quindi, quando diciamo inferso in zero correction, vuol dire che dobbiamo andare a guardare al secondo ordine. Questo è Questo è Questo della scambia. Quindi, al secondo ordine abbiamo che D è con ente, al secondo ordine è uguale all'elemento di matrice del potenziale della perturbazione tra F con ente in zero e F con ente in primo ordine, che è quello che quello diciamo, quello che ha l'obiettivo del potenziale di l'obiettivo del potenziale di un qualcolato. E questo qua è uguale a, sanmatoria, su K è un rapporto diverso da N di si K0 per di KN diviso la differenza di energia tra i due stradi perturbati, del messico è K, quindi, VKN è uguale, diciamo, alleguo l'ente di quello, però tra N e K, TN0, V, si K0. In questo punto ci raccoriamo questo, lo inseriamo uguale a, H dagli A2 è 2Mc, Vx, H, Smax, F, da una, quindi uguale a, H dagli A2 è 2Mc, V. Alla fine lo stesso di prima per Tb, uguale a T grande. Lo stesso, ah, lo stesso, C, di Z. Calcoliamo questo valore, perché se qualcuno non lo chiede, dato che poi ci chiederà di calcolare la correzione di valori in elettrovolto, ci sta dicendo i sostituti di valori, confreddiamo questo è quello tra l'Issa di Invertura e Varia, però calcoliamolo in elettrovolto, così quello confrontiamo con quello che ci vede di calcolare. Ah beh, ma si capisce menta da qua che lo controllano Si si si Perché altrimenti avremmo 70\% di mero, se fosse un carico RQ o ZM Noi abbiamo il campo in vault E dobbiamo molti.. quindi moltiulere la carica in cedano che da mia da era la commersione Ci devo scrivere esplicitamente l'alvatore della carica dell'elettroengoso per passare dal tontono e per il campo in volta otteniamo prendere il campo e bz diventa il campo in il controllo per questo è il colto su metro e bz diventa un stesso valore dentro un colto su metro la del colto del metro e quindi 10 allo? si 2 per 10 non è lo 7 si c'è il colto ok giusto dobro un achive nemmeno a che cazzo per favore 5 a 5 se ne va, non giusto c'è questo però c'è la mia domanda ecco questa è in dubbio io posso scrivere che 5 per 10 alla 6 volte su metro sono uguali a 5 per 10 alla 6 e 3 volte su metro se scrivo e bz però uguale a questa cosa si, se e bz in colto su metro e bz posso scrivere questo ok chiaramente ci vuole la live qua certo e per bz e andrò a fare il mio enorme si proprio la definizione di letto in volto non è che l'NG ha una carica che fa una differenza di potenza di un albino automaticamente è risultato adesso riuscirò il 12 minuto riusicano a mettermi in via quando vi hai atto? forza 2 per 10 quando mi müccide non sei?
   
   si ne non sei? Si sta parlando di campi molto piccoli perché il campo alla fine non è molto poco. Si 1-1, quindi qua stiamo considerando lo stato non è più, ecco attenzione questo non è più lo stato subito. E' stato su più una correzione con down in realtà. La perturbazione mescola livelli, cioè gli stati, quindi infatti qua bisogna far attenzione perché uno si confonde, uno qua mette il tipo spin up, al primo ordine e poi non lo risulta, quindi uno che ha stato eccitato al primo ordine. Ma per sé se mi scusi, viste che io personalmente l'ho mai fatta esplicitamente, nel caso in cui abbiamo un campo diretto lungo la SZ penso positivo. Lo stato fondamentale sarebbe quello up perché c'è il segno meno nella miltonna e quindi c'è un dermi energia, o mi sto confondendo io. Cioè la mia domanda è il ground state, il primo stato eccitato, visto che questo è un sistema di oliva. Cioè se l'energia ha quello minore il ground state, quello con il GPL, lo dico solo comunque, da questo lo capiamo. Ok, vado a fermarmi il condomone in componenti quindi. Ok. Allora questo è uguale a H tagliato 2 mcvx, diviso H tagliato mcv zc0, uguale un mezzo x, zspin down. Quindi questo è la first-order correction to the x-line states. E' essendo questa la correzione, se io calco sin uno fino al primo ordine, questo era quello di mezzo il vado più questa correzione. In questo senso qua veniamo che c'è il mescolamento. Ok, quando consideriamo veramente la SOMA, perché è una risponsiata intentativa, quindi questa è la correzione. Questa è la correzione. Questa invece è proprio l'energia fino a tale ordine della correzione. Ok quindi qua vediamo il mescolamento. Con un po' vediamo il peso, no questo è un peso, uno questo è un peso, un mezzo, un biz e un biz. E poi facciamo lo stesso per l'altro. E quindi poi quindi ddown up uguale, down v up uguale a, uguale a tagliato e 2 mcvz. E' una cosa di differenza di energia, ovviamente è la stessa però, con segno meno. E quindi si, il ground state della correzione al primo ordine è uguale, ecco che è fatta con le sostituzioni perché la stessa è indietro, x di zeta, si è aperto dello stato. E quindi il ground state, questo è uno stile, questo è l'exide state. E ora stiamo parlando del ground state. E quindi up to first order di ground state è uguale a, spin down, mezzo, mezzo, di x di zeta. 
   
   E' svinato. A questo mi dò che siamo qualcora la correzione perché abbiamo il ketten primo ordine. Quindi terza è 1. La correzione ha lo stato accidato in un secondo ordine, che uguale a si primo zero di si primo uno. E uguale a, a tagliato è mcvz. Si uno zero, si ma x, si uno uno. E questo fa, a tagliato è 4mcbx4 vincel di zeta. Calcolate poi io, se tu il valore, che soltanto lo che lo richiede. Mi capito che tre letto in voco questo è zero perfection. 10 e la, hanno 10. 2 per 10, 11 per 10. E l'altro. Quindi come vedete in una correzione come ci aspettiamo, è piuttosto piccola rispetto al valore che è radiceo, che è la m6 della differenza di livelli. E quindi ora, chiaramente questa è la correzione allo set accidato. Ora, calcoliamo la correzione allo set add plus state, è il zero al primo ordine, che è c0o0 potenziale, c0, c0, 1, 4, c1o, e questo risulta il meno, quindi la stessa cosa è proprio il meno. Quindi mettere prima la differenza dei livelli che l'ho cancellato, però va bene, è quello che quando ho piaciuto, la cosa che vi ho fatto al primo era la differenza tra i livelli, quindi avevamo qualcosa. È accadriato sulla mc per l'0, quella Ok, e per lì quel valore che era 10 a meno 6, era proprio questa differenza, vi pare che vi ho detto. Sì, sì. Perfetto. Quindi inizialmente abbiamo un livello 0, un livello 1, la cui differenza è quel valore di circa 10 alla meno 6, eletto un volto che avevamo calcolato. Ora, cosa è successo? Che lo stato fondamentale ha avuto una correzione negativa, quindi da qui c'è spostato, e lo stato eccitato ha avuto una correzione positiva dello stesso quantitato, la stessa gambezza, quindi si è alzata. E questa qua è dell'ordine di 10 alla meno 10, vi ho detto, mi pare, 10 alla meno 10. E di questa qua è dell'ordine di 10 alla meno 10. Ok, quindi questa è una cosa generale che trovate. Quando andate a recordare i livelli, i livelli si allontanano sempre. Quando guardate la correziazione d'ordine, ho studiato di trovare sempre, se non l'avevate trovato, c'è qualcosa estravole. E questo è collegato al terreno di croce livelli, che i due livelli non si crociano mai, quindi andando avanti, cosa dà la ricaricolazione. E' quel discorso del fatto che gli stati sopra tendono ad abbassare, quelli sotto tendono ad alzare. Sì, sì. Domanda? Solo una. All'inizio quando è capitato, ha fatto l'elenco delle formule, le pre-elezioni, cioè è quello che abbiamo usato al corso di teorica. Semplicemente c'era la formula accesirica, in questo modo, quella per lavorare sia al secondo ordine. Perché abbiamo preferito agire così, calcolare l'astratto corretto e tutto. Ah, così? Cioè questo è la stessa condattua Non è capito, è la stessa condattua quando è fatto Sì, sì, no. E quello che dico io è che la si considerava il módulo 4 degli elementi e poi la differenza, che è la stessa perché si ricavano così teoricamente. Però perché non abbia applicato in attente la formula? Ci siamo Ad esempio, qua abbia qualcosa di l'astratto corretto. Forse vi so Cioè il módulo 4, diciamo Cioè questo qua Allora, non capito, qua questo come lo conosciamo. Sì. Eh, non capito il módulo 4 di cosa. Se poi sostituiamo Volevo vedere se mi Sì, noi abbiamo messo questo. Cioè l'astratto corretto è questo, giusto? Ah, ok. Cioè perché non abbiamo usato questo? Sì, sì, sì, sì, sì, sì, sì, sì, sì, sì, sì, Ok, ok, direttamente. No, perché vi volevo far vedere, siccome qua chieste soltanto effettivamente la quadrazione di energia, però siccome non ne faremo altre di questo tipo e può capitarvi, ma che è un esercizio qui della correzione al cat, l'ho voluto fare così in modo da vedere anche come si fa il tattore della correzione al cat. Però, sì, uno lo può fare direttamente in questo modo perché sono collegati, certo, sì, sì. Dovranno D'altra domanda, tutti? Eh, va bene, va bene. No, facciamo una foto, non è un fatto, ci vediamo tra un battu d'orale
\end{soluzione}

\newpage
\setcounter{equation}{0}

\begin{esercizio}
   Given a proton that scatters on the potential
   \begin{equation*}
      V(r)
      =V_0 \frac{e^{-\alpha r}}{\alpha r}
   \end{equation*}
   with $\alpha=2 \; \rm fm^{-1}$.
   \begin{enumerate}[label=\alph*), leftmargin=0.6cm]
      \item Calculate the phase shift for $\ell=0$ and $\ell=1$ at first order of the Born approximation in the low energy limit and the corresponding total cross section when including both $\ell=0$ and $\ell=1$.
      \item Calculate at what beam energy (in MeV) the cross section associated to $\ell=1$ is equal to $1/10$ of the one associated to $\ell=0$.
      \item If $V_0=20 \rm \; MeV$ is the Born Approximation valid in the low energy limit?
   \end{enumerate}
\end{esercizio}
\begin{soluzione}
   Non abbiamo un dolopranza, ce ne potete una modova cabita. Non c'è spazio, non c'è spazio. Non c'è spazio, non non spazio. Non c'è spazio, non non non non non spazio, non c'è spazio. Non c'è c'è non non non Non c'è c'è non c'è c'è  All'MK1 la VK1 invece l'avevate fatta. Guarda che il prossimo dia c'è il prossimo di che non è distante, lo facciamo tutta la VK fatta.  Allora                                                                                                                                questo c'era però non ha un fermi alla meno uno ma allora non sapevo cosa facciamo, fatelo voi io vi seguo, facciamo così l'avete visto tutti? no facciamo un momento e me la metà metà di cose le ha fatto ci ha fatto un strame insieme così risulta come esercizio allora, quello è all'entrotto che tipo di potenziale abbiamo? giocalo giocalo sì, è un potenziale centrale cioè, spericamente similettico qual è il potenziale? uno suo alto uno suo alto, quindi range del potenziale per B, circa un po' uno suo alto e vediamo vabbè, uno su due fermi 0,25 ma questo è un pettù, sì? allora, ricordiamo un attimino perché in scattere in teorile conseguiamo più di più ma è stato un gilio di svoluzione, sono la Pasha Wave Analysis Analysis la Pasha Wave Analysis quando è che usiamo in genere la Pasha Wave Analysis? non abbiamo alcun tratto, dovete provarla di la Pasha Wave Analysis intanto perché sta parlando di Fesh Shift e poi già tutto accendo all'OEG Limit ricordate? penso di averlo accennato perché non l'abbiamo usato in qualche senso la Pasha Wave Analysis è usata usata fare quando abbiamo un potenziale similettico si si a che che andrà quindi la Pasha Wave Analysis è usata quando abbiamo un potenziale centrale quindi a metterli a spherica e questo è, dico, ma è inoltre ci sono dei casi in cui è molto utile guardando anche il testo sta parlando di Fesh Shift, sta sta di l'e uguale a 0 l'e uguale a 1 1 Pasha Wave Analysis è usata per per l'e uguale a a e a a per andare all'OEG Limit per andare all'imite dei bassini cioè quando abbiamo l'imite di basseleggia è utile perché? perché considera quanto lo è il limite dei Fesh Shift? perché Contribus Dominal contributo dei termini con L più basso il spesso già con solo con L uguale a 0 otteniamo un'ottima approssimazione del risultato in questo caso è stato includere 0 e quello successivo, 1 ma spesso ci si ferma anche a 0 quindi nell'OEG Limit la Pasha Wave Analysis è utile oltre che applicabile in questo caso per se per potenziale semplicemente aspetto nel caso invece di quando possiamo usare la porna post-emission cosa abbiamo per Fesh Shift quando utilizziamo la porna post-emission? l'angelesio utilizzano per esempio quando il potenziale è debole e in questo caso perché cosa abbiamo in Fesh Shift? questo lo avevo detto sicuramente la seconda volta con del potenziale debole avevamo detto che per onda che queste onde nella fascia popolianalisi si possiamo studiare indipendentemente come scattarono le onde con diverso L siccome c'è la conservazione della probabilità non cambierà il modulo quadro dell'onda e cambia solo il Fesh Shift e quindi se il potenziale è debole questo Fesh Shift 
   
   cambia poco la fase cambia poco e quindi Fesh Shift è piccolo ricordatevi di queste cose perché spesso in un certo momento non so questo se ci serve questa cosa specifica però quando bisogna di domande in cui chiede explain anzir sono proprio questi rinferiti che ti servono, cioè i limiti in cui si può applicare una certa appossimazione o di fuori della guardia non si può applicare quindi ora utilizziamo questo qua allora noi abbiamo quindi la richiesta di calcolare il Fesh Shift e abbiamo un potenziale quindi quando abbiamo genera già subito pensiamo alla formula integrale del Fesh Shift che è proprio la formula in cui potenziali risultà esplicito e questa formula nel caso generale rimarrivo a Tossimone e questa è elevato a I del Pelle, S è elevato a Pelle uguale a meno 2M su H della do quadro multiplica K in generale del zero è finito in R di R JL per KR i potenziali risultano di R QL KR dove queste sono le soluzioni della parte radiale delle conzolatione e di mele queste sono le sveglie al bestia faccio ok? questa è la formula integrale del Fesh Shift in generale ora come possiamo appossimarla? possiamo approssimare quindi quando approssimiamo ci sta dicendo il potenziali è piccolo, il Fesh Shift è piccolo quindi è elevato a I I Pelle e questo è il giusto è è è uguale al del Pelle e poi c'è un'altra cosa che possiamo fare, che abbiamo già usato G con L, C, P, S sono rientri a G no, agli scorsi quelle per il L è la G limita però possiamo collegare questa G con L quindi questo questo è uguale a R vergebamente quindi queste cose qua vale perché siamo in una formato approssimation questo è il discorso non posso applicare le cose, ma capire quando si posso utilizzare e qui a questo lungo con queste trussimazioni attengiamo del L, C uguale al meno 2, M, a caldegliato quadro tra l'ultimo integrale del 0 e quindi quindi quindi R, R quadro G, L, a caldegliato quadro da R, V, R ora appliciamo quello che sta dicendo il vostro bolleghe, invece che si come siamo G, L, G G possiamo usare una forma per G, L che è G, L, di da P, R circoguale a K, R elevato all'L diviso 2L più 1 doppio culturiale che avevo già detto ricordato qua serie del culturiale la scorsa volta questo è è certo certo questo vale perché siamo in low energy limit e la non-energy limit, che significa? KRB KRB molto meno di 1 cosa te che vi rende quando ne parliamo di piccole, grande lo low energy passa ma sempre riferimento a quattro salto quindi confrontiamo il K che è legato alle regge della particella con range del potenziale che cosa sono i suoi altri? ok, sostituendo questo punto e sostituendo anche l'espressione per PR proprio quando c'è un passaggio di meno QM a caldegliato quadro K elevato a 2L più 1 0 a infinito R elevato a 2L più 2 più con R 1 su 2L più 1 doppio culturiale tutte le cose ho raccolto come vi è venuto in questo caso caso tutti l'avete fatto un esercizio con la mano a questo che c'è un sottile sottilizio però l'avete tutti con quella raccolta di esercizio su sottili che non c'è la macchina che le colleghi in modo da esercitare più ulteriormente per le tante quindi questo è uguale a meno 2M K elevato a 2L più 1 doppio culturiale a 0 a caldegliato quadro 2L più 1 doppio culturiale a caldegliato quadro integrale con 0 a infinito in D elevato a 
   
   2L più 1 elevato a meno a 0 ok se se fosse manca un'alpha dell'uminatore sì, questo grazie ok, ora qua c'è un ninto un aiuto integrale da 0 a infinito dr elevato a n elevato a meno r di 0 a 0 in dr uguale m fattoriale meno r utilizzando questo aiuto stiamo velocemente in volto del questo integrale e otteniamo meno 2M K elevato a 2L più 1 d con 0 diviso a caldegliato quadro a alpha di 2L più 1 doppio culturiale 2L più 1 fattoriale 1 su alpha elevato a 2L più 1 e più 1,2 e vai in fattoriale nell'interno, cioè non è e cos'è e 0 a la e l'ho capito e dopo l'uguale e che ci fa e cos'è quello che quando risultato non deve diventare anche letto per il 0 ah, che risultato si, non ho ho che risultato sbagliato, ho copiato e questo qua si, grazie e questo qua vuole provare ma non c'è un potere non ha senso che risultate che diventasse ok quindi c'è anche un piatto piatto lo scrivono nella forma utile meno 2M di con 0 elevato a caldegliato quadro più 1 su alpha elevato più 2L più 1 più 1 per il il il più 1 per il potere potere alpha 2L più 1 ora questo qua già avevo fatto vedere l'altra volta da questa dipendenza di questo tipo c'è una dipendenza capa per il range del potenziale elevato alla 2L più 1 e quello che vi doveva aspettare quindi non l'ho già fatto notare l'altra volta quindi se non lo trovate controllate bene è il il che è meno che non c'ho un motivo un genere c'è e quindi poi per L uguale a 0 dobbiamo alcolare per L e 0 e L 1 per L uguale a 0 abbiamo quindi la delta 0 proporzionale a r k r v detta 1 proporzionale a k r v alla terza che sono quello che vi aspettavo ora si avrò questo è un esplosione delta 0 uguale a meno 2m v 0 k diviso a cadegliato quadro alfa punto e delta 1 uguale a meno 4m di con 0 k 1 diviso a cadegliato quadro alfa alla quinta la sezione di tutto che è quello quello chiede il problema anche usando entrambi i contributi allora per venire alla sezione di tutto con i pesci f d 4 y k quadro sommatoria L d 2 L 1 delta L4 questo è più generale quindi tutti i contributi questo caso dice che è veramente includere i primi ma per quello che abbiamo detto che è quello quello rigirimente ci aspettiamo che sono i primi bassi quindi il domino quindi per L uguale a 0 e 1 1 questo caso quindi abbiamo quadro y k quadro delta 0 quadro più 3 delta 1 quadro ora questo qua ci serviamo nel punto dopo quello che dice prossezione associata a L uguale a 1 prossezione associata a 2 in quasi 0 cosa sono? sono questi no? c'è questo sarebbe quella associata con L uguale a 0 e questo sarebbe quella associata a L uguale a 1 quindi si immagina in 1 ok? sopra le quelle che ci servono nel punto successivo quindi tutte e due due danno quadro y k quadro a preparazione così 2 m p0 k diviso a quadro alpha cubo tutto più 12 m p0 k cubo diviso a quadro alpha la quinta tutta l'altra questo è il risultato nel primo punto il risultato è alquale alquale alquale per L4 in 0 e L1 questi limiti e le due corresponde il prossezione tutta il prossezione per uno potrebbe scrivere un altro modo mette da affattare alcune cose però che portare come? manca un 3 allora il 3 c'è prima proprio l'ho messo dentro si ha stato in un passaggio c'è il 3 che poi c'è un 3 qua e un 3 dentro e un 4 dentro però il 3 è alquadrato e quindi praticamente entrando dentro così allora posso posso allora si giusto? c'è il 3 a molti di carica metti dentro e fai le 9 questo è dentro una potenza non gli intera dice ma se metto i dentro una potenza non gli intera dice? non non sbagliato allora facciamo aggiuntare il 3 l'ho porto fuori 3 4 si cancella rimane un 3 e quindi rimane non lo ha quindi quindi 4 13 si sbagliato giusto 4 13 si si giusto? si ok fai il B si e chiaro così dovete fare ricordate quello che ho detto prima è sigma 0 la perception associata è l'equal a 0 questa è sigma 1 e quest'altro il fattore per 3 io dovrei forre conoscere un 10 un 10 che che questo? 
      
   C'è un'altra cosa? C'è un'altra cosa? C'è un'altra cosa? C'è un'altra cosa? C'è un'altra cosa? C'è cosa? cosa? C'è un'altra C'è C'è un'altra cosa? un'altra un'altra cosa? C'è un'altra cosa? C'è un'altra cosa? C'è una partita di questa? Intanto. E non ti viene questa cosa? Ah, spiaggia. Non mi esplicitino in quel modo. Scusa, come mi scrive? Nel ton 0 e del ton 1? Che formulosi. Questo. E non sono Nel ton 4. Non sono quelli? Non sono quelli. Che cosa? La vervizia. Che siano tu Ah, come è corriscata H4? Sì. H4. C'è una volta che trova K. Ok, non la virovo. Ma così vado indicazioni potete continuare se è che è come va? Sì, è tutta plus. Trovate anche il valore non mai riposate? Eh io trovo Tutta scumco web? No, no, qua avevo Alpha qual è il verbo. Ah, infatti è che Avevo collegiose un web, e ho trascente. Un verbo tutti più, potrebbe essere spagliato io al site. Un verbo tutti più, per più, per più, Web? Il verbo è molto creato. Tutto tutti più, per quanto vedi All'altro io ho ritratato un morticone e zeglio, per quanto sono spagliato, a controllino. Poi cosa aspetta? Cosa ha detto lei? Un web è 37. Quindi comunque voi non siete consistenti, tra di voi. Che per esempio, m'hai ieri con quello, con Alpha, con un fermo ieri, ieri a 5, me lo Ma lasciate stare quello C'è sé dall'esame che chiedite, ah siccome ho fatto le secchizze a casa in cui c'era l'ora, ma te l'ho nuova? Quattro, tre, tre, quattro Quattro, tre, quattro Però diciamo, a parte da parte finale Sì, non ve lo dice, ce l'abbiamo. M.C .4 e Un Java, quindi 10 e 3 Mb, no? Come vale? Ok, però dico, a parte la radice di Tre che mi era fuori da quello che vi ho detto prima, tutti gli altri nuove gli possono scrivere, quindi arrivate comunque a quelle espressioni. Qua? Si trova il numero, è legato, è 2,2, no? Vedi 3,5. Vedi 2,5, no? Ah, mi chiamo voi siete consistenti, no? Quasi ne ho cercati, eh, non non Non mi è fatta, non mi è fatta. Perché c'è, però Alpha, ah quanto vale, Alpha? Quattro per 5, 4, 5. Alpha, ah, S, me la numero 2? M.C .4, no? No, non più. Per 12, 120, diviso radice di 30. De la nera la radice è quarta? Allora, lo facciamo insieme, magari ho sveletto io. Mi ha facendo invecemente, vera venuto 4 radice di 3, va 10 allo 2, quindi viene comunque quello che ha 10 allo 2, però non posso sveleto volto. Allora, ok certo, certo. Facciamo insieme, eh, guardando, abbiamo un antio, abbiamo riuscito a fare un volto medico, si credo che lo accordiamo, se non lo tutti rimandano. Allora, abbiamo, ok, un malzero per quale A4, greco, diviso, K quadro, che è moltidica, M, di con 0, K, ma qua, non ho certo, io non ho fatto il quadrato, questo è quadrato, ok? Ma io svegliato sicuro. Questo è quadrato, quindi questo diventa 3, un po' quadro, quindi si accorta, per l'opera, ho fatto il direttore in tecora, quindi 2, M, di 0, K, ha capagliato, al quadrato, questo è la sigma 0, la sigma 1 è uguale a 4, di 0, di greco, K quadro, 4, M, corregiatemi se è più male o male di 0, poi la capo a 2, diviso, radice di 3, ha capagliato quadro, alza la minita, tutto al quadrato. Ora, qua dicevi, sigma 1 deve essere, un decimo, di sigma 0, ovvero, 4,
   
   di greco, K quadro, ok, questo qua era, a parte quel fattore era, cioè tutto questo è delta 0, e tutto questo era il 3 delta 1, giusto? 1 quadro. 4 più K quadro, delta 0, è uguale a 1, 10, 4, di greco, K quadro, 3, delta 1, quadro. Questo è questo, quando via, quindi abbiamo che, delta 0, questa 1, tutto levate al quadrato, è uguale a 3, 10. Per partire da questo. C'è l'alcantato, ma indietro lo fare così, eh? Ma non è, cioè, in un decimo, la molti implica, sigma 0. Sì. Però, scuittene, che, la deve essere, quella la, in quella la c'è scritta, che sigma 0 deve essere un decimo, di sigma 1. No, ma può, scuittene, è sempre, quel 3 va in delta 1. Ok, la bella, facciamo, se rieviamo il delta 1, mettendo la mia schizzo, la mia schizzo è quella stessa quella quel modo. Ovviamente questo è il ruolo, ci aspettiamo, perché, sigma 0 è quello dominante, quindi questo è più piccolo. Ok, allora, questi qua e questi qua, quando via, quindi risulta 2m, di col 0k, h tagliato 4, alto al cubo, tutto al quadrato, per quale un decimo, di quello che avvia, 4m, di 0k, tutto, per il vice di 3, alto al quadro, ok, tanto al quadro, sono stimo, poi, un attimo, questo è proprio dentro, un decimo, è proprio dentro, e quindi 10, ci la portiamo dentro, diventa 10, la dice 10, tanto, ok, ci lo dobbiamo voi per il momento. A questo punto, semplicitiamo, questo, e questo, va via. Questo, e questo, va via. Questo, e questo, va questo, Questo, e questo, diventa al quadrato, di 0k pure, 2, e 2, va via. Diciamo, ok, non c'è più trovare cacqua che qui, e ce l'ho, ah, già è al quadrato, quindi conviene che vada a fare la radice, che non ci interessa, poi, non c'è il punto, questo, giusto? Quindi, k, non è sbagliato prendere una radice nel terremi membri, no? Quindi k4, è uguale a la lucindola la radice, la dice di 10, ecco, quindi, la quadra, che cosa c'è, ma quando c'è niente, giusto? Sono uno, sì. Quindi, andiamo, la abbiamo, la dice di 10, la dice di 3, di 10, la dice di 3, di 10, e così c'è un metodo. Qua e l'altra quadra. C'è un metodo, ma questo 2, che diventa un mezzo, che non c'è qui, ok, e poi, affiammo alco, di questo stesso derivatore, che la dice di 3, che la dice di 10, e l'altra quante era, 2, quindi siamo 4 per mi, a 9, 2, e ora, eh, andiamo, che è uguale, k4, k2, diviso 2m, al solito c'è h, al quadrato, k4, diviso 2m, c'è al quadrato, uguale, 4 per 10 alla 4, dentro il quadro, per mi, quattro, per, che abbiamo incappato, quindi diventa la dice di 3, per 4, fermi la meno 2, diviso 2, la dice di 10, e per qua abbiamo 2, per una parte c'è la ruotone, quindi 10 alla 3, e per me, questo, questo va via, me, per me, va via, questo e questo va via, rimane 10, questo va via, rimane 2, 2 e 2, va via, mi sembra che, facilmente, non vuole andare via nient'altro, quindi rimane 1, quattro, la dice di 3, per 10, me, e il minimisore è giusto, perché è quella dell'energia, e sotto c'è solo il radice di 10, che quindi questo, rimane dalla dice di 10, e quindi ho 4, radice di 30, me, e per di 2, dovrebbe venire quel valore che ha vado, quindi quant'è, 20 di 2, 23, ma, si, domande, no, spettavolo, è finito, non lo manca l'ultimo, allora, il secondo punto che si chiede, ci dà un valore specifico di B0, e dice, siamo nel low energy limit, e quindi possiamo applicare, e possiamo applicare di qualsiasi, sul low energy limit, vi ricordo un'aldione del termino, che è il low energy limit, quindi la K, reddì molto meno, quindi K, alfa, no, K, uno su alfa, molto meno di uno, e a questo punto, ci abbiamo qui, K è la radice di questo, quindi abbiamo, no, alfa, uno su alfa, molto meno di uno, questo è questo valvia, quanto fa questo, o che fa una radice radice, perché io ho fatto i valori, prima che non sbagliate, eh, sicuramente è meno di uno, sì, sicuramente è meno di uno, però se ce l'avrete, lo facciamo così, che scrediamo? È, 0,5, perché per esempio, il errore che ho fatto prima non ero, e non ero, non ero, non ero, non ero, non 0,5, ci cazzo le cinque? cubico, ah beh, comunque è, minore di uno, diciamo, non è proprio molto vero, non è proprio vero, proprio siamo comunque, low energy limit, un'altra cosa interessante da notare, è che non dipende dal P0, perché P0 è scompasso, cioè non c'è proprio dal vendenso del P0. Quindi questo è il fatto che se parla di un olore di limit è indipendente al valore del potenziale e in questo caso risulta più o meno dell'olore di limit. Allora, la cosa che abbiamo trovato è che questo KRB
   
   molto minore di 1 non dipende dal valore del potenziale. Quindi questo risultato che sia un olore di limit in un olore passimersho non dipende al valore specifico che abbiamo messo qua. Quindi questa cosa può far compondere, perché chi è del valore specifico o meno, non c'è nessun vendenso dell'olore di potenziale. Stiamo parlando sempre della stessa energia di fascio, che è quella formula derivata dall'energia volundaria. Sì, l'energia di fascio è l'energia delle particelle che incitano. Il fatto dell'olore di limit è un limite riguardo l'energia della particella incidente rispetto al tipo di potenziale, in particolare rispetto al reto potenziale. Perché risulta? Le mie dubbi sono questa energia? Cioè questa formula questa veniva dal caso specifico di un visicomone, era un 10, non è un 0. Sì, sì, sì. Quindi il punto cifra di pericolare questa cifra. Sì, sì, sì, sì, sì. Non è una cosa che succede spesso, è un esercizio. La volta che diciamo specificamente, usando il caso, trovo da un B, altremottero non lo dice, ma è chiaro che dov'è stato sale questo. Anche perché se no Il fatto che non dipende dal B0 fondamentalmente, è che l'ha fatta bene in posto quella di B0. Sì, sì, sì. I teimenti Sì, sì, sì. Sì, no, sto riferendo, ok, nel caso che invece non fosse collegato al B. A quel punto non avremo nessun capo da considerare. Ci si rimarrerebbe una disecuzione per capo. Capo, molto meno di qualcosa. Però potremmo sfegarle in condizione di vize. Sì. Cioè utilizzando il Utilizzando cose. Un po' meno utilizzando questa capo di cimicere. Cioè non è che l'energia è un fascio di vendere al potenziale. Io sparo un fascio di particello. Sì. Quindi no. No, è un business, ad esempio, l'assiccione di urto. Quindi il fatto che potuto usare il risultato di Ponto B, lo capito anche da questo, perché non stai chiedendo sotto quali valori e valori di quali energiini mi trattavano l'acciestato a trovare il capo a molto meno di qualcosa e quindi non impuore quello di prima. Così come qua invece vi stai chiedendo proprio, è valido il loro energi-limit, quindi una volta non potete collegare il capo a compi 0. Per me non mi ero un modo di collegarlo, ma anche intuitivamente, ma non mi sembra abbia senso collegare un potenziale con l'energia e il fascio che so indipendente. E quindi per caso si capisceicarci di usare quello di prima. Che siamo sì, nell'energia di mimito? La richiesta non è verificare che sia valida la porna approssimation. Che sia valida la porna approssimation è l'oenergilimito, c'è tutto quello che abbiamo fatto prima. Devo dire il problema della porna iniziale, la porna integrale è il phase shift e l'avevamo potuto risolvere questo problema solo utilizzando tutte e due perché senza porna approssimation non potevamo ottenere semplicemente delta zero circogliale e senza l'oenergilimito non potevamo utilizzare quella forma semplificata per JL, per le best function. Quindi qua ti sta dicendo, si può essere spesso meglio, ma ti sta dicendo questa approssimazione di sovrappostimation del oenergilimito è valida e perché ti ha dato alla fine un valore di P zero? proprio perché in generale la porna approssimation si utilizza per potenze a deboli, quindi ha senso anche dutimenti a prima di mettere un P zero. Perché ti sta dicendo che nella porna approssimation limit sai che lo devi controllare così. Il P zero, lo sto dicendo, qua controlla se queste approssimazioni valgono. E qua alla fine si risulta che non dipende dal P zero. Non so se Perché di tutto il tuo trovato è nella porna approssimation, no? Si, ho capito cosa è quello di dire, ha senso la sua domanda. Questo no? Cioè, quel risultato che abbiamo applicato nella porna approssimation? Si, però ho parlato con il nostro collega, noi così non controlliamo che siamo nella porna approssimation, è vero, non stiamo controllando che la porna approssimation in gelo energi, cioè dobbiamo vedere così, è vero, non stiamo controllando effettivamente la porna approssimation, questo è giusto. Sì, però rivedeci, come siamo partiti da quello, dall'applicare porna approssimation nell'oregi limit, tutto questo non avrebbe senso se non possiamo controllare il gelo energi limit. Però è vero, per essere precisi non lo stiamo controllando, anche perché controllare la validata della porna approssimation significherebbe controllare come varia la funzione d'onda uscente considerando che se siamo al centro di potenziale, ricordate la condizione che abbiamo usato in qualche esercito, cioè bisogna usare quella e controllare che è che è la porna approssimation, però non è quello che chiede. Ok, posso capire che uno dice aveva, io ho il dubbio, e se invece mi sta chiedendo quello, io faccio tutte e due. Qual è la forma per cui questo è fatto? La ricordiamo. La porna approssimation è scusata, però è stato che ha scindisportato. Ok. La piu' si piu' IL 3° CERTA 3° CERTA si fa. C'è fa. 
   
   C'è fa. C'è Questo è il quadro. queste sono le funzioni che tagliamo all'atteria, quindi ci facciamo la porna approssimation. E questa quale la fine risulta essere, che è quello che devo scrivere, poi in pratica 2m lisa accatagliato uno su 4t greco integrale in D3 x primo elevato ai kx primo diviso in k il reprimo diviso il reprimo vi x primo minore di 1 e questo è questo e ci sono alcune le zone che hanno portato il libro lo che si raccolta è un peccato che vi accorre un peccato è il libro, fondamentalmente noi cosa si fa quando si fa con la potenza non c'è quello che è il matrice dove è approfondata la funzione completa con l'onda fiana facendo la volutazione tra la differenza tra queste funzioni d'onda valutando l'accentro del potenziale mini 0, questa è la zona dove il potenziale è più forte se questo risulta ovviamente il relativo, la differenza relativa risulta molto minore di 1 allora siamo nel caso di potenza del potenziale, per la born approximation mi pare che la non siasta in un esercizio la seconda lezione della terza altre domande? la cosa della born approximation alla fine la abbiamo utilizzato nel dire che il potenziale era molto forte sì, diciamo che non potevamo andare a volte senza poteniale alla fine di mande appunto a legge e del fasce da tutto il potenziale quindi qui è con esercizio stessa dicendo questo è il potenziale l'energia è quello che è il punto prima, vedi se il potenziale è molto minore di 1 perché noi da questa cosa vi è in effetti la condizione con i ammi alla stessa zona però posso capire che è vero un problema che apposso in questo collega che possorge e chi uno si potrebbe chiedere, ok sto colcolando che pare l'energia di vita ok ma che proprio sia pagata la born approximation, sì l'ho utilizzata per quella formula ma vale, va se il punto di vista è per un'unica storia una volta che abbiamo il viser di salto e capone, possiamo colcolare i delta 0, i delta 1 che abbiamo ottenuto con questa approssimazione se vi vedo se è per chi è, sono tanto piccoli ah ok, c'è a vedere che proprio riguarda il numero di delta 1, ok, sì, sì, certo certo, questo è un territorio riguarda 1, certo, sì, sì, giusto, buona idea per essere pagata al born approximation sì, anche questo ok, l'unica cosa che sta prendendo in mente è che dovete bisogna fare attenzione che discorso lo energie, i energy, perché mentre perché riguarda il potenziale delbore, non c'è, ok, l'energia questo sia bassa che avere un potenziale delbore e quindi posso usare sia il born approximation limit che il labor approximation per il potenziale delbore però invece la validata del labor approximation è anche quando sono nell'i energy quindi questa legurezza del genzi è potuta usare solo nell'one ogginine va bene, altri commenti, domande questo va bene, ci vediamo tra un battito e un po' di decino
\end{soluzione}

\newpage
\setcounter{equation}{0}

\begin{esercizio}
   An electron movers in a one-dimensional square well potential
   \begin{equation*}
      V(x)=
      \begin{cases}
         0 & \text{if } |x|<L\\
         \infty & \text{if } |x|>L
      \end{cases}
   \end{equation*}
   And is perturbed by an electric field.
   \begin{enumerate}[label=\alph*), leftmargin=0.6cm]
      \item Consider a weak uniform electric field of strenght $\varepsilon_0$ acting on the electron as the figure ($V(x)= e \varepsilon_0 x$). Use WKB approximation to calculate the energy levels, and find the minimum value of $\varepsilon_0$ to have at least one negative energy level for $L=3 \; \rm nm$.
      \item Consider now the case of a time-dependent electric field
      \begin{equation*}
         \varepsilon(t)
         =\varepsilon_0 e^{-t/\tau}
      \end{equation*}
      for $t>0$. Calculate the transition probability from the ground state to the first excited state in first-order time-dependent perturbation theory for timers $t \gg \tau$ and the value of $\varepsilon_0$ ad $L$ from a) and $\tau=10^{-14} \; \rm s$.
   \end{enumerate}
\end{esercizio}
\begin{soluzione}
   Accentrarmato, due. Ora La schiavo dopo. Per chi anticicò cosa dice dopo dopo ci sarà un tag di dependent electric field e bisogna calcolare la provveda di transizione. Quindi la cosa è una volta. Allora Questo filet d'edda, dai. 23, 23, 22. Allora, lascio iniziare voi. Fate attenzione. Ma forse è sempre che che che chi ci ci ci ci ci ci ci l'unico! Ora allora pure iniziamo con la parte per quanto fa fa Anche quando quando quando quando quando po' po' po' situazione è questa, che Se, come aspe, consideriamo quello che dice dopo, dice, può di calcolare il volo di erosso non zero per avere almeno un livello di ergeneggiativa, quindi diciamo che si trovi per come è fatto fatto disegno in questa regione, giusto? Quindi Si vede bene, quindi quindi esempio questa è l'energia, il volo dell'energia, e devo applicare la vkp approximation. La formula è La quantità e il conditione che serve per calcolare il livello di energia è integrale, considerando queste le mie militante points, quindi quindi poi ci sono le regioni classicamente morbide, dove sono le regioni morbide, di radice di 2m, a rivare, di se è, meno vx. Come devo considerare questa buca? Perché se vi ricordate, abbiamo tre possibili quantizzation condition, uno quando aveva una buca con due pareti, una con una parete e una una una una è con una parete, giusto? Questa è con una parete. Perché il menù L sarebbe da cioè per c'è infinito, no? Cioè il potenziale è infinito e poi ha un andamento lineare. Sì, per questo, cioè cioè ho una sola parete, sì, io considero l'energia qui. Ma questo, anche se io poi c'è il potenziale, va infinito dopo? Questo è perché sono interessata con la quantità negativa. Con quelle levenine energia dove c'è al meno un livello di energia. Quindi io, perché se volessi calcolare un'energia in questa regione, chiaramente due pareti, però se qui mi interessa, addirittura questa regione, comunque quando mi interessa una regione che sta sotto, sotto questo valore, ho sempre un turning point che dove poi la regione è infinito, ma la turning point è invece dove il potenziale è finito. E' un un un un un levenine. Occhi sono i tuoi negativi per le regioni. Sì, perché con le quali ma se non ci possesseva quella richiesta, comunque avrei dovuto dividere il problematico in due due Quello, qua sotto, più quello là sopra. In questo caso mi posso limitare questo perché perché ho fatto la domanda, diciamo, si capisce supplementarissima, ma che mi interessa solo questa persona. Svc, sì. In un un un però che intersega sempre la retto. No, la stessa cosa che faccio. No, quello stiamo facendo la vale, proprio in questa questa che sta in questo zone. Exacto, quindi comunque il primo turning point è in in questo lo dire. Tm-L, in Tm-L, sia in questa regione, in Tm-L, in in in Tm-L, Tm-L, Tm-L, Tm-L, Tm-L, Tm-L, Tm-L, cambia il turning point al destro, in Tm-L, in Tm-L, in Tm-L, in in in in in Tm-L, in in in in in in in in mentre la persona persona la regione superiore, non c'è bisogno di metterla in Tm-L. Sì, sì, sì. E quello che cambia non è soltanto il valore di questo, ma cambia anche la condizione di metto qua. Certo. Perché la condizione che metto qua ora, per considerare questa regione, diciamo in questi perergie in questo livello, in questa regione, è a cadagliato, Pichere, a cadagliato, Nm-4. E vi ricordo sempre che le estressioni cammi, in questo questo sto considerando 1, Qm, che parte da 1. Che se a volte ho trovato le libri, quale me ne parte a 0, quindi cambia il 2. Quindi, questa è uguale a integrale da Tm-L a x con 0. Siamo questo qua, x con 0, il turning point, da x integrale di 2m-e-e-x con 0x. E di meno il potenziale che dice chiaramente qua. Ok. Vi contate come Io ho detto in modo che la mia stessa sembra un sesso per risolvere questi integrali. Cosa fareste? La costituzione. E? Di cosa? Di che la radice è? Tutto ciò che c'è dentro la radice. Sì. Quindi, tutto questo non è uguale a una nuova variabile, la Gm-Z. Se qui lo tengo meno, uno su 2m-e-x con 0, integrale, questo diventa 0. E qua se invece 2m-e-x con 0 è il di z da radice di z.
   
   Tu vuole? E ci solle. Uno diviso 3m-e-x con 0. E come come tifto 2m, un tifto e u, e piccolo, e x con 0 l. Tutto le valo al tremendo. Quindi questo deve essere uguale a questo. E ora ora un po' di bastaggio di teoria. 2m-e-x con piccolo, e x con 0 l. Tutto le valo al tremendo. E ora, per me, diciamo solo l'ora. E le vuole 2-3. Tutto le valo al tremendo. 3m-e-x con 0, e e e piccolo, e piccolo, piccolo, piccolo, e e e e Tutto le valo al tremendo. E perché grande, io mi devo trovare Vittor, trovali i livelli di energia. No, quindi devo ottenere la sinistra energia. Mi faccio, faccio un po' di di di teoria qua. Quindi di energia, ora e con m, uguale, m, e piccolo, x con 0 l, più 1 su 2m, che moltiplica 3m-e-x con 0, di greve, attagliato l-4, e le cattove tersi. Vi avviamo atte del quesito a e risolto, o chiede appunto quel discorso di quello, nel c'è negativa. Quindi calcolare, e fangere il minimo del u x con 0, uguale da trist 1 nega di vera angileva, il foro non c'è attualo, orglielle. E quindi amminterezza il minimo di qui. Non le ho mai parlato alla fine, e con 1. Ricordo che questa è la prene, uguale 1, non è uguale per l'ecce, però. E con 1, uguale a meno e x con 0 l, più 1 su 2m, e 3m-e-x con 0, direzza attagliato l-4, le vado a 2 tersi, in tersi non lo vado a 0. Una una inseguazione, facendo un po' di passaggi che, magari, li li a casa, o otteniamo e x con 0, deve essere ammigioro uguale di 81, più greca 4, accadegliato 4, 128, m, l-5. Fate ora voi il calcolo per negli uguale di 1.000. Che ha senso, sì. Ma in realtà questa fa 3, giusto? Sì. Tanto, perché c'è un 25,3, la tua tersi? Cos'è? Mezzo mezzo, vero, non è così. Mezzo mezzo, penso, un 25,3, 16, un 0, è raggiore. Ma, beh, meglio. In questa linea, il trecento e vantottessimi, posso anche rivolire? Eh, qua? No. Cosa? No. Tu, poi devi, poi, mettiti sotto, così, che ti trovesse l'altro, un botto sul 9 metro. Un G4. Voi ho deciso, cosa? Alla linea. Prima. Aspetta, dove è il sonzero? Anche me, eh, non è il il No, ma in un vero 8,0, il 2 è il 3, il 3 è il 7, il sonzero. Cioè, il valore è, qui c'è lo questo, io diciamo, o uno di questi, o 8,7, però non c'è lo altro. 8,7 ,8, cosa calcolabile? R, il sonzero è un'elettroplastica. 4, il sonzero è il 3,5, non c'è l'altro. No, No, è in elettroplastica, in voltro, non è in gossium. Non è è è non è in gossium. 1, 1, 6, 6, 6, 6, 6, 6, 6, 6, 0, non non rispondere. Sono un'elettroplastica. Con Con Con 18,5 che è 16,4, fortale. Allora Alla quanta risulta? Un'altanze cinci, un'altanze cinci, un'altanze un'altanze un'altanze cinci, un'altanze cinci Cioè, con quadra? Con quello che tutti collega? Sì, ho ho sì, cioè mi viene Allora, il metro submetro è 18.5, in eletto, in volt submetro In eletto in volt submetro? Sì, cioè E' il metro in eletto in volt submetro mi ha dato? Sì, è di 18.5, quindi Sì, certo. deci alla scena di eletto. Giusto, agli altri risultato? 1,7 per 17 elettro in volt submetro. Sì. Dici alla metodo. Dici alla sette. Dici alla sette. Sì, sì. Siamo Siamo fare 188, 188, 188 Sì, sì, lo farò. Ma sono subito costo con un cista. E' 3 di chissà 6, 6, 6, 6, 6, 6, 6 Prove sess, ma in questi casi, agli elettori unità di misura divogono di troveremo campi elettrici, e biofero elettro in volt submetro, cioè se io riporti risultato in eletto in volt submetro, trovo abbia lo stesso Allora, se non è specificato, lo lo abbia lo stesso. Io per esempio ho fatto in eletto in volt submetro, se questo di volta non fa le coperci, non lo vedo. E poi spiuto, ho schipto le roline posti submetro, cioè ho schipto tutte e due. Se non è specificato, lo stiamo facendo. Chiaramente, mi aspetto però, appunto, elettro in volt submetro o elettro in volt volt Non mi aspetto elettro in volt volt perché è un un non non vedo. Dove Dove il nanometro, quindi Ok, prima cosa, cercate di fare questo. Se Se fermi, se avete elettro in volt submetro, non lo metrici, in questo senso, è tutto un po' consistente, un po' di sta fisico. Però, per me è spagliato. Ok, quindi quindi risultato è F110, uguale a 1.8 Ho schipto io per 10 alla meno 2. Per 10, volt submetro. E questo è il primo volt submetro. Pazzesco. Epson 0, 2 per 17, volt submetro. Ho visto quello del sistema internazionale, quindi per questa volta Ok, seconda domanda devo scriverla. B, consider now the case of a time dependent. E lecce of fit Epson dT, uguale a Epson 0, è elevato a meno T, diviso tau. Form T, maggiori di 0. Falturelte. La transizione probabilità. Dio del grand state. La prima stada excitante. In prima ordine, in seguito alla teoria di teoria per le tue volte, per chi molto aggiore in taw e per le tue volte, per l'extronde zero e per le tue volte, per gli ore Police expressing di pare. Allora inizia a trasvolgerlo voi. Per se se però, non una domanda perché qui l'altro è un'esercizio. Quando si interessa il transenter, il primo stato eccitato, si interessa quello dell'armamentale di esercizia, la perturbazione che avevamo prima, oppure vi devo calcolare i autofunzioni con il metodo VKT,
   
   per poi lo vattugare. Poi si dici che poi si capisce vary, dicendo considerate il now the case. Ok. Quindi, se in campo prima avevamo volo e fissu e ps.0, uccidentemente iniziava a avere un'altra squadra di tendenza temporale. Quindi c'è il ground state. E' quello dell'armamentale in Porto Bal, dell'armamentale d'Aula. E' quello del quanto un bel senso. Sì, sì. In questo caso, visto che la buca simmetrica, quali erano le autofunzioni? L'autofunzione è un'inspire, ma si tratta di uno svel, così si tratta di un svel, quindi si tratta di tratta di tratta si tratta di un svel, così si così si un svel. Qua otteniamo una Cnx gre ai, quindi questo otteniamo una, no, vieta rispetto al centro della buca. In caso di me lo so, sono così precisi. In caso di me di me di me di me di di me lo so, sono so, sono so, sono so, sono so, so, sono così precisi. così precisi. così precisi. così precisi. così così precisi. così precisi. così precisi. lo so, sono così precisi. In caso In caso lo so, lo so, lo so, lo so, me lo me lo me lo me lo di me lo so, sono così precisi. In precisi. In precisi. lo so, sono così precisi. così caso di me lo me sono così precisi. In caso In caso In so, sono così sono In caso di me lo me sono così sono così caso di me lo me sono così precisi. In caso di me lo so, sono così sono In caso di me lo so, sono così sono Per me è differente. Io gli ho detto più che altro, perché lei ha detto che aveva la necessità di finire entro questa settimana. Sì, per me è differente. Si pone per tutte le lezioni. Devo capire prima come è concorso a concorno con l'asaliazione. Perché lei giustando e c'è il tutto dopo l'epidemia fisica 1? No, alle 11. Quindi, scomai, l'unico momento, quelle sono ancora più importanti qua, se non avevano si potuto un occhio a fare un'occhio dopo la settimana. C'è destra, ma, deciso, a un livello di direzione. No. No, però No, visto, c'è il gioco, ho visto, pure, almeno il portavio già forte. Quindi, secondo me, qualcosa di rambio. Mi sono visto, è stato lasciato, si. S'ha intubbio, il ciso. Sono inserito, mi dico, perdono, allora Calcolato No, ma devo fare, perché io Calcolato, è un'incredibile, è un'incredibile, è un'incredibile, sì, sì. Ma, per tempo, è molto molto nuovo, che dà ossigenza che nell'integrale devo metterci più infinito o tie? No, tutti. La forza dell'integrale, se l'interesse può essere. Non lo so, ora che ti ha fatto interagere, se ti dico Quindi, è quella cosa. C'è una piggere da una volta che mi ha fatto. C'è una piggere da piggere da piggere da piggere da piggere ok? Ok. Duelle. Duelle. Duelle. Cosa? Qual è il dubbio? Sì, sì, sì. Cioè, il duelle è due metodi, perché la lunghezza della lunghezza è duelle. Viene più inizio, se diccierebbe essere, se diccientra, la lunghezza è la più grande. Cioè, il duelle il duelle il duelle se Certo, cioè, certo. Non mi rispettano questo momento, se me lo scrivo, fatto non me. E, in generale, deve essere due mellequadro, per in questo caso, essendo ci L4, che è la lunghezza duelle, quindi diventa due per quattro in questo. Cioè, se lo farò, Ma, diciamo, ci siamo con l'altro. Certo. Non lo fai con l'altro. Cioè, non lo farò. Ma, si fa? Tutti, la vero, parti. Ti va per parti, eh? È giusto così, vero? Sì, così. Ora devi Ti raccolto questo. Che spoi? Cosa? Ah, c'è un Ah, non c'è Ma non Devi prendere punizia con la cagola. Sono stupido. Che punto siamo? Cioè, c'è qualche Ti può iniziare a conseguire per lo d'anima nel stai boccato. Questi problemi in particolare, perché dubbio No, solo canzoni. No, va bene, no. L'importante è questo, perché se invece c'è un dubbio, non lo trovi da vedere, ma è un boccato di Calcoli quando meccanisci. Cos'è? La La monta a gritte. Ma non la monta a gritte, però ho detto a poce che c'è un minte, ma la monta a Ah, ok. E' la pericolata di A com'è? Pericolata di affamettia. Sì, c'è, è un linto. Eh? C'è un piatto pericolato di affamettia. Che cosa? La meno L, l'A, L. E X. Ah, sì, sono stupido. Sì, è un piatto. Sì, è un piatto. E X, L. E X, L. L, X. L, L. Cioè che poi si può fare, però, è tutto solo, è difficile che più Miancari, Fienti, fare Tutto il gel. L, 4. Gli 3 a 4. Quanto viene? 32, 9. 32, 9. Sì, era arrivato in qua. che si era proprio questo, giusto? Sì, è stata rica. Che è che è che Sì, sono sempre stupido. Ti prendi da come di Warner? Ma va, che è successo? Non è successo? Ma sono, sono da niente, la sua La sua struttura. La struttura. È un interno di questo tipo? Infatti gli hanno detto, bundi, perché la foto del Mende Benetim, è un giovane nuovo, che tu ci devi arrivare all'ultima casa, ed esattamente, se sei troppo bravo, torni indietro. Che? Non voglio che sei troppo bravo, ma è tutto tutto. Uff, e poi io ando. Siamo un bel uca per i bambini. Poi, la messa. No, vai. Assolutamente, siamo tutti d'accordo. Ti senti sempre poi dopo che sei un giardino spessore per le tue tante giorni, però è un interno di perdo, per il cacchio. Ok, allora Chi ratti? Ma quindi ti resta solo? Eh, eh, eh. eh, ne va da meno io, me che effetti, però ne va da meno a ti, su Tau. E questo lo vuoi fare tranquillamente? Sì. Cioè, l'integrale viene questo fattore. E noi, su H3, per Hepsion0, per 30 No, per Per Hepsion0 e T. Che noia. A me viene pure così. Se pari a cogli un meno, questo diventa col meno pure. Perché? Tutte due coi più, so. Cioè, se metti il meno, l'accontravantice deve essere il più. Eh, ne va da meno io, me che effetti e ne va da No, è No, no, me che effetti. Sì, sì. Fa una passanza sicura che si non più. Ah, no, raggià, no. Ho sempre torto, ma più non c'è una trenta. Un vodo. E non è sperminto, è C'è cosa? Sì, vero? C'è la città. C'è. No, non No, È la gente. Sì, sì, sì, sì, sì, sì. No, va be', parti un mix. 
   
   , che informazioni tramite dei canali in questo caso la unico modo che io ho potuto contattarmi sono l'avviso sì cioè io ho la me di una cosa che per oggi non dico beh metti l'avviso, oltre a giorno abbiamo visto sì, infatti va bene, allora che punto siamo? che sono che ho vinto comunque a risolvere però abitemi un po' la rispoltà così io capisco se sono in calco e ci sono le lezioni di un'altra cosa lele di estremi di integrazione vanno da zero ahhh e t più e t grigo non lo so non dovrebbe essere t generico t generico, sì perché? ah, t generico sì, t generico perché qua dice quindi ok, sì sì sì sì ok, ok allora risolvere ok perché poi se tu consigli, quindi questo è un po' prossimo cioè in generale, un'igel e io ho fatto una proprimitare che diventa al derro t così poi più però dice che ti dà considero che sono quelli in molti maggiori diri da o potendo una significazione in cambia non si fa l'imp yellow sì sì sì sì poi come un risolto, dito? eh no, i conti procedo nei conti, ok ma beh, qua non manca molto, lo faccio comunque però vedo che sono queste in i calcoli non sono informatiche, teoriche, progettuali. Allora, quindi abbiamo questo campo elettrico e quindi il potenziale di Pimo di T con la rettograzione time dependente, sono zero, possiamo chiamare, siamo scriviti in questo modo, se possiamo. Ciao. No, ma non ce ne esponete. È la transition. From ground state to first side of state C21 che posso che sempre 1, 2, così, se non ci lo scondiamo. Ma io lì sono accadagliato, integrare da zero al T e i omega 21 in Pimo V, no, si deve avere a 2, 1, Pimo, il Pimo, in Pimo, dove omega 21 che uguale a e 2 meno e 1 diviso accadagliato e uguale a 3 accadagliato Y quadro diviso 8M L quadro. Ovviamente, se mentre scrivo, vedete qualcosa che conviene, cedite lo subito. Sito e 1 che uguale a C2 e sono zero, dove c'è il potenziale. Di C1 Vedete qua è dove mi serve il fatto che stiamo considerando quelle nella poca centrata. Vedi chiaramente a più anche all'ermore, esplicitivo, questo qua è sono zero c'è da T e a una meno di F1 4 5 6 7 8 9 9 10 11 12 20 La teta la metto prima, perché sto dicendo questo qua si accende a T equals a zero. Per realtà, una volta che io comunque sto calcolando l'integrale tra zero e T la teta non si è più ok. Quindi X2 e X si uno. Quale? E, X2 zero è alla meno T su tau con Swell regalata meno LL nelle X, X su questo qua si trovano punzioni in greca X di Swell con servito in greca X di Swell a questo punto uso l'aiuto e diventa E è più sono zero e meno T su tau 32 e 9 il greco quadro LL perché ho tolto un L ma perché c'è questo qua? L e quindi questo è usuale a meno I su accadagliato e X on zero 32L di disso 9 in greca quadro integrable tra zero e T quindi T primo è elevato a I omega 21 meno 1 su tau T primo sto già per interiale con Swell meno I accadagliato e X on zero 32 di disso 9 in greca quadro LL è elevato a I omega 21 meno 1 su tau con I omega 21 con I con 21 meno 1 su tau a questo punto posso utilizzare quello che dice il greco che direi essere molto maggiori di tau quindi per quindi molto maggiori di tau questo diventa meno I su accadagliato e X on zero LL 32 9 in greca quadro tau diviso 1 meno I omega 21 tau e tutto così e quindi poi P meno 2 è uguale a sino 2 quadrado uguale a 32 e X on zero L disso 9 in greca quadro accadagliato tutto il quadrado tau quadro meno 1 su tau omega 21 al quadrado tau quadro e poi vi dà dei valori specifici di tau L era quello dato prima X on zero quello trovato prima e quindi possiamo calcolare questa probabilità allora se calcolate omega 21 la avete calcolato 3 accadagliato C e le calcolate C 8 M4 L4 L4 L2 sono decolificate perché in genere nel C utilizziamo 70 per l'H3D4 l'H4Mc4 N2 però rimane non pensate per insomma del valore a me risultato sono un paternone 0.5 perchè è a 14 secondo me la ne ho non la avete calcolato se poi compro le tre da casa quindi più di 1 sostituendo i valori come vedete 10 alla 14 sono 1.000 a meno a meno a quindi è fatta a posta quindi è fatta a posta per semplificarsi rendere i calcoli facili quindi più di 1 a me risultate 8.4 per 10 è una meno 2 per 1.08 che è 8.000 a 1.0 insomma è chiaramente nel periodo tuttativo valido i calcoli sono i calcoli per 8.000 volte quindi a parte modulo qualche errore che ci sempre però nel caso della testa di tutta questa 9.2 no va bene ci vediamo per la dite per il 14 a 14
   \begin{equation*}
      \int_{x_1}^{x_2} \dd{x} \sqrt{2m \bigl[ E - V(x) \bigr]}
      =\qty( n - \frac{1}{4} )\pi \hbar
      \qq{,}
      n=1,2,\ldots
   \end{equation*}
   \begin{equation*}
      \int_{-L}^{x_0} \dd{x} \sqrt{2m ( E - e \varepsilon_0 x )}
   \end{equation*}
   \begin{equation*}
      z=2m ( E - e \varepsilon_0 x )
   \end{equation*}
   \begin{equation*}
      -\frac{1}{2 m e \varepsilon_0} \int_{2m(E + e \varepsilon_0 L)}^{0} \dd{z} \sqrt{z}
      =\frac{1}{3 m e \varepsilon_0} \bigl[ 2m(E + e \varepsilon_0 L) \bigr]^{\frac{3}{2}}
   \end{equation*}
   \begin{equation*}
      2m(E + e \varepsilon_0 L)
      =\qty[ 3 m e \varepsilon_0 \pi \hbar \qty( n - \frac{1}{4} ) ]^{\frac{2}{3}}
   \end{equation*}
   \begin{equation*}
      E_n
      =-e \varepsilon_0 L + \frac{1}{2m} \qty[ 3 m e \varepsilon_0 \pi \hbar \qty( n - \frac{1}{4} ) ]^{\frac{2}{3}}
   \end{equation*}
   \begin{equation*}
      E_1
      =- e \varepsilon_0 L + \frac{1}{2m} \qty( 3 m e \varepsilon_0 \pi \hbar \frac{3}{4} )^{\frac{2}{3}}
   \end{equation*}
\end{soluzione}

\newpage
\setcounter{equation}{0}